\documentclass[11pt,a4paper,oneside,openright,titlepage]{book}
\usepackage[utf8]{inputenc}
%\usepackage[italian]{babel}
\usepackage{amsmath,amscd}
\usepackage{amsfonts}
\usepackage{amssymb}
\usepackage{natbib}
\usepackage{braket}

\usepackage{amsmath}


\usepackage{setspace} %per la copertina
\usepackage{graphicx} %per le immagini

\usepackage{color}		% testo colorato
\usepackage[dvipsnames]{xcolor} % testo colorato

\usepackage{hyperref}  %s

% DICHIARAZIONI
% operatore traccia
\DeclareMathOperator{\Tr}{Tr}
%commenti in rosso
\newcommand{\cmmnt}[1]{\textcolor{Mahogany}{\emph{#1}}}


\author{Riccardo Antonelli}

\setlength\parindent{0pt}
\begin{document}




\frontmatter


\begin{titlepage}
\begin{center}
 
% Upper part of the page
\includegraphics[scale=.5]{images/logoBlack}
 
\textsc{\LARGE Università degli Studi di Padova}\\[1.5cm]
 
\textsc{\Large Dipartimento di Fisica e Astronomia\\[0.2cm] Corso di Laurea Magistrale in Fisica}\\[2cm]
  
% Title

{\Huge \doublespacing \bfseries \begin{spacing}{1}{Holographic effective field theories: a case study}\end{spacing}}
~\\[4cm]
 
% Author and supervisor
\begin{minipage}{0.4\textwidth}
\begin{flushleft} \large
\emph{Laureando:}\\
Riccardo \textsc{Antonelli}
\end{flushleft}
\end{minipage}
\begin{minipage}{0.4\textwidth}
\begin{flushright} \large
\emph{Relatore:} \\
Luca \textsc{Martucci}
\end{flushright}
\end{minipage}
 
\vfill
 
% Bottom of the page
{\large Anno accademico 2015/2016\\
	\cmmnt{bozza compilata \today}}
 
\end{center}

\end{titlepage}

%\begin{frontespizio}
%\Universita{Padova}
%\Facolta{Scienze Matematiche, Fisiche e Naturali}
%\Corso[Laurea]{Matematica}
%\Titoletto{Tesi di laurea}
%\Titolo{Equivalenze fra categorie di moduli\\
%e applicazioni}
%\Candidato[145822]{Enrico Gregorio}
%\Relatore{Ch.mo Prof.~Adalberto Orsatti}
%\Annoaccademico{19??-19??}
%\end{frontespizio}

%\begin{abstract}

%\lipsum[1]

%\cmmnt{bozza compilata il giorno \today}\\

%\end{abstract}

\thispagestyle{plain}
\begin{center}
   % \Large
   % \textbf{Thesis Title}
    
   % \vspace{0.4cm}
   % \large
   % Thesis Subtitle
    
   % \vspace{0.4cm}
   % \textbf{Author Name}
    
   % \vspace{0.9cm}
    \textbf{Abstract}
\end{center}
The identification of the low-energy effective field theory associated with a given microscopic strongly interacting theory constitutes a fundamental problem in theoretical physics, which is particularly hard when the theory is not sufficiently constrained by symmetries.
Recently, a new approach has been proposed, which addresses this problem for a large class of four-dimensional minimally supersymmetric strongly coupled superconformal field theories, admitting a dual weakly coupled holographic description in string theory. This approach provides a precise prescription for the holographic derivation of the associated effective field theories. The aim of the thesis is to further explore this approach by focusing on a specific model, whose effective field theory has not been investigated so far. \cmmnt{(modificare abstract alla fine del lavoro.)}



\tableofcontents

\mainmatter

\chapter{IIB superstrings and branes}

\section{Superstring theory}

String theory either does not admit a nonperturbative Lagrangian formulation, or this formulation is unknown. An action functional can only be written upon choosing a perturbative vacuum; since we anticipate a string theory must include gravity, a choice of vacuum will also require a choice of background metric - in the simplest case Minkowski spacetime. With this choice the action for a string in the simplest case of bosonic string theory is the Polyakov action:

\begin{equation}
	S_B = -\frac{T}{2} \int d^2\sigma \sqrt{-g} g^{ab} \partial_a X^\mu \partial_b X_\mu
\end{equation}

where the $D$ fields $X^\mu$ describe the embedding of the string's worldsheet in the $D$-dimensional target spacetime, and the integral is performed over the worldsheet coordinates $\sigma^a = (\tau,\sigma)$. The $X^\mu$ are of course scalars from the point of view of the worldsheet. The auxilary field $g_{ab}$ is a metric on the worldsheet. The action displays worldsheet diffeomorphism and Weyl invariance, and thus perturbative string theory is naturally a two-dimensional conformal field theory. These symmetries must be quotiented out someway on quantization. The most straightforward way is to eliminate them by fixing a particular gauge and then quantizing (canonical quantization). The three symmetry generators can kill the three degrees of freedom in the metric to fix it to the 2D Minkowski: $g_{ab} = \eta_{ab}$. We get\\

\begin{equation}
	S_B = - \frac{1}{2\pi} \int d^2\sigma \partial_a X^\mu \partial^a X_\mu
\end{equation}

where indices are raised with $\eta^{ab}$.\\

There are at least two different approaches to introducing supersymmetry into a string theory. The path followed by the RNS (Ramond-Neveu-Schwarz) formalism is to impose SUSY at the worldsheet level; explicitly, fermions $\psi^\mu$ to act as superpartners to the bosons $X^\mu$. The action is extended to

\begin{equation}
	S = S_B + S_F = -\frac{1}{2\pi} \int d^2\sigma \partial_a X^\mu \partial^a X_\mu + \bar \psi^\mu \rho^a \partial_a \psi_\mu
\end{equation}

The spinors' equation of motion, the Dirac equation, is actually the Weyl condition in two dimension. This brings the real degrees of freedom in the spinor for each $\mu$ from $4$ to $2$. Recalling that in $(2\bmod 8)$ dimensions there exist Weyl-Majorana spinors satisfying both the Weyl and Majorana conditions, imposing the latter on $\psi$ halves again the on-shell polarizations to $1$. Thus we have a match between bosonic and fermionic degrees of freedom. It can be proven the theory above is indeed worldsheet supersymmetric.\\

To quantize canonically, we introduce canonical commutation/anticommutation relations:

\begin{align}
	[X^\mu(\sigma),X^\nu(\sigma')] = \eta^{\mu\nu} \delta^2(\sigma-\sigma') && \{\psi^\mu(\sigma),\psi^\nu(\sigma')\} = \eta^{\mu\nu} \delta^2(\sigma-\sigma')
\end{align}

Note the $X^0$ and $\psi^0$ would create negative norm states, but these modes are eliminated by resorting to superconformal invariance. Classically this symmetry imposes the stress-energy tensor $T^{\mu\nu}$ and the supercurrent $J^a_\alpha$ vanish; imposing that in the quantum theory they annihilate physical states yields the restriction that removes the longitudinal ghosts from the spectrum. These take the name of super-Virasoro constraints.\\

Then the procedure for building the string spectrum is to expand the classical solutions in terms of Fourier modes, identify creators and destructors, and then select the states of the Fock basis that satisfy the super-Virasoro constraints.\\

Boundary conditions for $\psi^\mu$ for an open string can actually be satisfied in two different by imposing periodicity or antiperiodicity, giving rise to the NS (Neveu-Schwarz) and R (Ramond) sectors, built over two grounds $\ket{0}_{NS}$ and $\ket{0}_R$. Closed strings have four: $\ket{0}_{NS-NS}$, $\ket{0}_{R-R}$, $\ket{0}_{R-NS}$, $\ket{0}_{NS-R}$.

\cmmnt{Review molto rapida dei creatori e distruttori nei quattro settori, formula di massa e spettro massless. Proiezione GSO}

\section{Type II superstrings and D-brane content}

\cmmnt{Proiezione GSO per le tipo II; chiralità dei fermioni; potenziali RR e p-form electrodynamics; D-brane corrispondenti. Magari T \& S duality?}

%\section{From 11D SUGRA to type IIB SUGRA}

At scales much lower than the Planck scale (equivalently: when the curvature radii are $\gg$ than the string size), all massive modes of a string theory decouple and a good description is given by an effective field theory comprising only the massless excitation. Since the string length goes to zero in this limit strings in massless states are essentially pointlike and the quantum theory will correspond to a local quantum field theory.\\

The effective field theories of the five superstring theories are the five supergravity (SUGRA) theories in $10$ dimensions. The name of each SUGRA coincides with that of the superstring theory it's the effective theory of (e.g., IIB SUGRA is the effective theory of IIB superstrings). Supergravities are supersymmetric theories containing general relativity. Just like Einstein gravity, they are nonrenormalizable, reflecting their origin as effective theories. As field theories, they are considerably simpler than general strings to find background solutions to; therefore we will make extensive use of the supergravity approximation in the context of holography.\\

$10D$ SUGRAs are perhaps easier to introduce starting instead from the unique $11D$ SUGRA. The field content of $11D$ SUGRA is as follows (number of physical polarizations in parentheses):

\begin{itemize}
\item graviton $g_{MN}$ ($44$)
\item 3-form $A_3$ ($84$)
\item Majorana gravitino $\psi_M$ ($128$)
\end{itemize}

As required by supersymmetry, the number of on-shell boson and fermion states are equal. These states form an irreducible supermultiplet, a gravity multiplet.

Upon dimensional reduction on a circle, in $10D$ these fields decompose into those of type IIA SUGRA:

\begin{itemize}
\item graviton $g_{\mu\nu}$ ($35$), Kalb-Ramond 2-form $B_2$ ($28$), dilaton $\phi$ ($1$)
\item 1-form $A_1$ ($8$), 3-form $A_3$ ($56$)
\item two Weyl-Majorana gravitinos of opposite chirality $\psi_\mu$ ($56$ each), two Weyl-Majorana dilatinos of opposite chirality $\lambda$ ($8$ each)
\end{itemize}

Obviously, again we find that the total bosonic states are $35+28+1+8+56 = 128$ and the fermions $2\cdot (56+8) = 128$. This is a theory with $\mathcal{N}=(1,1)$ SUSY, meaning there's two Weyl-Majorana (we recall again the existence of Weyl-Majorana fermions in $D=10$) SUSY generators of opposite chirality.\\

We will mainly be interested, however, in type IIB SUGRA, which is not obtainable from dimensional reduction, but rather is the T-dual of type IIA. The field content is as follows:

\begin{itemize}
\item graviton $g_{\mu\nu}$ ($35$), Kalb-Ramond 2-form $B_2$ ($28$), dilaton $\phi$ ($1$)
\item 0-form $A_0$ ($1$), 2-form $A_2$ ($28$), 4-form $A_4$ with self-dual field strength ($35$)
\item two Weyl-Majorana gravitinos of equal chirality $\psi_\mu$ ($56$ each), two Weyl-Majorana dilatinos of equal chirality $\lambda$ ($8$ each)
\end{itemize}

IIB SUGRA has $\mathcal{N}=(2,0)$ supersymmetry.\\

This net of relationships between SUGRAs in $10$ and $11$ dimension is actually the effective limit of dualities between string/M-theories of which these SUGRAs are effective field theories. The relevant part of the scheme is as follows:

\[\begin{CD}
\text{"M-theory"}     @> \text{dim. red. on }\mathbb{S}^1>>  \text{IIA strings} @> \text{T-duality} >> \text{IIB strings} \\
@VV\text{eff. th.}V        @VV\text{eff.th.}V  @VV \text{eff.th.} V\\
\text{11D SUGRA}     @> \text{dim. red. on }\mathbb{S}^1>> \text{IIA SUGRA} @> \text{T-duality}>>  \text{IIB SUGRA}
\end{CD}\]
\\
In both IIA and IIB, the RR sector admits the following gauge transformations:

\begin{align}
	B_2 \rightarrow B_2 + d\Lambda_1 && A_p \rightarrow A_p + d\Lambda_{p-1} - H_3 \wedge \Lambda_{p-3}
\end{align}

for any set of arbitrary k-forms $\Lambda_p$, leaving invariant the field strengths:

\begin{equation}
\begin{aligned}
	H_3 &:= dB_2 \\
	F_{p+1} &:= dA_p + H_3 \wedge A_{p-2} \label{fieldstrengths}
\end{aligned}
\end{equation}

Where $A_{p}$ with $p<0$ is set to $0$. Now, the RR potential $A_{p}$ obviously couples to $D(p-1)$-branes by an interaction term which is the integral of $A_{p}$ over the worldvolume; this is an electric coupling of the $D(p-1)$-brane to $F_{p+1}$. The coupling however could also be magnetic, electric-magnetic duality being implemented in general through Hodge duality. We define $F_p$ for additional values of $p$ through

\begin{equation}
	F_{9-p} = \widetilde\star F_{p+1}
\end{equation}

note that for the IIB $F_5$ this is actually a constraint. The new field strengths can then be locally trivialized as of \ref{fieldstrengths} and so we end up with a complete set of potentials $A_0, \ldots A_8$ for IIB and $A_1 \ldots A_9$ for IIA. The duality between potentials would be given by $A_p \leftrightarrow A_{8-p}$, and if $D(p-1)$-branes couple electrically to $A_p$, then $D(7-p)$-branes couple magnetically to it, that is to say electrically to $A_{8-p}$.\\

Therefore, the magnetic dual to a $Dp$-brane is a $D(6-p)$-brane.

\section{Action functional for IIB SUGRA}

There is a considerable obstacle to a covariant (i.e. explictly supersymmetric) formulation of type IIB supergravity in the self-duality constraint for the field strength 5-form $\tilde{F}_5$. We will take the common path of formulating the Lagrangian theory ignoring the constraint (and thus in excess of bosonic polarizations with respect to an explicity supersymmetric theory) and then imposing self-duality by hand after deriving the equations of motion. Therefore the action will not be supersymmetric itself, while the Euler-Lagrange equations augmented with the constraint will be.\\

Actually, for the purpose of building classical solutions, where spinor fields vanish anyway, the fermionic sector of the action will not be important. The bosonic sector is as such:

\[ S_B = S_{NS} + S_R + S_{CS} \]

where $S_{NS}$ is the action relevant to the fields originally from the superstring NS-NS sector:

\[ S_{NS} = \frac{1}{2\kappa^2} \int d^{10} x \sqrt{-g} \, e^{-2\phi} \left( R + 4 \partial_\mu \phi \partial^\mu \phi - \frac{1}{2} | H_3 |^2 \right) \]

Then $S_R$ is for R-R fields, essentially just kinetic terms for the $A$ forms:

\[ S_R = -\frac{1}{4\kappa^2} \int d^{10} x \sqrt{-g} 
\left(| F_1 |^2 + | \tilde{F}_3 |^2 + | \tilde{F}_5 |^2 \right)\]

And finally we supplement with a Chern-Simons type term:

\[ S_{CS} = -\frac{1}{4\kappa^2} \int A_4 \wedge H_3 \wedge F_3 \]

note the untilded $F_3$. This is evidently a purely topological term.

\section{D-brane action}

\cmmnt{azione BDI per la D-brana}

\chapter{D3-brane stacks on Calabi-Yau cones}

Perhaps the most essential ingredient for the conception of the idea of holography was the fact that coincident D3-branes (a "stack") naturally feature a 4D gauge theory on their world-volume, where the 4D fields emerge from the modes of open strings stretching between them. In the simplest and most famous example, a stack of $N$ D3-branes is placed in otherwise Minkowski $\mathbb{R}^{1,9}$; the corresponding field theory is the maximally supersymmetric Yang-Mills in four dimensions (SYM4).\\

Setting the stack on a different background geometry instead gives rise to a large family of different field theories; a particularly interesting subset is given by spacetimes of the form:

\begin{equation} M = \mathbb{R}^{1,3} \times X_6 \end{equation}

where the $\mathbb{R}^{1,3}$ is parallel to the branes (and must be identified with the field theory spacetime) and $X_6$ is a 6-dimensional Calabi-Yau cone over a compact 5-fold base $Y_5$. In this language, the SYM4 example above corresponds to $X_6 = \mathbb{R}^6 = \mathbb{C}^3$, which is (trivially) a cone over $\mathbb{S}^5$. This is the only case where $X_6$ turns out to be smooth; in general it will feature a conical singularity in the origin. Other choices for the base will typically yield theories with reduced (even minimal) supersymmetry, which are considerably more challenging to study.

\section{Brane stack in $\mathbb{C}^3$ and $\mathcal{N}=4$ super-Yang-Mills}

If $X_6 = \mathbb{C}^3$, the branes are invariant under half of the $16 \times 2 = 32$ IIB supercharges. The only possibility for a 4D theory to have $16$ supercharges is to be an $\mathcal{N}=4$, superconformal field theory\footnote{Indeed, the number of supercharges is $2$ for the components of a 4D Majorana spinor, times $\mathcal{N}$, times a factor of $2$ since the superconformal algebra has twice the supercharges of the usual SUSY algebra.}. Moreover, the theory features gluons as the massless spin-1 modes for string stretching between brane $i$ and brane $j$ ($i$ and $j$ being called Chan-Paton indices), so that the gauge group is $U(N)$, and the Chan-Paton indices lie in the fundamental. The information that the theory is a $U(N)$ gauge theory and is maximally supersymmetric is enough to uniquely fix it.\\

In $\mathcal{N}=1$ language (which we employ even though the model has $\mathcal{N}=4$) the theory describes the dynamics of $U(N)$ gauge vector supermultiplets $A_\mu$ and three complex chiral superfields $(X^a)_{i\dot j}$, $a=1,2,3$ in the adjoint of the gauge group (we will frequently omit gauge indices). These are nothing else than the parametrization of the D3-branes' position in $\mathbb{C}^3$ and therefore transform in the fundamental of $SU(3)$. The superpotential is the only one allowed by gauge and $SU(3)$ invariance:

\begin{equation} W(X) = \epsilon_{abc} \Tr ( X^a X^b X^c) \end{equation}

\section{Orbifolds}

\section{The conifold and the Klebanov-Witten model}

In \cite{KW_SCFT} the case of $X_6$ being the conifold was studied. The conifold is a specific Calabi-Yau 3-cone defined for example as the following variety in $\mathbb{C}^4$:

\begin{equation}
	z_1^2 + z_2^2 + z_3^2 + z_4^2 = 0
\end{equation}

This is notably not an orbifold of $\mathbb{C}^3$. The base can be found by quotienting by dilations $z_i \rightarrow \lambda z_i$ (with $\lambda \in \mathbb{R}_+$) and turns out to be the homogeneous space $SO(4)/U(1) = SU(2)\times SU(2) / U(1)$, where the $U(1)$ is a diagonal subgroup generated by, say, $T^3_L + T^3_R$. We will therefore have $SU(2)\times SU(2)$ as part of the isometry group of both $Y_5$ and $X_6$. An equivalent description of the topology of the conifold is as a $U(1)$ bundle over $\mathbb{C}P^1 \times \mathbb{C}P^1$; in these terms the metric on the base that makes the cone Calabi-Yau is

\begin{equation}
	ds^2_5 = \frac{1}{9} (d\psi + \cos\theta_1 d\phi_1 + \cos\theta_2 d\phi_2)^2 + \frac{1}{6} (d\Omega_1^2 + d\Omega_2^2)
\end{equation}

where $\Omega_i^2 = d\theta_i^2 + \sin\theta_i^2 d\phi_i^2$ is the metric on the $\mathbb{C}P^1_i$, and $\psi$ is the fibral coordinate with period $4\pi$.


\section{The $Y^{(2,0)}$ orbifold theory}


\chapter{Holography}

\section{Maldacena duality}

\cmmnt{Lo stack di D3-brane come soluzione di supergravità; lo stesso sistema in termini di stringhe aperte; disaccoppiamento della IIB su $\mathbb{M}^{10}$ e dualità di Maldacena fra la SYM N=4 e $AdS \times \mathbb{S}^5$}

\section{Features of AdS/CFT}

\cmmnt{relazione fra on-shell nel bulk e off-shell nel boundary; operator-state e le funzioni di partizione; massa e scaling dimension; rinormalizzazione e dimensione extra}

\section{Large $N$ limit}

We will now clarify what is meant by large $N$ limit for a Yang-Mills theory.\\

The Lagrangian is

\[\mathcal{L} = \Tr \left(F^2\right) + \ldots \]

with $F_{\mu\nu} = \partial_\mu A_\nu - \partial_\nu A_\mu + i g_{YM} [A_\mu,A_\nu]$ and $\ldots$ can include fields in the fundamental, adjoint, bifundamental, etc. These will all of course be representable as object with a certain number of colour indices (and symmetries between them).\\

We can modify the standard Feynman prescription for pictorially representing amplitudes to get a "double line" or "ribbon" representation in which each colour index is carried by a line. For example, the gluon self energy diagram becomes as such:\\

\cmmnt{INSERIRE DIAGRAMMA}\\

colour indices $i$, $\bar i$, $j$, $\bar j = 1 , \ldots , N$ are fixed, while $k$ must be summed over. Also, the amplitude has two three-gluon vertices, each carrying a factor of $g_{YM}^2$, for an overall factor of $g_{YM}^2 N$.\\

It's easy to convince oneself that as long as we restrict to planar diagrams, that is diagrams that can be drawn on the plane (or more precisely the sphere), adding one strip will always introduce exactly one additional loop and two additional vertices, again carrying a factor of $g_{YM}^2 N$. The combination $\lambda := g_{YM}^2 N$ is the 't Hooft coupling, and is better suited to represent the strength of the gauge interaction than $g_{YM}$ if we are to modify the number of colours.\\

So the 't Hooft large $N$ limit is defined as:

\begin{equation}
N \rightarrow \infty, \quad \mathrm{but \; keeping } \; \lambda \; \mathrm{fixed}
\end{equation}

A useful rescaling of the fields shifts all the $g_{YM}$ dependence of the Lagrangian to a factor in front:

\begin{equation} \label{rescaling} \mathcal{L} = \frac{1}{g_{YM}^2} \left( \Tr F^2 + \ldots \right) \end{equation}

so that now all types of vertices bring $g_{YM}^2 = \lambda/N$ and propagators bring $1/g_{YM}^2 = N/\lambda$.\\

We extend to nonplanar graphs by noting these can always be drawn on some Riemann surface of genus $g$, and, since they induce triangular tilings of said surface, the famous formula for the Euler characteristic holds:

\[ F - V + E = \chi = 2 - 2g \]

$F$, $V$, $E$ being the number of faces, vertices, edges respectively. Now each face (loop) carries a factor of $N$, each vertex a factor of $\lambda/N$, and each edge $N/\lambda$, so that the total contribution is

\[ \lambda^{E-V} N^{F-V+E} = \lambda^{E-V} N^{2-2g} \]

so that at fixed $\lambda$, an expansion in $N$ (or better $1/N$) is a genus expansion reminiscent of the loop expansion in perturbative string theory. This for example means that the free energy admits a power expansion in $1/N$:

\begin{equation}
F = \sum_{g=0}^\infty f_g(\lambda) N^{2-2g}
\end{equation}

One could be perplexed by the $N^2$ divergence of the genus zero contribution. This is not problematic however; it's an artifact of the rescaling \ref{rescaling} which makes the Lagrangian itself diverge as $g_{YM}^{-2} \Tr F^2 \sim N/\lambda \cdot N$, since the trace of a matrix in the adjoint scales as $N$.

\section{AdS/CFT over a cone}

As seen above, the original motivation for the AdS/CFT conjecture is the identification of a system of $N$ coincident D3-branes in a $\mathbb{M}^{10}$ Minkowski background and the corresponding 3-brane supergravity solution. In an appropriate low-energy limit a system of closed IIB strings on flat spacetime decouples in both pictures, suggesting it should be conjectured that the remaining parts are equivalent. These are respectively $\mathcal{N}=4$, $SU(N)$ SYM on $\mathbb{M}^4$ and IIB strings on $AdS_5 \times S_5$.\\

We repeat this reasoning, but in the more interesting case where the background for the D3-branes is generalized as $\mathbb{M}^4 \times X_6$, where $X_6$ is a cone over a base 5-manifold $Y_5$. We anticipate the bulk dual in this case is IIB strings over $AdS_5 \times Y_5$. By $X_6$ being a cone over $Y_5$ it is meant that the metric on it is

\begin{equation}
	ds^2_6 = dr^2 + r^2 ds_5^2 \label{conemetric}
\end{equation}

where of course $ds_5^2$ is the metric on $Y_5$. If $Y_5 = \mathbb{S}^5$ with the unit round metric then the cone is $X_6 = \mathbb{R}^6$ and one returns to the flat case.\\

For this to be a string background, $X_6$ should be Ricci-flat. This is equivalent to $Y_5$ being Einstein of positive curvature. $ds_6^2$ is conformally equivalent to the canonical metric on a cylinder over $Y_5$, as evidenced by the reparametrization $\phi = \ln r$:

\begin{equation}
	ds_6^2 = e^{2\phi} \left( d\phi^2 + ds_5^2 \right)
\end{equation}

Recalling the transformation law of the Ricci tensor in $n$ dimensions under conformal rescalings:

\begin{equation}
	R_{ij}' = R_{ij} - (n-2)\left( \nabla_i \partial_j \phi - \partial_i \phi \partial_j \phi \right) + \left( \nabla^2 \phi - (n-2) \nabla_k \phi \nabla^k \phi \right) g_{ij}
\end{equation}

And noting that for the cylinder the restriction of $R_{ij}$ to $Y_5$ indices gives $Y_5$'s own Ricci tensor $R_{ij}^{(5)}$, we obtain

\begin{equation}
	R^{(5)}_{ij} = 4 g_{ij}^{(5)}
\end{equation}

proving $Y_5$ is Einstein.\\

We are also interested in $X_6$ being Calabi-Yau, that is being K\"ahler with holonomy $\subset SU(3)$. We define $Y_5$ to be Sasaki-Einstein iff the corresponding cone is Calabi-Yau. The complex structure on the cone induces a vector field on the base, the Reeb vector:%nota: da scholarpedia

\begin{equation}
	\xi := J (r \partial_r)
\end{equation}

where $J$ is the complex structure on the cone and $\xi$ is to be thought of as restricted to, say, ${r=1} \cong Y_5$; this is a Killing vector on the base, inducing a 1-dimensional foliation. The dual form, $\theta = g_{ij} \xi^i dx^j$, is a contact form for the base, contact meaning the 2-form on the cone

\begin{equation}
	\omega = t^2 d\theta + t dt \wedge \theta
\end{equation}

is symplectic. This is of course the symplectic form associated to the hermitian structure.\\

After placing 3-branes in this $X_6 \times \mathbb{R}^4$ background, parallel to the Minkowski, the resulting geometry from their backreaction is:

\begin{equation}
ds^2 = H^{-1/2}(r,y) \, dx\cdot dx + H^{1/2}(r,y) \, ds_6^2
\end{equation}

Where $x^{0,\ldots,3}$ are coordinates parallel to the brane stack, $dx\cdot dx = -(dx^0)^2 + (dx^i)^2$, $r$ is the radial coordinate and the remaining $y^{1,\ldots,5}$ parametrize the cone's base $Y_5$. This is a simple generalization of the well-known black 3-brane solution by substitution of $\mathbb{S}^5$ with $Y_5$.\\

Ricci-flatness implies the function $H$ is harmonic: $\nabla H(r) = 0$. The linearity of this equation arises from the fact that D-branes are BPS states, corresponding in the gravitational picture to extremal p-branes; these notably do not interact mutually.\\

If the branes are coincident, the corresponding harmonic potential is

\begin{align}
 H(r) = 1 + \frac{R^4}{r^4} && R^4 = 4 \pi g_s N \alpha'^2 
\end{align}

The near-horizon limit ($r\rightarrow 0$) in that case can be read immediately:

\begin{equation}
ds^2 = \frac{ dx \cdot dx + dz^2}{z^2} + ds_5^2
\end{equation}

where $z := 1/r$; this is evidently the product metric on $AdS_5 \times Y_5$, where of $AdS_5$ we're only considering the Poincaré patch. \\

We note that the introduction of a conical singularity results in reduced supersymmetry. Unbroken SUSY generators are identified from the Killing spinor equation:

\begin{equation}
	\left(\partial_\mu + \frac{1}{4} \omega_{\mu\alpha\beta} \Gamma^{\alpha\beta} \right) \eta = 0
\end{equation}

Explicitly for the cone metric \ref{conemetric}:

\begin{equation}
	\left(\partial_i + \frac{1}{4} \omega_{ijk} \Gamma^{jk} + \frac{1}{2} \Gamma^r_i \right) \eta = 0
\end{equation}

this is, as expected, coincident with the ($Y_5$ sector of) Killing spinor equation for the backreacted $AdS_5 \times Y_5$ geometry, including also the effect of $F_5$. This is to show there is a match between the unbroken SUSYs in the bulk theory and in the boundary.\\

If the cone is of holonomy $SU(n)$, this will result in a reduction of supersymmetries by a factor of $2^{1-n}$ with respect to $\mathbb{M}^{10}$ Minkowski. In particular, if $X_6$ is Calabi-Yau, then the $32 = 16 \times 2$ fermionic generators of IIB SUGRA are reduced to $32 \times 2^{-2} = 8$, which means the SCFT in $4D$ has $\mathcal{N}=1$ (in contrast to the usual SUSY algebra, the $\mathcal{N}=1$ $4D$ superconformal group has both $4$ supertranslations and $4$ additional fermionic superconformal generators). If, instead, we were to consider the more restrictive case of manifolds of $SU(2)$ holonomy, there would be $16$ unbroken supersymmetries signaling an $\mathcal{N}=2$ dual SCFT.

\section{The Klebanov-Witten model}

A well-known specific example of SCFT holographically dual to D-branes in a Calabi-Yau cone has been introduced in \cite{KW_SCFT}. In this case the base of the cone is the manifold $T^{1,1} = (SU(2)\times SU(2))/U(1)$, where $U(1) \subset SU(2)_L\times SU(2)_R$ is generated by $\sigma^3_L + \sigma^3_R$.\\

We give a characterization of the cone $X_6$ over $T^{1,1}$ as a submanifold of $\mathbb{C}^4 \ni (A^1,A^2,B^1,B^2)$ given by

\begin{equation}
	|A^1|^2 + |A^2|^2 - |B^3|^2 - |B^4|^2 = 0
\end{equation}

quotiented by $U(1)$ acting on $A^i$ with charge $1$ and $B^i$ with charge $-1$. This makes the $SU(2)\times SU(2) \approx SO(4)$ symmetry manifest with the two copies of $SU(2)$ acting respectively only on $A^i$ and $B^i$.\\


The holographic dual theory to a stack of $N$ D3-brane moving in the background given by $\mathbb{R}^4$ times the cone over $T^{1,1}$ is found to be an $\mathcal{N}=1$ superconformal quiver gauge theory with gauge group $SU(N)\times SU(N)$, with chiral superfields $A^i$, $B^i$ ($i=1,2$) transforming respectively in the bifundamentals $(\mathbf{N},\mathbf{\bar N})$, $(\mathbf{\bar N},\mathbf{N})$. The $SU(2)\times SU(2)$ isometry in the bulk is here implemented as a "flavour" global symmetry acting separately on $A^i$ and $B^i$. Moreover, the diagonal $U(1)$ reappears as a baryonic symmetry where again $A^i$ has charge $1$ and $B^i$ charge $-1$.\\


\cmmnt{relazione fra i campi della CFT e la posizione delle D-brane, moduli space mesonico e totale, un po' di più sul KW}

\chapter{Holographic effective field theories}

\cmmnt{definizione di HEFT come in \cite{MZ}}

\section{Bulk moduli}

\cmmnt{identificazione dei moduli del bulk: numeri di Betti e condizioni globali sui potenziali, moduli della struttura K\"ahler}

We now identify the moduli of the bulk string theory. This will include deformations of the background metric and of the RR potentials, parametrized by particular types of $k$-forms. Therefore, a great deal of information about them can be obtained just from examining the topology of the cone $X_6$.\\

First of all, we take as an assumption that the third Betti number of the cone vanishes:

\begin{equation}
	b_3(X) = 0 \label{bettiX3}
\end{equation}

It can be proven from Myers' theorem that $Y_5$ being Sasaki-Einstein means the following Betti numbers vanish:

\begin{equation}
	b_1(Y) = b_4(Y) = 0 \label{bettiY14}
\end{equation}

It's also possible to prove that

\begin{equation}
	b_1(X) = b_5(X) = b_6(X) = 0
\end{equation}

The vanishing of the odd Betti numbers for $X_6$ means $Dp$-branes with $p = 1,3,5$ cannot be wrapped around nontrivial $p$-cycles.\\

We recall the long sequence involving relative homology groups:

\begin{equation}
	\ldots \rightarrow H^{i-1}(Y) \rightarrow H^{i}(X,Y) \rightarrow H^i(X) \rightarrow H^i(Y) \rightarrow H^{i+1}(X,Y) \rightarrow \ldots
\end{equation}

where $H^i(X,Y;\mathbb{R})$ is the relative homology group - closed $k$-forms on $X$ vanishing on $Y$ modulo exact forms with the same property - and when the $;\mathbb{R}$ is omitted we implicitly mean the base field is $\mathbb{R}$. We cut the sequence short by setting $i=2$ and noting $H^1(Y) = 0$ as of \ref{bettiY14} and $H^3(X,Y) \subset H^3(X) = 0$ as of \ref{bettiX3}; the short exact sequence is

\begin{equation}
	0 \rightarrow H^2(X,Y) \rightarrow H^2(X) \rightarrow H^2(Y) \rightarrow 0
\end{equation}

Implying $H^2(X) = H^2(Y) \oplus H^2(X,Y)$. Applying Poincaré duality on the two components and counting dimensions gives

\begin{equation}
	b_2(X) = b_3(Y) + b_4(X)
\end{equation}

This result will be useful in parametrizing deformations of the axio-dilaton $\tau$ and the 2-forms $C_2$ and $B_2$. In particular, combining these fields into a single complex $2$-form $C_2 - \tau B_2$, for any given value of $\tau$ the deformations of this form are decomposable in $b_2(X)$ complex parameters, to which we then append one additional parameter for deformations of $\tau$. Making use of the above splitting, these are divided in $b_3(Y) + 1$ 'boundary' complex parameters counting non-dynamical deformations, and $b_4(X)$ 'bulk' dynamical complex moduli.\\

\cmmnt{$C_4$ moduli}

Having dealt with RR moduli, we now consider the moduli of the K\"ahler structure of the background. Since $b_3(X) = 0$ by hypothesis, the complex structure is rigid. There are instead moduli for the K\"ahler form $J$; in particular we know from \cmmnt{citazione teoremi esistenza} that every cohomology class $[J]$ of $H^2(X)$ contains a single representative Ricci-flat K\"ahler form $J$, so that $H^2(X)$ is the moduli space for the K\"ahler structure. We can expand the cohomology class as

\begin{equation}
	[J] = v^a [\omega_a] \label{integraldecomposition}
\end{equation}

with $[\omega_a]$ being a basis for the integral cohomology $H^2(X;\mathbb Z)$, as the latter modulo torsion is a lattice sitting in $H^2(X;\mathbb R)$. This means

\begin{equation}
	\delta [J] = \delta v^a [\omega_a]
\end{equation}

meaning there exist representatives in the classes such that the equation without square brackets holds. Since small variations of the K\"ahler form must be $(1,1)$ harmonic forms \cmmnt{reference}, we then know there exist $(1,1)$ harmonic representatives $\omega_a$ for the aforementioned basis of classes. Returning to \ref{integraldecomposition} we can rewrite it as

\begin{equation}
	J - v^a \omega_a \in [0]
\end{equation}

But for the LHS to belong to the zero class just means to be exact. Therefore

\begin{equation}
	J = J_0 + v^a \omega_a \label{JandJ0}
\end{equation}

with $J_0$ being exact and $(1,1)$. Note the linearity of this parametrization is an illusion of notation: the condition $\omega_a = 0$ depends on the metric and so on both $J_0$ and $v^a$.


\section{Boundary moduli}

\cmmnt{descrizione della CFT holografica tipica (quiver), di nuovo classificazione dei moduli}

\section{Effective action}

\cmmnt{azione efficace calcolata in \cite{MZ}; forse anche la derivazione?}

\section{The Klebanov Witten HEFT}

\cmmnt{Calcolo esplicito della HEFT per il KW}

\chapter{The $Y^{(2,0)}$ HEFT}

\section{K\"ahler moduli}

The general Calabi-Yau deformation of the $Y^{2,0}$ cone is already well-known (\cmmnt{refs}) in real coordinates as:

\begin{equation}
ds^2 = \kappa^{-1}(r)dr^2 + \frac{1}{9} \kappa(r) r^2 (d\psi + \cos\theta_L d\phi_L + \cos\theta_R d\phi_R)^2 + \frac{1}{6} r^2 d\Omega_L^2 + \frac{1}{6}(r^2+a^2) d\Omega_R^2 \label{y20metric} \end{equation}
\begin{equation}
	\kappa(r) = \frac{1 + \frac{9a^2}{r^2} - \frac{b^6}{r^6}}{1+ \frac{6a^2}{r^2}}
\end{equation}

with $a,b$ the two unique real moduli. The topology is that of an $\mathbb{R}^2$ bundle over $\mathbb{S}^2 \times \mathbb{S}^2$.\\

For the purpose of building the effective theory, however, this metric must be rewritten in a complex chart. To that end, we try to find the general CY metric on a $\mathbb{C} \rightarrow \mathbb{CP}^1 \times \mathbb{CP}^1$ bundle; on the spheres of the base we take the round metric, given by the K\"ahler forms $j^L$ and $j^R$. It's easy to verify explicitly that, given any set of complex coordinates on the base $(y_L,y_R)$,

\begin{equation}
	j^L \wedge j^R = e^{-\Lambda k} dy^L \wedge dy^R \wedge d\bar y^L \wedge d\bar y^R
\end{equation}

with $k = k^L + k^R$ the total base potential, and for some $\Lambda$ depending on the overall size of the spheres (for unit radius, $\Lambda = 1$).\\

We also introduce the function $t$ of the fibral coordinate $\zeta$ as 

\begin{equation}
	t = |\zeta|^2 e^{\Lambda k}
\end{equation}

We then start from the following ansatz for the K\"ahler potential:

\begin{equation}
	k_X = f(t) + \alpha^L k^L + \tilde\alpha^R k^R
\end{equation}

where $\alpha,\tilde\alpha$, controlling the volume at $t=0$ of the base 2-spheres, parametrize the Ricci-flat K\"ahler resolutions of the cone.\\

The corresponding K\"ahler form is straightforward:

\begin{equation}
	J = A^L j^L + A^R j^R + i e^{\Lambda k} (f' + t f'') (d\zeta + \Lambda \zeta \partial k) \wedge (\mathrm{c.c.})
\end{equation}

with $A^i = \alpha^i + \Lambda t f'(t)$. This is more simply $J = J_M + M \omega \wedge \bar \omega$, where $J_M$ is the purely basal part, $\omega = d\zeta + \Lambda \zeta \partial k$ and $M$ is a scalar factor. The volume form is then easy

\begin{equation}
	J \wedge J \wedge J = 3 A^L A^R M \, j^1 \wedge j^2 \wedge \omega \wedge \bar \omega
\end{equation}

As all other terms in the cube vanish. Since the volume form is $\sqrt{\det g} \, d\Omega \wedge \bar \Omega$, with $\Omega = d\zeta \wedge dy^L \wedge dy^R$, and the Ricci tensor for a K\"ahler space is proportional to $\partial \bar \partial \ln \det g$, then the condition for Ricci-flatness is equivalent to the prefactor of $\Omega \wedge \bar \Omega$ in $J\wedge J \wedge J$ being constant, that is to say

\begin{equation}
	(\alpha^L + \Lambda t f')(\alpha^R + \Lambda t f') \frac{d}{dt} (\Lambda t f') = c \label{rflatcondition}
\end{equation}

or, having defined $y = \Lambda t f'$,

\begin{equation}
	(\alpha^L + y)(\alpha^R + y) y' = c
\end{equation}

Since $f(t)$ must be regular as $t=0$, and $f' = \frac{y}{\Lambda t}$, it must be that $y$ goes to zero at least as fast as $t$ as $t$; this condition eliminates the freedom from the constant of integration for equation \ref{rflatcondition}. The constant $c$ on the other hand can be readily reabsorbed into a $t$ rescaling. Therefore there should be a unique $y$ (and so a unique $f$ up to constant shifts) that gives a Ricci-flat metric. Let us see this explicitly: we integrate \ref{rflatcondition} to obtain

\begin{equation}
	\frac{y^3}{3} + \frac{\alpha^L + \alpha^R}{2} y^2 + \alpha^L \alpha^R y = ct + d \label{rflatintegrated}
\end{equation}

And then the regularity condition $y(0)=0$ is satisfied with $d=0$, and this cubic equation for $y$ is immediately seen to have one single real solution for any positive values of $\alpha^i$, $c$.\\

Before exhibiting the explicit form of $y(t;\alpha^i)$, let us express the K\"ahler form in terms of $y$ and show it's actually equal to the real-coordinate metric \ref{y20metric}. We have

\begin{align}
	J & =  (\alpha^L + y) j^1 + (\alpha^R + y) j^2 + \frac{ie^{\Lambda k}}\Lambda y' \omega \wedge \bar \omega\\
	  & =  (\alpha^L + y) j^1 + (\alpha^R + y) j^2 + \frac{ie^{\Lambda k} \,c}{\Lambda (\alpha^L + y)(\alpha^R + y)} \omega \wedge \bar \omega
\end{align}

Now, we parametrize the fiber as $\zeta = e^{-\Lambda k/2} t^{1/2} e^{i\psi}$, and the 2-spheres with $\theta_i$, $\phi_i$ which fixed $\Lambda = 1$. Then the metric corresponding to $J$ is

\begin{equation}
	ds^2 = A^L d\Omega^2_L + A^R d\Omega^2_R + \frac{1}{c\Lambda t A^1 A^2} \left( \frac{dt^2}{4} + t^2 (d\psi + \sigma)^2 \right)
\end{equation}

Where $\sigma = -i\frac{\Lambda}{2}(\partial k - \bar \partial k)$. But this is simply

\begin{equation}
	ds^2 = \frac{1}{4y't} dy^2 + (y' t) (d\psi + \sigma)^2
\end{equation}

Using both \ref{rflatcondition} and its integrated form \ref{rflatintegrated} we rewrite

\begin{align}
	y't & = \frac{1}{A^L A^R} \left( \frac{y^3}{3} + \frac{\alpha^L + \alpha^R}{2} y^2 + \alpha^L \alpha^R y \right)\\
	& = 3cr^2 \frac{1+ \frac{3}{2} \frac{\alpha^R - \alpha^L}{r^2} + \frac{\alpha^{L2}(\alpha^L - 3 \alpha^R)}{2r^6} }{1+ \frac{\alpha^R -\alpha^L}{r^2} }\\
	& = 3cr^2 \kappa(r)
\end{align}



The coordinate change to the $r$ coordinate is then given by $r^2 = A^L = y + \alpha^L$ - note this renders the inherent symmetry between the left and right 2-cycles non-manifest.\\




\appendix

\chapter{Appendix}

\section{AdS space}

Anti-de Sitter $n$-space is best understood as the Lorentzian analogue of hyperbolic $n$-space. It can be built by considering the following locus in the mixed-signature space $\mathbb{R}^{2,n-1}$:

\begin{equation} \label{ads locus}
x^\mu x_\mu = -(t^1)^2 - (t^2)^2 + \sum_{i=1}^{n-1} (x^i)^2  =  - R^2
\end{equation}

which is reminiscent of the embedding of hyperbolic $n$-space in $\mathbb{R}^{1,n}$:

\begin{equation}
x^\mu x_\mu = -t^2 + \sum_{i=1}^{n} (x^i)^2 = - R^2
\end{equation}

Equation \ref{ads locus} is explicitly preserved by $SO(2,n-1)$, and this group acts transitively on it, so that the locus inherits a Lorentzian metric from the ambient Minkowski space with that same symmetry group. This means the locus is a maximally symmetric space, having the same number of symmetries as $\mathbb{R}^{1,n-1}$ since $\dim SO(2,n-1) = \dim \left( \mathbb{R}^n \rtimes SO(1,n) \right)$. (To press on with the analogy, in the Riemannian case $\mathbb{H}^n$ has the same number of Killing vectors as $\mathbb{R}^n$ since $\dim SO(1,n) = \dim \left(\mathbb{R}^n \rtimes SO(n) \right)$).\\

The locus has constant negative scalar curvature (using $S$ for the Ricci scalar to avoid confusion with the $R$ radius introduced above):

\begin{equation}
S = - \frac{n(n-1)}{R^2} 
\end{equation}

However, the locus built above is not suitable to be used as a spacetime for a reasonable physical theory, as it contains closed timelike curves (CTCs), signaling a pathological causal structure. An example of CTC is the unit circle in the $t^1 t^2$ plane. It's possible however to consider the covering space of the locus, which will be what we will refer to as anti-de Sitter $n$-space, AdS$_n$. The covering space is again a maximally symmetric space, but it's now simply-connected and CTC-free.\\

AdS, similarly to dS, admits multiple useful coordinate charts. The Poincaré chart is the analogue of the Poincaré half plane model, and the metric is:

\begin{equation} \label{poincarechart}
ds^2 = \frac{R^2}{z^2} \left(dz^2 + dx^\mu dx_\mu \right)
\end{equation}

where $z>0$, $x^\mu \in \mathbb{R}^{1,n-2}$, and $dx^\mu dx_\mu$ is the standard metric on $\mathbb{R}^{1,n-2}$. The Poincaré chart, unlike the Riemannian case, is not global and only maps a particular wedge of the full AdS. A global chart would be given by the following coordinates, accordingly called global coordinates or cylindrical coordinates:

\begin{equation}
ds^2 = R^2 \left( -\cosh^2 \chi \, d\tau^2 + d\chi^2 + \sinh^2 \chi \, d\Omega^2 \right)
\end{equation}

With $d\Omega^2$ the line element on $\mathbb{S}^{n-2}$. Note that constant $\tau$ slices are copies of $\mathbb{H}^{n-1}$. Remapping the radial coordinate as $d\chi = d\rho/\cos\rho$ to a finite range ($0\le \rho \le \pi/2$) this can also be rewritten as

\begin{equation} \label{polarrho}
	ds^2 = R^2 \frac{1}{\cos^{2} \rho} \left( - dt^2 + d\rho^2 + \sin^2 \rho d\Omega^2  \right)
\end{equation}

\section{Conformal boundary and symmetries}

The last set of coordinates \ref{polarrho} are a starting point for building the Penrose diagram of AdS. For fixed $\Omega_i$ the $t$,$\rho$ part of the metric is sent to the flat metric by multiplication with the conformal factor $\cos^2 \rho$. AdS is thus represented as an infinite solid cylinder.\\

We can read the induced topology and metric on the boundary, with the caveat that the conformal factor was arbitrary (provided it was such the metric did not diverge), and thus the boundary's metric will be defined up to a conformal rescaling - we can only identify a natural conformal class for the boundary. This will prove to have physical relevance as possible holographic duals will be conformal.\\

The topology of the boundary is therefore $\mathbb{S}^{n-2} \times \mathbb{R}$ and a representative of the conformal class is given by setting $\rho = \pi/2$:

\begin{equation}
	ds^2 = dt^2 - d\Omega^2 
\end{equation}

which is a Lorentzian metric. The conformal boundary of AdS is itself a spacetime; this is a nontrivial fact which has to be compared with the other constant-curvature manifolds of the same signature: the boundary of Minkowski space $\mathbb{R}^{1,n-1}$ has a vanishing (null) metric, being composed of null past and future, while the positive curvature case, de Sitter, has two spacelike boundaries in the infinite past and future. The relevance of this for the realization of holography should be evident. Only the negative curvature case seems to be able to naturally incorporate a Lorentzian structure on the boundary.\\

It will be much more useful for the application to holography to consider the boundary in the form it comes out from the Poincaré patch. This is located at $z=0$ and is only a part of the full boundary. Taking the metric \ref{poincarechart} and factor a conformal $z^2$ we just obtain

\begin{equation}
	ds^2 = x^\mu x_\mu
\end{equation}

that is, the boundary is (locally) Minkowski $(n-2)$-space. This will be our preferential choice of representative metric.\\

We now turn to the description of the interplay between the bulk's and the boundary's symmetries. Essentially, isometries of AdS will induce conformal transformations on its boundary. As we've seen through its construction, the isometry group of AdS is $SO(2,n-1)$, this also coincides with the conformal group on $\mathbb{R}^{1,n-2}$.

\cmmnt{+altre banalità di geometria}

\backmatter

\bibliographystyle{plain}
\bibliography{bibliography}

\end{document}
