
In the previous chapter we explained how the dynamics of brane stacks, in particular D3-branes in type IIB, are described by gauge field theory on their worldvolumes. It's however important to note that parallel to this ``open string'' picture of the brane stack system there is also a dual description in terms of the curved spacetimes generated by their mass. Insisting these two viewpoints are equivalent, one is able to deduce an exact correspondence between the gauge theory and string theory on the near-horizon geometry.\\

This kind of duality is exotic as it connects a local field theory in four dimensions with an essentially five-dimensional string (and so, inherently gravitational) theory through a perfect mapping. It is reasonable in fact to identify the spacetime of the field theory with the conformal boundary of the higher-dimensional gravitational background it's dual to (the bulk), for reasons we will clarify - so that in more colloquial language the dynamics in the bulk are ``encoded'' in the screen at infinity, hence the adjective ``holographic'' for this sort of correspondences.\\

Explicit holographic correspondences are not only interesting by themselves as ; they're also extremely practical tools for studying the theories involved on both sides of them. It's certainly very attractive for the purpose of quantum gravity or the definition of string theory - non-local theories without action functionals - if these situations happen to be equivalent to a local quantum field theory.\\

%For example: if a gravitational, and therefore stringy, theory on a certain background is dual to a 4D field theory, entropy will scale as the boundary (hyper-)area, and not like the bulk volume, as we'd expect if the gravitational side were a local field theory. The nonlocality

However in this work our interest will be focused on the opposite direction, investigating the dynamics of the field theory by exploiting the dual gravitational system. The power of holographic dualities lies in the fact that they map the strongly-coupled regime for the field theory to the regime where the bulk dynamics can be approximated by supergravity. The traditionally untreatable strong coupling region for some gauge QFTs in four dimensions can then be probed by studying the relatively tamer dynamics of a smooth dual spacetime.

\section{Maldacena duality}

We now consider the $IIB$ supergravity solution modeling the spacetime created by a system of D3-branes in a background $\mathbb{R}^{1,9}$. This is given by

\begin{align}
	ds^2 = H^{-1/2} dx_\mu dx^\mu + H^{1/2} (dr^2 + r^2 d\Omega^2_5)\label{black3metric}\\	
	e^\Phi = \mathrm{const} =: g_s \\
	F_5 = dH^{-1} \wedge dx^0 \wedge dx^1 \wedge dx^2 \wedge dx^3
\end{align}

\begin{equation}
	H(r) = 1 + \left( \frac{R}{r} \right)^4
	\label{}
\end{equation}

where $x^\mu$, $\mu = 0,\ldots,3$ are coordinates parallel to the brane stack and $d\Omega_5$ is the standard metric on $\mathbb{S}^5$.\\

The curvature radius $R$ is given by

\begin{equation}
	R^4 = 4\pi g_s N \alpha'^2
	\label{}
\end{equation}

where $N$ is the number of D3-branes in the stack.\\

\cmmnt{note on the throat.}\\

This system (IIB string theory on the metric \ref{black3metric}) must then be equivalent to the stack of D3-branes in the background Minkowski, taking into account both open and closed string interactions. The action is schematically:

\begin{equation}
	S = \frac{1}{g_s} \int d^4 x F^2 + \frac{1}{\alpha'^4} \int d^{10}x \sqrt{g} R e^{-2\phi} + O(\alpha') + \dots
\end{equation}

(where, we recall, $g_{YM}^2 \sim g_s$). The first two terms are respectively the actions for SYM and free IIB SUGRA in the Minkowski background; the following terms, with higher powers of $\alpha'$, establish the coupling between these two systems. It's clear that in the limit $\alpha' \rightarrow 0$ the free SUGRA part decouples from the SYM.\\

We repeat this decoupling limit for the black 3-brane metric. If $\alpha' \rightarrow 0$, so does $R$, and effectively the metric seems to converge to flat spacetime. We have however to take into consideration the throat described before. \cmmnt{decoupling limit}

\section{Features of AdS/CFT}

\cmmnt{relazione fra on-shell nel bulk e off-shell nel boundary; operator-state e le funzioni di partizione; massa e scaling dimension; rinormalizzazione e dimensione extra}

\section{Large $N$ limit}

We will now clarify what is meant by large $N$ limit for a Yang-Mills theory.\\

The Lagrangian is

\[\mathcal{L} = \Tr \left(F^2\right) + \ldots \]

with $F_{\mu\nu} = \partial_\mu A_\nu - \partial_\nu A_\mu + i g_{YM} [A_\mu,A_\nu]$ and $\ldots$ can include fields in the fundamental, adjoint, bifundamental, etc. These will all of course be representable as object with a certain number of colour indices (and symmetries between them).\\

We can modify the standard Feynman prescription for pictorially representing amplitudes to get a "double line" or "ribbon" representation in which each colour index is carried by a line. For example, the gluon self energy diagram becomes as such:\\

\cmmnt{INSERIRE DIAGRAMMA}\\

colour indices $i$, $\bar i$, $j$, $\bar j = 1 , \ldots , N$ are fixed, while $k$ must be summed over. Also, the amplitude has two three-gluon vertices, each carrying a factor of $g_{YM}^2$, for an overall factor of $g_{YM}^2 N$.\\

It's easy to convince oneself that as long as we restrict to planar diagrams, that is diagrams that can be drawn on the plane (or more precisely the sphere), adding one strip will always introduce exactly one additional loop and two additional vertices, again carrying a factor of $g_{YM}^2 N$. The combination $\lambda := g_{YM}^2 N$ is the 't Hooft coupling, and is better suited to represent the strength of the gauge interaction than $g_{YM}$ if we are to modify the number of colours.\\

So the 't Hooft large $N$ limit is defined as:

\begin{equation}
N \rightarrow \infty, \quad \mathrm{but \; keeping } \; \lambda \; \mathrm{fixed}
\end{equation}

A useful rescaling of the fields shifts all the $g_{YM}$ dependence of the Lagrangian to a factor in front:

\begin{equation} \label{rescaling} \mathcal{L} = \frac{1}{g_{YM}^2} \left( \Tr F^2 + \ldots \right) \end{equation}

so that now all types of vertices bring $g_{YM}^2 = \lambda/N$ and propagators bring $1/g_{YM}^2 = N/\lambda$.\\

We extend to nonplanar graphs by noting these can always be drawn on some Riemann surface of genus $g$, and, since they induce triangular tilings of said surface, the famous formula for the Euler characteristic holds:

\[ F - V + E = \chi = 2 - 2g \]

$F$, $V$, $E$ being the number of faces, vertices, edges respectively. Now each face (loop) carries a factor of $N$, each vertex a factor of $\lambda/N$, and each edge $N/\lambda$, so that the total contribution is

\[ \lambda^{E-V} N^{F-V+E} = \lambda^{E-V} N^{2-2g} \]

so that at fixed $\lambda$, an expansion in $N$ (or better $1/N$) is a genus expansion reminiscent of the loop expansion in perturbative string theory. This for example means that the free energy admits a power expansion in $1/N$:

\begin{equation}
F = \sum_{g=0}^\infty f_g(\lambda) N^{2-2g}
\end{equation}

One could be perplexed by the $N^2$ divergence of the genus zero contribution. This is not problematic however; it's an artifact of the rescaling \ref{rescaling} which makes the Lagrangian itself diverge as $g_{YM}^{-2} \Tr F^2 \sim N/\lambda \cdot N$, since the trace of a matrix in the adjoint scales as $N$.

\section{AdS/CFT over a cone}

As seen above, the original motivation for the AdS/CFT conjecture is the identification of a system of $N$ coincident D3-branes in a $\mathbb{M}^{10}$ Minkowski background and the corresponding 3-brane supergravity solution. In an appropriate low-energy limit a system of closed IIB strings on flat spacetime decouples in both pictures, suggesting it should be conjectured that the remaining parts are equivalent. These are respectively $\mathcal{N}=4$, $SU(N)$ SYM on $\mathbb{M}^4$ and IIB strings on $AdS_5 \times S_5$.\\

We repeat this reasoning, but in the more interesting case where the background for the D3-branes is generalized as $\mathbb{M}^4 \times X_6$, where $X_6$ is a cone over a base 5-manifold $Y_5$. We anticipate the bulk dual in this case is IIB strings over $AdS_5 \times Y_5$. By $X_6$ being a cone over $Y_5$ it is meant that the metric on it is

\begin{equation}
	ds^2_6 = dr^2 + r^2 ds_5^2 \label{conemetric}
\end{equation}

where of course $ds_5^2$ is the metric on $Y_5$. If $Y_5 = \mathbb{S}^5$ with the unit round metric then the cone is $X_6 = \mathbb{R}^6$ and one returns to the flat case.\\

For this to be a string background, $X_6$ should be Ricci-flat. This is equivalent to $Y_5$ being Einstein of positive curvature. $ds_6^2$ is conformally equivalent to the canonical metric on a cylinder over $Y_5$, as evidenced by the reparametrization $\phi = \ln r$:

\begin{equation}
	ds_6^2 = e^{2\phi} \left( d\phi^2 + ds_5^2 \right)
\end{equation}

Recalling the transformation law of the Ricci tensor in $n$ dimensions under conformal rescalings:

\begin{equation}
	R_{ij}' = R_{ij} - (n-2)\left( \nabla_i \partial_j \phi - \partial_i \phi \partial_j \phi \right) + \left( \nabla^2 \phi - (n-2) \nabla_k \phi \nabla^k \phi \right) g_{ij}
\end{equation}

And noting that for the cylinder the restriction of $R_{ij}$ to $Y_5$ indices gives $Y_5$'s own Ricci tensor $R_{ij}^{(5)}$, we obtain

\begin{equation}
	R^{(5)}_{ij} = 4 g_{ij}^{(5)}
\end{equation}

proving $Y_5$ is Einstein.\\

We are also interested in $X_6$ being Calabi-Yau, that is being K\"ahler with holonomy $\subset SU(3)$. We define $Y_5$ to be Sasaki-Einstein iff the corresponding cone is Calabi-Yau. The complex structure on the cone induces a vector field on the base, the Reeb vector:%nota: da scholarpedia

\begin{equation}
	\xi := J (r \partial_r)
\end{equation}

where $J$ is the complex structure on the cone and $\xi$ is to be thought of as restricted to, say, ${r=1} \cong Y_5$; this is a Killing vector on the base, inducing a 1-dimensional foliation. The dual form, $\theta = g_{ij} \xi^i dx^j$, is a contact form for the base, contact meaning the 2-form on the cone

\begin{equation}
	\omega = t^2 d\theta + t dt \wedge \theta
\end{equation}

is symplectic. This is of course the symplectic form associated to the hermitian structure.\\

After placing 3-branes in this $X_6 \times \mathbb{R}^4$ background, parallel to the Minkowski, the resulting geometry from their backreaction is:

\begin{equation}
ds^2 = H^{-1/2}(r,y) \, dx\cdot dx + H^{1/2}(r,y) \, ds_6^2
\end{equation}
\cmmnt{già fatto! ref ref ref}
Where $x^{0,\ldots,3}$ are coordinates parallel to the brane stack, $dx\cdot dx = -(dx^0)^2 + (dx^i)^2$, $r$ is the radial coordinate and the remaining $y^{1,\ldots,5}$ parametrize the cone's base $Y_5$. This is a simple generalization of the well-known black 3-brane solution by substitution of $\mathbb{S}^5$ with $Y_5$.\\

Ricci-flatness implies the function $H$ is harmonic: $\nabla H(r) = 0$. The linearity of this equation arises from the fact that D-branes are BPS states, corresponding in the gravitational picture to extremal p-branes; these notably do not interact mutually.\\

If the branes are coincident, the corresponding harmonic potential is

\begin{align}
 H(r) = 1 + \frac{R^4}{r^4} && R^4 = 4 \pi g_s N \alpha'^2 
\end{align}

The near-horizon limit ($r\rightarrow 0$) in that case can be read immediately:

\begin{equation}
ds^2 = \frac{ dx \cdot dx + dz^2}{z^2} + ds_5^2
\end{equation}

where $z := 1/r$; this is evidently the product metric on $AdS_5 \times Y_5$, where of $AdS_5$ we're only considering the Poincaré patch. \\

We note that the introduction of a conical singularity results in reduced supersymmetry. Unbroken SUSY generators are identified from the Killing spinor equation:

\begin{equation}
	\left(\partial_\mu + \frac{1}{4} \omega_{\mu\alpha\beta} \Gamma^{\alpha\beta} \right) \eta = 0
\end{equation}

Explicitly for the cone metric \ref{conemetric}:

\begin{equation}
	\left(\partial_i + \frac{1}{4} \omega_{ijk} \Gamma^{jk} + \frac{1}{2} \Gamma^r_i \right) \eta = 0
\end{equation}

this is, as expected, coincident with the ($Y_5$ sector of) Killing spinor equation for the backreacted $AdS_5 \times Y_5$ geometry, including also the effect of $F_5$. This is to show there is a match between the unbroken SUSYs in the bulk theory and in the boundary.\\

If the cone is of holonomy $SU(n)$, this will result in a reduction of supersymmetries by a factor of $2^{1-n}$ with respect to $\mathbb{M}^{10}$ Minkowski. In particular, if $X_6$ is Calabi-Yau, then the $32 = 16 \times 2$ fermionic generators of IIB SUGRA are reduced to $32 \times 2^{-2} = 8$, which means the SCFT in $4D$ has $\mathcal{N}=1$ (in contrast to the usual SUSY algebra, the $\mathcal{N}=1$ $4D$ superconformal group has both $4$ supertranslations and $4$ additional fermionic superconformal generators). If, instead, we were to consider the more restrictive case of manifolds of $SU(2)$ holonomy, there would be $16$ unbroken supersymmetries signaling an $\mathcal{N}=2$ dual SCFT.

\section{The Klebanov-Witten model}

A well-known specific example of SCFT holographically dual to D-branes in a Calabi-Yau cone has been introduced in \cite{KW_SCFT}. In this case the base of the cone is the manifold $T^{1,1} = (SU(2)\times SU(2))/U(1)$, where $U(1) \subset SU(2)_L\times SU(2)_R$ is generated by $\sigma^3_L + \sigma^3_R$.\\

We give a characterization of the cone $X_6$ over $T^{1,1}$ as a submanifold of $\mathbb{C}^4 \ni (A^1,A^2,B^1,B^2)$ given by

\begin{equation}
	|A^1|^2 + |A^2|^2 - |B^3|^2 - |B^4|^2 = 0
\end{equation}

quotiented by $U(1)$ acting on $A^i$ with charge $1$ and $B^i$ with charge $-1$. This makes the $SU(2)\times SU(2) \approx SO(4)$ symmetry manifest with the two copies of $SU(2)$ acting respectively only on $A^i$ and $B^i$.\\


The holographic dual theory to a stack of $N$ D3-brane moving in the background given by $\mathbb{R}^4$ times the cone over $T^{1,1}$ is found to be an $\mathcal{N}=1$ superconformal quiver gauge theory with gauge group $SU(N)\times SU(N)$, with chiral superfields $A^i$, $B^i$ ($i=1,2$) transforming respectively in the bifundamentals $(\mathbf{N},\mathbf{\bar N})$, $(\mathbf{\bar N},\mathbf{N})$. The $SU(2)\times SU(2)$ isometry in the bulk is here implemented as a "flavour" global symmetry acting separately on $A^i$ and $B^i$. Moreover, the diagonal $U(1)$ reappears as a baryonic symmetry where again $A^i$ has charge $1$ and $B^i$ charge $-1$.\\


\cmmnt{relazione fra i campi della CFT e la posizione delle D-brane, moduli space mesonico e totale, un po' di più sul KW}

