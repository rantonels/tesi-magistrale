

Perhaps the most essential ingredient for the conception of the idea of holography was the fact that coincident D3-branes (a "stack") naturally feature a 4D gauge theory on their world-volume, where the 4D fields emerge from the modes of open strings stretching between them. In the simplest and most famous example, a stack of $N$ D3-branes is placed in otherwise Minkowski $\mathbb{R}^{1,9}$; the corresponding field theory is the maximally supersymmetric Yang-Mills in four dimensions (SYM4).\\

Setting the stack on a different background geometry instead gives rise to a large family of different field theories; a particularly interesting subset is given by spacetimes of the form:

\begin{equation} 
	M = \mathbb{R}^{1,3} \times X_6 
\end{equation}

where the $\mathbb{R}^{1,3}$ is parallel to the branes (and must be identified with the field theory spacetime) and $X_6$ is a 6-dimensional Calabi-Yau cone over a compact 5-fold base $Y_5$. By $X_6$ being a cone it is meant there exists a conical radial coordinate $r$ such that the metric on $X_6$ is of the form

\begin{equation}
	ds^2 = dr^2 + r^2 ds_5^2
	\label{}
\end{equation}



In this language, the SYM4 example above corresponds to $X_6 = \mathbb{R}^6 = \mathbb{C}^3$, which is (trivially) a cone over $\mathbb{S}^5$. This is the only case where $X_6$ turns out to be smooth; in general it will feature a conical singularity in the origin. Other choices for the base will typically yield theories with reduced (even minimal) supersymmetry, which are considerably more challenging to study.

\section{Superconformal field theory}

We now provide a short introduction to 4D conformal field theories, their supersymmetric variants, and the relevant terminology.

For a given $d$-dimensional spacetime with metric $g_{\mu\nu}$, conformal transformations are defined to be diffeomorphisms $x^\mu \rightarrow x'^\mu$ which leave the metric unchanged in form up to an $x$ dependent scalar function (a conformal factor):

\begin{equation}
%	ds'^2 = g'_{\mu\nu}(x') dx'^\mu dx'^\nu = \Omega(x) g_{\mu\nu}(x) dx^\mu dx^\nu
	ds^2 \rightarrow ds'{}^2 = \Omega(x) ds^2,
	\label{}
\end{equation}

or, equivalently:

\begin{equation}
	g'_{\mu\nu}(x') = \Omega(x) g_{\mu\nu}(x).
	\label{}
\end{equation}

The group of these transformations is known as the conformal group associated with the metric. Our case of interest is flat spacetime\footnote{to be precise, the conformal group is clearly equal for two metrics that are conformally equivalent ($g_{\mu\nu} = e^{\phi(x)} h_{\mu\nu}$), so that the group essentially depends only on the conformal class of the manifold. Therefore, what we will find for $\mathbb{R}^{1,D-1}$ will apply equally to all conformally flat metrics.}, $g_{\mu\nu} = \eta_{\mu\nu}$, and the corresponding group is commonly known as \emph{the} conformal group.

In $d>2$, the conformal group will turn out to be a finite-dimensional Lie group, of which we specify now the connected component. An obvious subgroup is maps that leave the metric unchanged, so Poincar\'e transformations, with generators $P_\mu$ and $J_{\mu\nu}$. A second easy guess is the subgroup with constant conformal factors, that is scale transformations or dilations

\begin{align}
	x^\mu \rightarrow \lambda x^\mu && \eta_{\mu\nu} \rightarrow \lambda^{-2} \eta_{\mu\nu}
	\label{}
\end{align}

whose generator is called $D$. To generate the whole conformal group a final class of transformations must be introduced, special conformal transformations, generated by $K_\mu$ and with finite action

\begin{equation}
	x^\mu \rightarrow b_\mu x^2 - 2x_\mu (b_\nu x^\nu).
	\label{}
\end{equation}

Together, $P_\mu$, $J_{\mu\nu}$, $D$ and $K_\mu$ generate the connected component of the conformal group in $D$ dimensions. The extension of the Poincar\'e algebra to the conformal one is characterized by the following additional commutators (using hermitian generators)

\begin{align}
	[J_{\mu\nu},K_\rho] 	&= 2i \eta_{\rho[\mu} K_{\nu]}	\label{kisvector}\\
	[J_{\mu\nu},D] 		&= 0 				\label{disscalar}\\
	[D, P_\mu] 		&= i P_\mu 			\label{pisraising}\\
	[D, K_\mu] 		&= -i K_\mu			\label{kisraising}
\end{align}

Equations \eqref{kisvector} and \eqref{disscalar} just confirm $K_\rho$ is a vector and $D$ is a scalar. \eqref{pisraising} and \eqref{kisraising} instead state that $P_\mu$ and $K_\mu$ are respectively raising and lowering operators for $D$. It is definitely worth of notice that this group is actually $SO(2,D)$, the Lorentz group in mixed signature $(2,D)$. This can be shown by combining the generators in

\begin{equation}
	J_{MN} = 
	\begin{pmatrix}
		J_{\mu\nu} 		& (K_\mu - P_\mu)/2	& - (K_\mu+P_\mu)/2 \\
		(P_\mu - K_\mu)/2	& 0			& D \\
		(K_\mu+P_\mu)/2		& -D			& 0
	\end{pmatrix}
	\label{}
\end{equation}

and then it can be verified that $J_{MN}$ satisfy the algebra of $\mathfrak{so}(2,D)$. This equivalence will be relevant when we will introduce AdS/CFT, since $SO(2,D)$ is also the isometry group of $AdS_{D+1}$.

A quantum field theory which has the conformal group as symmetries is called a conformal field theory (CFT). In such a theory, particles lie in irreducible representations of the conformal group; since the mass $P^2$ is not a Casimir for the whole group, it becomes useful to replace it with more relevant quantum numbers. Consider the dilation operator: in the quantum theory it will be represented by

\begin{equation}
	D = -i (x^\mu\partial_\mu + \Delta)
	\label{}
\end{equation}

where $\Delta$ gives the intrinsic scaling dimension of a field, which will in general transform as $\phi(x) \rightarrow \lambda^{\Delta} \phi(\lambda x)$. $\Delta$ is therefore a good quantum number. Considering the role of $P$ and $K$ as ladder operators, changing the conformal dimension by $\pm 1$, we can deduce states will come in multiplets of ever-increasing dimension $\Delta_{(0)} + n$, $n\geq 0$, and that the lowest-dimension state will be annihilated by $K_\mu$. Fields in the kernel of $K_\mu$ will be called primary, and others, obtained by applying powers of $P_\mu$ (hence, derivatives) will be called descendants.

A primary field is then identified by its conformal dimension and its representation under the Lorentz group, so, now specializing to $D=4$, by quantum numbers $(\Delta,j_L,j_R)$.

A classically conformal field theory very often fails to be conformal when quantized. This happens because the dilation symmetry is anomalous. Classical scale invariance clearly implies all couplings are adimensional; in the quantum theory these couplings $g^i$ will run under renormalization with a corresponding $\beta$ function, as in

\begin{equation}
	\frac{dg^i}{d\ln\mu} =: \beta^i(g)\,.
	\label{}
\end{equation}

The dependency of the running coupling on the energy scale, or equivalently the creation of a mass scale by dimensional transmutation, means the conformal symmetry is spoiled\footnote{We are here using conformal and scale (i.e. dilation) invariance interchangeably, but they are not identical. Conformal symmetry obviously includes dilations, but scale invariance $+$ Poincar\'e does not generate the whole conformal group, as special conformal transformations are independent. Scale invariant but not conformal theories are known explicitly\cite{scalebutnotconf}, but they are rare. We will work with the assumption dilation-invariant $\Rightarrow$ conformal.}. This happens for example in quantum chromodynamics, a classically conformal theory with a scale anomaly giving rise to the $\Lambda_\text{QCD}$ mass scale, or quantum electrodynamics where the scale is at the Landau pole. Since the N\"other current corresponding to dilations is the trace of the energy-momentum tensor, the anomaly will be detectable by the appearence of a nonzero matrix element $\langle T^{\mu}_{\;\,\mu} \rangle \propto \beta(g) \neq 0$.

Only if all the $\beta$ functions vanish identically, i.e. if the theory is finite, is quantum conformal invariance guaranteed. We will encounter an example of such a theory in section \ref{SYM4}. Otherwise the theory will only be conformal for specific values of the $g^i$ at which all the $\beta$ functions vanish, that is to say at fixed points. In general a quantum field theory will flow under renormalization from a non-conformal point towards an attracting IR fixed submanifold, the locus of $\left\{ \beta^i(g) = 0 \right\}$, called the conformal manifold.

An important point is that after the theory has regained its classical conformal symmetry after converging through RG flow to an IR fixed point, the quantum scaling dimensions $\Delta$ of operators will not coincide with the original value they had in the classical theory, the canonical dimension $\Delta_0$. They will be modified by quantum corrections that add an anomalous dimension

\begin{equation}
	\Delta = \Delta_0 + \gamma(g*)\,,\quad \gamma(g) = - \frac{1}{2}\frac{d \ln Z}{d \ln\mu},
	\label{}
\end{equation}

where $\sqrt Z$ renormalizes the wavefunction\footnote{It should be noted some authors prefer to define $\gamma = - \frac{d\ln Z}{d \ln \mu}$.}, and $g*$ are the values of the couplings at the conformal fixed point.

Having introduced the extension of the Poincar\'e group to $SO(2,4)$, we would like to press this further to include supersymmetry. The latter is implemented by adding $\mathcal{N} \leq 4$ Weyl supercharges $Q^A$, $\scj{Q}_A$ ($A=1,\ldots,\mathcal{N}$) to generate the super-Poincar\'e supergroup $\operatorname{ISO}(1,3|\,\mathcal{N})$. The superconformal group $SO(2,4|\,\ssn)$ is the minimal supergroup containing both. The first important feature is that a second set of supercharges $S_A$, $\scj{S}^A$ must be introduced to close the algebra, since

\begin{align}
	[K_\mu , Q^A] = - \sigma_\mu \scj{S}^A\,, && [P_\mu, S_A] = \scj{Q}_A  \scj\sigma_\mu \,;
	\label{}
\end{align}

so that superconformal symmetry $SO(2,4|\,\mathcal{N})$ involves twice as many supercharges as normal supersymmetry for a given $\mathcal{N}$. Another relevant excerpt from the table of commutators (which we do not reproduce in full) states $Q^A$ and $S_A$ are also ladder operators for dilations,

\begin{align}
	[D,Q^A] = \frac{i}{2} Q^A\,, && [D,S_A] = - \frac{i}{2} S_A\,,
	\label{}
\end{align}

raising and lowering the dimension $\Delta$ by $\pm 1/2$. In a superconformal field theory (SCFT) we then expect multiplets of dimension $\Delta = \Delta_0 + \frac{n}{2}$. Primary operators must now be annihilated by both $K_\mu$ and $S_A$, and are classified again by dimension and spin $(\Delta,j_L,j_R)$ but also by the $U(1)\times SU(\mathcal{N})$ R-symmetry quantum numbers $(R,\mathbf{r})$ ($\mathbf{r}$ denoting a generic irrep of $SU(\ssn)$). Then, by acting with the raising operator $Q^A$ charges one can reconstruct a finite-dimensional supermultiplet, as in normal supersymmetry. Instead, powers of $P_\mu$ reconstruct the infinite ladder of derivatives forming an infinite representation of the conformal group; these can be recombined into a field by Taylor expansion. In conclusion, an infinite representation of the superconformal group is given by a superfield

\begin{equation}
	\Phi_{\ldots} (x^\mu , \theta^A, \scj\theta^A)
	\label{}
\end{equation}

where $\ldots$ stands for Lorentz and R-charge-$SU(\ssn)$ indices for the primary.

\cmmnt{unitarity bounds \& R-charge <-> dimension}

\section{General features of D3-brane worldvolume theories}
The resulting gauge group will be $U(N)^g$, where $g = 1 + b_2(X_6) + b_4(X_6)$, and the theory will be populated by chiral fields in ``bifundamental'' representations, i.e. with an index in the fundamental of one $U(N)$ and a second in the antifundamental. These sort of theories are termed quiver gauge theories and they can be encoded in a quiver diagram, where $U(N)$ factors are denoted by nodes and bifundamental fields as directed arrows stretching between two nodes.\\

We will be in particular interested in the moduli spaces of these theories, so the spaces of distinct vacua. Because of supersymmetry, the quantum moduli space will often coincide with the classical one, which is the locus of the F-flatness condition:

\begin{equation}
	F^i = \pder{ W}{\phi_i} = 0
	\label{}
\end{equation}

where $W(\phi_i)$ is the superpotential function of the chiral fields $\phi_i$, and the D-flatness condition:

\begin{equation}
	D^a_{U(N)^g} = - \sum_i \phi_i^\dagger T^a \phi_i = 0
\end{equation}

where $T^a$ are the gauge generators. (The $U(N)^g$ subscript indicates the index $a$ spans over all generators of the $g$ factors of $U(N)$). The space $\mathcal{M}$ of simultaneous solution of the F and D-flatness conditions will be a complex manifold.\\

A subspace of $\mathcal{M}$ is given by the so-called mesonic moduli space $\mathcal{M}_m$. Points of $\mathcal{M}_m$ will correspond to the position of the $N$ branes on the background cone - therefore $\dim_\mathbb{C} \mathcal{M}_m = 3N$. The moduli space is not exhausted in the purely mesonic directions though; to investigate the remaining ``baryonic'' directions we first anticipate we will be mainly concerned with the IR limit, in which our theories will flow to (super) conformal field theories ((S)CFTs). In the IR limit, the abelian $U(1)$ factor in each $U(N)$ node ``freeze'' and become global baryonic symmetries. Therefore their D-flatness condition is relaxed and one is left with only the D-term for the $SU(N)^g$ part. So

\begin{align}
	D^a_{SU(N)^g} = 0 && D^i_{U(1)^g} = V^i
	\label{}
\end{align}

($i=1,\cdots,g$). $V^i$ are classically functions of the fields and in the quantum version will be gauge-invariant operators. Their $g$ VEVs $\langle V^i \rangle =: \xi^i$ will parametrize the missing flat directions of moduli space. To be precise, however, since the overall trace $U(1)$ (generated by the sum of the generators of the $g$ abelian trace factors) is completely decoupled, we have to impose $\sum \xi^i = 0$. Therefore that there are really only $g-1$ baryonic moduli, corresponding to $g(N^2-1)+1$ independent D-flatness conditions.\\

Thus we conclude $\dim \mathcal M = 3N + g - 1$. While the $3N$ mesonic directions have a direct geometrical interpretation as D3-brane movement, the baryonic directions correspond in terms of the superstring description to deformations of the $X_6$ background metric itself - generally resulting in a resolution of the conical singularity.


\cmmnt{spostare le cose dall'intro}

\section{Brane stack in $\mathbb{C}^3$ and $\mathcal{N}=4$ super-Yang-Mills} \label{SYM4}

If $X_6 = \mathbb{C}^3$, the branes are invariant under half of the $16 \times 2 = 32$ IIB supercharges. The only possibility for a 4D theory to have $16$ supercharges is to be an $\mathcal{N}=4$, superconformal field theory\footnote{Indeed, the number of supercharges is $2$ for the components of a 4D Majorana spinor, times $\mathcal{N}$, times a factor of $2$ since the superconformal algebra has twice the supercharges of the usual SUSY algebra.}. Moreover, the theory features gluons as the massless spin-1 modes for the sector of strings stretching between brane $i$ and brane $j$ so that the gauge group is $U(N)$, as seen in \ref{sec:branestacks}. The information that the theory is a $U(N)$ gauge theory and is maximally supersymmetric is enough to uniquely fix it.\\

In $\mathcal{N}=1$ language (which we employ even though the model has $\mathcal{N}=4$) the theory describes the dynamics of $U(N)$ gauge vector supermultiplets $A_\mu$ and three complex chiral superfields $(X^a)_{i\dot j}$, $a=1,2,3$ in the adjoint of the gauge group (we will frequently omit gauge indices). These are nothing else than the parametrization of the D3-branes' position in $\mathbb{C}^3$ and therefore transform in the fundamental of $SU(3)$. The superpotential is the only one allowed by gauge and $SU(3)$ invariance:

\begin{equation} W(X) = \epsilon_{abc} \Tr ( X^a X^b X^c) \end{equation}

and the quiver diagram is quite simple:

\tikzset{% arrow close to the source: the 0.2 determines where the arrow is drawn
  ->-/.style={decoration={markings, mark=at position 0.5 with {\arrow{stealth}}},
              postaction={decorate}},
}

\begin{figure}[!h]
	\centering
\begin{tikzpicture}[every node/.style={circle,draw},thick,scale=1.5]
  \node(NL) at (0,0){};
  \draw[->-](NL.north) to [loop above, min distance=20mm, in=-45, out=45, looseness=8]	(NL.south);
  \draw[->-](NL.north) to [loop above, min distance=24mm, in=-45, out=45, looseness=8]	(NL.south);
  \draw[->-](NL.north) to [loop above, min distance=28mm, in=-45, out=45, looseness=8]	(NL.south);
\end{tikzpicture}
\end{figure}

\cmmnt{nota sul moduli space triviale}

\section{The conifold and the Klebanov-Witten model} \label{sec:KW}

In \cite{KW_SCFT} the case of $X_6$ being the conifold was studied. The conifold is a specific Calabi-Yau 3-cone defined for example as the following variety in $\mathbb{C}^4$:

\begin{equation}
	z_1^2 + z_2^2 + z_3^2 + z_4^2 = 0
\end{equation}

or, after a simple change of variables:

\begin{equation}
	u v - xy = 0
	\label{}
\end{equation}

The base can be found by quotienting by dilations $z_i \rightarrow \lambda z_i$ (with $\lambda \in \mathbb{R}_+$) and turns out to be the homogeneous space $SO(4)/U(1) = SU(2)\times SU(2) / U(1)$, where the $U(1)$ is a diagonal subgroup generated by, say, $T^3_L + T^3_R$. We will therefore have $SU(2)\times SU(2)$ as part of the isometry group of both $Y_5$ and $X_6$, and thus will also appear as a global symmetry of the wordvolume theory. An equivalent description of the topology of the conifold is as a $U(1)$ bundle over $\mathbb{C}P^1 \times \mathbb{C}P^1$; in these terms the metric on the base that makes the cone Calabi-Yau is

\begin{equation}
	ds^2_5 = \frac{1}{9} (d\psi + \cos\theta_1 d\phi_1 + \cos\theta_2 d\phi_2)^2 + \frac{1}{6} (d\Omega_1^2 + d\Omega_2^2)
\end{equation}

where $\Omega_i^2 = d\theta_i^2 + \sin\theta_i^2 d\phi_i^2$ is the metric on the $\mathbb{C}P^1_i$, and $\psi$ is the fibral coordinate with period $4\pi$.\\

The corresponding gauge field theory on the worldvolume is a $U(N)\times U(N)$ field theory featuring two chiral doublets $A_i$, $B_j$ with $i,j = 1,2$ transforming in opposite bifundamentals, that is $A_i$ in $(N,\bar N)$ and $B_j$ in $(\bar N, N)$. Or more succintly, this can be depicted in a quiver diagram:

\begin{figure}[!h]
	\centering
\begin{tikzpicture}[every node/.style={circle,draw},thick,scale=1.5]
  \node(NL) at (0,0){};
  \node(NR) at (2,0){};
  \draw[->-](NL.north) to [bend left=60]	(NR.north);
  \draw[->-](NL.north) to [bend left=90]	(NR.north);
  \draw[->-](NR.south) to [bend left=60]	(NL.south);
  \draw[->-](NR.south) to [bend left=90]	(NL.south);
\end{tikzpicture}
\end{figure}

\cmmnt{~\\Inserire labels\\}

The $i$ and $j$ indices, instead, are acted upon respectively by the global left and right $SU(2)$ symmetries. Finally, $A$ and $B$ have R-charge $1/2$. The symmetries and R-charges fix the form of the superpotential:

\begin{equation}
	W = \frac{\lambda}{2} \epsilon^{ij} \epsilon^{kl} \Tr \left( A_i B_k A_j B_l \right)
	\label{}
\end{equation}

While this theory won't be in general superconformal, unlike the $\mathcal{N}=4$ SYM seen before, it will flow through renormalization in the IR to a conformal submanifold in the space of couplings $(\lambda,g_1,g_2)$, the locus where the $\beta$ functions for these three couplings vanish. It turns out these three conditions are all equivalent. In particular, requiring $\beta_{g_1} = 0$ and making use of the NSVZ expression for the $\beta$ function of a supersymmetric gauge theory, this unique condition is equivalent to


\begin{equation}
	3 T[\mathrm{Adj}] - \sum_i T[R_i] ( 1- 2\gamma_i) = 0
	\label{}
\end{equation}

where $T[R]$ is the Dynkin invariant of representation $R$, the sum is over charged fields and $\gamma_i$ is the anomalous dimension\footnote{Note we use the definition $\gamma = -\frac{1}{2}\frac{d\ln Z}{d \ln \mu}$, where $\sqrt Z$ renormalizes the field.}. When evaluating this, care should be taken with the fact that $A_i$ and $B_j$ have a $U(N)_2$ index which is uncharged under $U(N)_1$ and must be summed over. This gives, also noting $\gamma_{A_1} = \gamma_{A_2}$ and the same for $B$ because of the global symmetry:

\begin{equation}
	\gamma_A + \gamma_B + \frac{1}{2} = 0
	\label{}
\end{equation}

Being $\gamma_{A,B}$ functions of the couplings, this equation defines a critical 2-surface in parameter space. We note this equation is consistent with the relationship $\frac{3}{2}R-1 = \gamma$ between R-charge and the anomalous dimension of an operator in a SCFT, with the given assignment of R-charges.\\

Position in moduli space $\mathcal{M}$ should be parametrized by the expectation values of gauge-invariant operator (hadrons). In particular, the classical moduli space is given by the F-term and D-term conditions, which specialized to the particular case are

\begin{equation}
	\epsilon^{ij} A_i B_a A_j = \epsilon^{ij} B_i A_a B_j = 0
	\label{}
\end{equation}

\begin{equation}
	A_i A_i^\dagger - B_i B_i^\dagger = A^\dagger_i A_i - B^\dagger_i B_i = \xi \mathbbm{1}
	\label{}
\end{equation}

the equation have to be understood to hold for VEVs. Note the first and second D-term condition are respectively from the left and right gauge group. We set temporarily $\xi = 0$ to study the mesonic moduli space $\mathcal{M}_\text{mes}$. To reinterpret the F-flatness condition, we introduce the four matrices $\Phi_{ij} = A_i B_j$ and note

\begin{align}
	\left[ \Phi_{ij} , \Phi_{jk} \right] = 0\\
	\Phi_{11} \Phi_{22} = \Phi_{21}\Phi_{12}
	\label{}
\end{align}

which can be immediately checked to follow from the vanishing of the F-term. Since these commute, they can be simultaneously diagonalized and their $N$ eigenvalues (one for each brane) satisfy the conifold's equation:

\begin{equation}
	\phi^I_{11} \phi^I_{22} = \phi_{21}^I \phi_{12}^I
	\label{}
\end{equation}

so these quite literally parametrize the motion of the $N$ D3-branes on the background cone. These are actually VEVs of mesonic operators, mesons being generated by prototipical trace operators:

\begin{equation}
	M_{(ab\ldots),(ij\ldots)} =	\Tr\left( (A_a B_i)(A_b B_j)\cdots \right)
	\label{}
\end{equation}

(Note mesons are built by tracing over closed loops in the quiver diagram to make a gauge-invariant operator). All of these operators are actually expressable as products of $\Phi$ matrices, and as we've seen, only $3$ out of $4$ of those are independent. In the end, there are (accounting for gauge indices) $3N$ independent mesons whose VEVs parametrize mesonic moduli space, coincident with the $\operatorname{Sym}^N C$, where $C$ is the conifold.\\

Operators of non-zero baryon number can also be constructed by using the antisymmetric invariant gauge tensor $\epsilon^{a_1\ldots a_n}$, as such:

\begin{equation}
	B^A = \epsilon^{a_1\ldots a_n} \epsilon_{b_1\ldots b_n} A_{a_1}^{b_1} \ldots A^{b_n}_{a_n}
	\label{}
\end{equation}

where the indices on the $A$ fields are gauge indices and we have omitted the $SU(2)$ indices. There are only $N+1$ different assignment for the $SU(2)$ indices anyway, so that there are $N+1$ fundamental baryons of the form of $B^A$. The same could be done by swapping the two gauge groups and using $B$ fields, to get additional $N+1$ $B^B$ baryons. These all have baryon number $N$.\\

All gauge-invariant operators in the theory are built out of these fundamental mesons, fundamental baryons, and their respective antiparticles (made out of the conjugate fields $A^\dagger$, $B^\dagger$. However, as we have anticipated, we only expect $g-1 = 1$ baryonic VEV to be independent. This VEV will be associated with the resolution of the cone singularity into a $\mathbb{CP}^1$, and will essentially coincide with $\xi$. To see an example of this deformation of the background geometry, let us set $\xi$ to a constant nonzero value. Then hypothesizing the $A_1, A_2, B_1, B_2$ matrices commute, applying F-term conditions we get that each set of eigenvalues satisfies

\begin{equation}
	a_1/a_2 = b_1/b_2
	\label{}
\end{equation}

So that $a_i$ and $b_i$ are proportional vectors of $\mathbb{C}^2$, therefore

\begin{equation}
	a_i = a e^{i\theta_A} n_i, \quad \quad b e^{i\theta_B} n_i
	\label{}
\end{equation}

where $a,b$ are real and $n_i$ belongs to a $\mathbb{CP}^1$. The phases are cancelled by modding gauge invariance, and $a$ and $b$ then are involved in the D-term:

\begin{equation}
	a^2 - b^2 = \xi
	\label{}
\end{equation}

so that essentially our mesonic VEVs are composed of $N$ copies of one non-compact radial coordinate (say, $a$) and a point on $\mathbb{CP}^1$. This means the conical singularity has disappeared to be replaced by a two-cycle on which the branes can move.\\

The explicit form of the baryon generating this deformation in terms of the fundamental hadrons is very challenging to determine\cite{Forcella}; fortunately, we will not need it for our purposes. In any case, all of this information will be clarified in the context of holography.

\section{The $Y^{(2,0)}$ orbifold theory}\label{sec:squares}

The same construction on a $\mathbb{Z}_2$ orbifold of the conifold yields a quiver gauge theory which will be the main interest of this work. The geometry of the base of the cone is very simply introduced in polar coordinates as 

\begin{equation}
	ds^2_5 = \frac{1}{9} (d\psi + \cos\theta_1 d\phi_1 + \cos\theta_2 d\phi_2)^2 + \frac{1}{6} (d\Omega_1^2 + d\Omega_2^2)
\end{equation}

i.e., exactly the same metric in form as the conifold, but with $\psi$ now with period $2\pi$. This background and the resulting worldvolume field theory are just one entry $Y^{2,0}$ of an infinite class $Y^{p,q}$ of examples introduced in \cite{benvenutiInfinite}.\\

The quiver diagram ``splits'' to yield four doublets of bifundamental chiral fields stretching in a square between four nodes:\\


\begin{figure}[!h]
	\centering
\begin{tikzpicture}[every node/.style={circle,draw},thick,scale=1.5]
  \node(NL) at (0,0){};
  \node(NR) at (2,0){};
  \node(NDR) at (2,2){};
  \node(NDL) at (0,2){};
  \draw[->-](NL.east) to [bend right=20]	(NR.west);
  \draw[->-](NL.east) to [bend left=20]	(NR.west);
  \draw[->-](NR.north) to [bend right=20]	(NDR.south);
  \draw[->-](NR.north) to [bend left=20]	(NDR.south);
  \draw[->-](NDR.west) to [bend right=20]	(NDL.east);
  \draw[->-](NDR.west) to [bend left=20]	(NDL.east);
  \draw[->-](NDL.south) to [bend right=20]	(NL.north);
  \draw[->-](NDL.south) to [bend left=20]	(NL.north);
\end{tikzpicture}
\end{figure}

\cmmnt{inserire labels}\\

and the superpotential can be shown to have the form

\begin{equation}
	W = \lambda \epsilon^{ij} \epsilon^{kl} \Tr\left( A_i B_k C_j D_l \right)
	\label{}
\end{equation}

from which it's clear that the $SU(2) \times SU(2)$ isometry of the cone, corresponding to a global symmetry of the field theory, must now act with the left factor on $A_i$ and $C_i$, and the right on $B_i$ and $D_i$. This time three of the four gauge $\beta$ functions are independent:

\begin{align}
	\gamma_A + \gamma_D + \frac{1}{2} = 0 \\
	\gamma_B + \gamma_A + \frac{1}{2} = 0 \\
	\gamma_C + \gamma_B + \frac{1}{2} = 0 \\
\end{align}

$\beta_\lambda = 0$ is also not independent. At any superconformal point, $\frac{3}{2}R - 1 = \gamma$, so that the condition that $W$ be scale invariant, which is equivalent to it having R-charge $2$, becomes

\begin{equation}
	2 = R_W = R_A + R_B + R_C + R_D \Rightarrow \gamma_A + \gamma_B + \gamma_C + \gamma_D + 1 = 0
	\label{}
\end{equation}

which is indeed equivalent to the above system. Three independent equations in a five-parameter space define, again, a critical 2-submanifold.\\

Turning to the investigation of the moduli space, the F-term condition for the given superpotential read

\begin{align}
	\begin{split}
	A_\alpha B_\sigma C_\beta \epsilon^{\alpha\beta} = 0 \\
	B_\alpha C_\sigma D_\beta \epsilon^{\alpha\beta} = 0 \\
	C_\alpha D_\sigma A_\beta \epsilon^{\alpha\beta} = 0 \\
	D_\alpha A_\sigma B_\beta \epsilon^{\alpha\beta} = 0 
	\label{}
\end{split}
\end{align}

while the vanishing of the D-term takes the form

\begin{align}
	\begin{split}
		A_i A^\dagger_i - B_i B^\dagger_i = \xi_1 \mathbbm{1} \\
		B_i B^\dagger_i - C_i C^\dagger_i = \xi_2 \mathbbm{1} \\
		C_i C^\dagger_i - D_i D^\dagger_i = \xi_3 \mathbbm{1} \\
		D_i D^\dagger_i - A_i A^\dagger_i = \xi_4 \mathbbm{1}
	\end{split}
	\label{}
\end{align}

with the constraint $\sum_i \xi_i = 0$ (obvious by summing the four equations). To sketch out mesonic moduli space, again we make the simplifying assumption the eight $A, B, C, D$ matrices commute and can be simultaneously diagonalized. So, for each of the $N$ rows of corresponding eigenvalues,

\begin{align}
	a_1/a_2 = c_1/c_2 && b_1/b_2 = d_1 / d_2
	\label{}
\end{align}

thus again $a_\alpha \propto c_\alpha$ and $b_\alpha \propto d_\alpha$, so we can ``projectivize'':

\begin{align}
		a_\alpha = a \, e^{i\theta_A} n_\alpha && b_\alpha = b \, e^{i\theta_B} n_\alpha \\
		c_\alpha = c \, e^{i\theta_C} m_\alpha && d_\alpha = d \, e^{i\theta_D} m_\alpha 
\end{align}

and again the phases are modded out by gauge symmetry, and the $a, b, c, d$ real numbers are reduced to a single coordinate (schematically $r^2$) by the three independent D-flatness conditions. Therefore the resolved geometry of the singularity is now $\mathbb{CP}^1 \times \mathbb{CP}^1$, with a clear correspondence with the explicit metric of the resolved $Y^{2,0}$.\\

In this case, the presence of $g-1 = 3$ independent $\xi$ parameters (matching with three independent baryons) should perplex, as the resolution of the singularity should be controlled by the \emph{two} volumes of the spheres. In fact, the general Calabi-Yau deformation of the $Y^{2,0}$ metric will indeed depend on two moduli. In this case, the third modulus is not interpretable as due to deformation of the background metric, but is actually connected to the flux of IIB two-form fields. We will review this fact in a holographic context.\\

For completeness we adapt the construction of hadronic operators. We note fundamental mesons are now built using $ABCD$ loops (omitting $SU(2)^2$ indices):

\begin{equation}
	M = \Tr \left( (ABCD)(ABCD) \ldots \right)
	\label{}
\end{equation}

and four classes of fundamental baryons can be introduced as before

\begin{equation}
	B^A = \epsilon^{a_1\ldots a_n}\epsilon_{b_1 \ldots b_n} A^{b_1}_{a_1} \ldots A^{b_n}_{a_n}
	\label{}
\end{equation}

and this can be repeated for $B$, $C$, $D$. Similarly to the previous case, we expect only three baryons to be truly indepedent and a suitable triple of combinations should generate the three aforementioned flat shifts.



