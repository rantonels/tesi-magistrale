

Perhaps the most essential ingredient for the conception of the idea of holography was the fact that coincident D3-branes (a "stack") naturally feature a 4D gauge theory on their world-volume, where the 4D fields emerge from the modes of open strings stretching between them. In the simplest and most famous example, a stack of $N$ D3-branes is placed in otherwise Minkowski $\mathbb{R}^{1,9}$; the corresponding field theory is the maximally supersymmetric Yang-Mills in four dimensions (SYM4).\\

Setting the stack on a different background geometry instead gives rise to a large family of different field theories; a particularly interesting subset is given by spacetimes of the form:

\begin{equation} 
	M = \mathbb{R}^{1,3} \times X_6 
\end{equation}

where the $\mathbb{R}^{1,3}$ is parallel to the branes (and must be identified with the field theory spacetime) and $X_6$ is a 6-dimensional Calabi-Yau cone over a compact 5-fold base $Y_5$. In this language, the SYM4 example above corresponds to $X_6 = \mathbb{R}^6 = \mathbb{C}^3$, which is (trivially) a cone over $\mathbb{S}^5$. This is the only case where $X_6$ turns out to be smooth; in general it will feature a conical singularity in the origin. Other choices for the base will typically yield theories with reduced (even minimal) supersymmetry, which are considerably more challenging to study.\\

The resulting gauge group will be $U(N)^g$, where $g = 1 + b_2(X_6) + b_4(X_6)$, and the theory will be populated by chiral fields in ``bifundamental'' representations, i.e. with an index in the fundamental of one $U(N)$ and a second in the antifundamental. These sort of theories are termed quiver gauge theories and they can be encoded in a quiver diagram, where $U(N)$ factors are denoted by nodes and bifundamental fields as directed arrows stretching between two nodes.\\

We will be in particular interested in the moduli spaces of these theories, so the spaces of distinct vacua. Because of supersymmetry, the quantum moduli space will often coincide with the classical one, which is the locus of the F-flatness condition:

\begin{equation}
	F^i = \pder{ W}{\phi_i} = 0
	\label{}
\end{equation}

where $W(\phi_i)$ is the superpotential function of the chiral fields $\phi_i$, and the D-flatness condition:

\begin{equation}
	D^a_{U(N)^g} = - \sum_i \phi_i^\dagger T^a \phi_i = 0
\end{equation}

where $T^a$ are the gauge generators. (The $U(N)^g$ subscript indicates the index $a$ spans over all generators of the $g$ factors of $U(N)$). The space $\mathcal{M}$ of simultaneous solution of the F and D-flatness conditions will be a complex manifold.\\

A subspace of $\mathcal{M}$ is given by the so-called mesonic moduli space $\mathcal{M}_m$. Points of $\mathcal{M}_m$ will correspond to the position of the $N$ branes on the background cone - therefore $\dim_\mathbb{C} \mathcal{M}_m = 3N$. The moduli space is not exhausted in the purely mesonic directions though; to investigate the remaining ``baryonic'' directions we first anticipate we will be mainly concerned with the IR limit, in which our theories will flow to (super) conformal field theories ((S)CFTs). In the IR limit, the abelian $U(1)$ factor in each $U(N)$ node ``freeze'' and become global baryonic symmetries. Therefore their D-flatness condition is relaxed and one is left with only the D-term for the $SU(N)^g$ part. So

\begin{align}
	D^a_{SU(N)^g} = 0 && D^i_{U(1)^g} = V^i
	\label{}
\end{align}

($i=1,\cdots,g$). $V^i$ are classically functions of the fields and in the quantum version will be gauge-invariant operators. Their $g$ VEVs $\langle V^i \rangle =: \xi^i$ will parametrize the missing flat directions of moduli space. To be precise, however, since the overall trace $U(1)$ (generated by the sum of the generators of the $g$ abelian trace factors) is completely decoupled, we have to impose $\sum \xi^i = 0$. Therefore that there are really only $g-1$ baryonic moduli, corresponding to $g(N^2-1)+1$ independent D-flatness conditions.\\

Thus we conclude $\dim \mathcal M = 3N + g - 1$. While the $3N$ mesonic directions have a direct geometrical interpretation as D3-brane movement, the baryonic directions correspond in terms of the superstring description to deformations of the $X_6$ background metric itself - generally resulting in a resolution of the conical singularity.

\section{Brane stack in $\mathbb{C}^3$ and $\mathcal{N}=4$ super-Yang-Mills}

If $X_6 = \mathbb{C}^3$, the branes are invariant under half of the $16 \times 2 = 32$ IIB supercharges. The only possibility for a 4D theory to have $16$ supercharges is to be an $\mathcal{N}=4$, superconformal field theory\footnote{Indeed, the number of supercharges is $2$ for the components of a 4D Majorana spinor, times $\mathcal{N}$, times a factor of $2$ since the superconformal algebra has twice the supercharges of the usual SUSY algebra.}. Moreover, the theory features gluons as the massless spin-1 modes for the sector of strings stretching between brane $i$ and brane $j$ so that the gauge group is $U(N)$, as seen in \ref{sec:branestacks}. The information that the theory is a $U(N)$ gauge theory and is maximally supersymmetric is enough to uniquely fix it.\\

In $\mathcal{N}=1$ language (which we employ even though the model has $\mathcal{N}=4$) the theory describes the dynamics of $U(N)$ gauge vector supermultiplets $A_\mu$ and three complex chiral superfields $(X^a)_{i\dot j}$, $a=1,2,3$ in the adjoint of the gauge group (we will frequently omit gauge indices). These are nothing else than the parametrization of the D3-branes' position in $\mathbb{C}^3$ and therefore transform in the fundamental of $SU(3)$. The superpotential is the only one allowed by gauge and $SU(3)$ invariance:

\begin{equation} W(X) = \epsilon_{abc} \Tr ( X^a X^b X^c) \end{equation}

and the quiver diagram is quite simple:

\tikzset{% arrow close to the source: the 0.2 determines where the arrow is drawn
  ->-/.style={decoration={markings, mark=at position 0.5 with {\arrow{stealth}}},
              postaction={decorate}},
}

\begin{figure}[!h]
	\centering
\begin{tikzpicture}[every node/.style={circle,draw},thick,scale=1.5]
  \node(NL) at (0,0){};
  \draw[->-](NL.north) to [loop above, min distance=20mm, in=-45, out=45, looseness=8]	(NL.south);
  \draw[->-](NL.north) to [loop above, min distance=24mm, in=-45, out=45, looseness=8]	(NL.south);
  \draw[->-](NL.north) to [loop above, min distance=28mm, in=-45, out=45, looseness=8]	(NL.south);
\end{tikzpicture}
\end{figure}

\cmmnt{nota sul moduli space triviale}

\section{The conifold and the Klebanov-Witten model}

In \cite{KW_SCFT} the case of $X_6$ being the conifold was studied. The conifold is a specific Calabi-Yau 3-cone defined for example as the following variety in $\mathbb{C}^4$:

\begin{equation}
	z_1^2 + z_2^2 + z_3^2 + z_4^2 = 0
\end{equation}

The base can be found by quotienting by dilations $z_i \rightarrow \lambda z_i$ (with $\lambda \in \mathbb{R}_+$) and turns out to be the homogeneous space $SO(4)/U(1) = SU(2)\times SU(2) / U(1)$, where the $U(1)$ is a diagonal subgroup generated by, say, $T^3_L + T^3_R$. We will therefore have $SU(2)\times SU(2)$ as part of the isometry group of both $Y_5$ and $X_6$, and thus will also appear as a global symmetry of the wordvolume theory. An equivalent description of the topology of the conifold is as a $U(1)$ bundle over $\mathbb{C}P^1 \times \mathbb{C}P^1$; in these terms the metric on the base that makes the cone Calabi-Yau is

\begin{equation}
	ds^2_5 = \frac{1}{9} (d\psi + \cos\theta_1 d\phi_1 + \cos\theta_2 d\phi_2)^2 + \frac{1}{6} (d\Omega_1^2 + d\Omega_2^2)
\end{equation}

where $\Omega_i^2 = d\theta_i^2 + \sin\theta_i^2 d\phi_i^2$ is the metric on the $\mathbb{C}P^1_i$, and $\psi$ is the fibral coordinate with period $4\pi$.\\

The corresponding gauge field theory on the worldvolume is a $U(N)\times U(N)$ field theory featuring two chiral doublets $A_i$, $B_j$ with $i,j = 1,2$ transforming in opposite bifundamentals, that is $A_i$ in $(N,\bar N)$ and $B_j$ in $(\bar N, N)$. Or more succintly, this can be depicted in a quiver diagram:

\begin{figure}[!h]
	\centering
\begin{tikzpicture}[every node/.style={circle,draw},thick,scale=1.5]
  \node(NL) at (0,0){};
  \node(NR) at (2,0){};
  \draw[->-](NL.north) to [bend left=60]	(NR.north);
  \draw[->-](NL.north) to [bend left=90]	(NR.north);
  \draw[->-](NR.south) to [bend left=60]	(NL.south);
  \draw[->-](NR.south) to [bend left=90]	(NL.south);
\end{tikzpicture}
\end{figure}

\cmmnt{~\\Inserire labels\\}

The $i$ and $j$ indices, instead, are acted upon respectively by the global left and right $SU(2)$ symmetries. Finally, $A$ and $B$ have R-charge $1/2$. The symmetries and R-charges fix the form of the superpotential:

\begin{equation}
	W = \frac{\lambda}{2} \epsilon^{ij} \epsilon^{kl} \Tr \left( A_i B_k A_j B_l \right)
	\label{}
\end{equation}

While this theory won't be in general superconformal, unlike the $\mathcal{N}=4$ SYM seen before, it will flow through renormalization in the IR to a conformal submanifold in the space of couplings $(\lambda,g_1,g_2)$, the locus where the $\beta$ functions for these three couplings vanish. It turns out these three conditions are all equivalent. In particular, requiring $\beta_{g_1} = 0$ and making use of the NSVZ expression for the $\beta$ function of a supersymmetric gauge theory, this unique condition is equivalent to


\begin{equation}
	3 T[\mathrm{Adj}] - \sum_i T[R_i] ( 1- 2\gamma_i) = 0
	\label{}
\end{equation}

where $T[R]$ is the Dynkin invariant of representation $R$, the sum is over charged fields and $\gamma_i$ is the anomalous dimension\footnote{Note we use the definition $\gamma = -\frac{1}{2}\frac{d\ln Z}{d \ln \mu}$, where $\sqrt Z$ renormalizes the field.}. When evaluating this, care should be taken with the fact that $A_i$ and $B_j$ have a $U(N)_2$ index which is uncharged under $U(N)_1$ and must be summed over. This gives, also noting $\gamma_{A_1} = \gamma_{A_2}$ and the same for $B$ because of the global symmetry:

\begin{equation}
	\gamma_A + \gamma_B + \frac{1}{2} = 0
	\label{}
\end{equation}

Being $\gamma_{A,B}$ functions of the couplings, this equation defines a critical 2-surface in parameter space. We note this equation is consistent with the relationship $\frac{3}{2}R-1 = \gamma$ between R-charge and the anomalous dimension of an operator in a SCFT, with the given assignment of R-charges.

\cmmnt{descrizione moduli space}

\section{The $Y^{(2,0)}$ orbifold theory}\label{sec:squares}

The same construction on a $\mathbb{Z}_2$ orbifold of the conifold yields a quiver gauge theory which will be the main interest of this work. The geometry of the base of the cone is very simply introduced in polar coordinates as 

\begin{equation}
	ds^2_5 = \frac{1}{9} (d\psi + \cos\theta_1 d\phi_1 + \cos\theta_2 d\phi_2)^2 + \frac{1}{6} (d\Omega_1^2 + d\Omega_2^2)
\end{equation}

i.e., exactly the same metric in form as the conifold, but with $\psi$ now with period $2\pi$. This background and the resulting worldvolume field theory are just one entry $Y^{2,0}$ of an infinite class $Y^{p,q}$ of examples introduced in \cite{benvenutiInfinite}.\\

The quiver diagram ``splits'' to yield four doublets of bifundamental chiral fields stretching in a square between four nodes:\\


\begin{figure}[!h]
	\centering
\begin{tikzpicture}[every node/.style={circle,draw},thick,scale=1.5]
  \node(NL) at (0,0){};
  \node(NR) at (2,0){};
  \node(NDR) at (2,2){};
  \node(NDL) at (0,2){};
  \draw[->-](NL.east) to [bend right=20]	(NR.west);
  \draw[->-](NL.east) to [bend left=20]	(NR.west);
  \draw[->-](NR.north) to [bend right=20]	(NDR.south);
  \draw[->-](NR.north) to [bend left=20]	(NDR.south);
  \draw[->-](NDR.west) to [bend right=20]	(NDL.east);
  \draw[->-](NDR.west) to [bend left=20]	(NDL.east);
  \draw[->-](NDL.south) to [bend right=20]	(NL.north);
  \draw[->-](NDL.south) to [bend left=20]	(NL.north);
\end{tikzpicture}
\end{figure}

\cmmnt{inserire labels}\\

and the superpotential can be shown to have the form

\begin{equation}
	W = \lambda \epsilon^{ij} \epsilon^{kl} \Tr\left( A_i B_k C_j D_l \right)
	\label{}
\end{equation}

from which it's clear that the $SU(2) \times SU(2)$ isometry of the cone, corresponding to a global symmetry of the field theory, must now act with the left factor on $A_i$ and $C_i$, and the right on $B_i$ and $D_i$. This time three of the four gauge $\beta$ functions are independent:

\begin{align}
	\gamma_A + \gamma_D + \frac{1}{2} = 0 \\
	\gamma_B + \gamma_A + \frac{1}{2} = 0 \\
	\gamma_C + \gamma_B + \frac{1}{2} = 0 \\
\end{align}

$\beta_\lambda = 0$ is also not independent. At any superconformal point, $\frac{3}{2}R - 1 = \gamma$, so that the condition that $W$ be scale invariant, which is equivalent to it having R-charge $2$, becomes

\begin{equation}
	2 = R_W = R_A + R_B + R_C + R_D \Rightarrow \gamma_A + \gamma_B + \gamma_C + \gamma_D + 1 = 0
	\label{}
\end{equation}

which is indeed equivalent to the above system. Three independent equations in a five-parameter space define, again, a critical 2-submanifold.

\cmmnt{~\\modulispace}
