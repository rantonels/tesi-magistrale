\begin{titlepage}
\begin{center}
 
% Upper part of the page
\includegraphics[scale=.5]{images/logoBlack}
 
\textsc{\LARGE Università degli Studi di Padova}\\[1.5cm]
 
\textsc{\Large Dipartimento di Fisica e Astronomia\\[0.2cm] Corso di Laurea Magistrale in Fisica}\\[2cm]
  
% Title

{\Huge \doublespacing \bfseries \begin{spacing}{1}{Holographic effective field theories: a case study}\end{spacing}}
~\\[4cm]
 
% Author and supervisor
\begin{minipage}{0.4\textwidth}
\begin{flushleft} \large
\emph{Laureando:}\\
Riccardo \textsc{Antonelli}
\end{flushleft}
\end{minipage}
\begin{minipage}{0.4\textwidth}
\begin{flushright} \large
\emph{Relatore:} \\
Luca \textsc{Martucci}
\end{flushright}
\end{minipage}
 
\vfill
 
% Bottom of the page
{\large Anno accademico 2015/2016\\
	\cmmnt{bozza compilata \today}}
 
\end{center}

\end{titlepage}

%\begin{frontespizio}
%\Universita{Padova}
%\Facolta{Scienze Matematiche, Fisiche e Naturali}
%\Corso[Laurea]{Matematica}
%\Titoletto{Tesi di laurea}
%\Titolo{Equivalenze fra categorie di moduli\\
%e applicazioni}
%\Candidato[145822]{Enrico Gregorio}
%\Relatore{Ch.mo Prof.~Adalberto Orsatti}
%\Annoaccademico{19??-19??}
%\end{frontespizio}

%\begin{abstract}

%\lipsum[1]

%\cmmnt{bozza compilata il giorno \today}\\

%\end{abstract}

\thispagestyle{plain}
\begin{center}
   % \Large
   % \textbf{Thesis Title}
    
   % \vspace{0.4cm}
   % \large
   % Thesis Subtitle
    
   % \vspace{0.4cm}
   % \textbf{Author Name}
    
   % \vspace{0.9cm}
    \textbf{Abstract}
\end{center}
The identification of the low-energy effective field theory associated with a given microscopic strongly interacting theory constitutes a fundamental problem in theoretical physics, which is particularly hard when the theory is not sufficiently constrained by symmetries.
Recently, a new approach has been proposed, which addresses this problem for a large class of four-dimensional minimally supersymmetric strongly coupled superconformal field theories, admitting a dual weakly coupled holographic description in string theory. This approach provides a precise prescription for the holographic derivation of the associated effective field theories. The aim of the thesis is to further explore this approach by focusing on a specific model, whose effective field theory has not been investigated so far. \cmmnt{(modificare abstract alla fine del lavoro.)}

