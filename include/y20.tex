We now finally tackle the explicit determination of the Lagrangian of the effective field theory describing the low-energy dynamics of the $Y^{2,0}$ field theory described in \ref{sec:squares}, through techniques introduced in \cite{MZ}.

\section{K\"ahler moduli}

The general Calabi-Yau deformation of the $Y^{2,0}$ cone is already well-known (\cmmnt{refs}) in real coordinates as:

\begin{equation}
ds^2 = \kappa^{-1}(r)dr^2 + \frac{1}{9} \kappa(r) r^2 (d\psi + \cos\theta_L d\phi_L + \cos\theta_R d\phi_R)^2 + \frac{1}{6} r^2 d\Omega_L^2 + \frac{1}{6}(r^2+a^2) d\Omega_R^2 \label{y20metric} \end{equation}
\begin{equation}
	\kappa(r) = \frac{1 + \frac{9a^2}{r^2} - \frac{b^6}{r^6}}{1+ \frac{6a^2}{r^2}}
\end{equation}

with $a,b$ the two unique real moduli. The topology is that of an $\mathbb{R}^2$ bundle over $\mathbb{S}^2 \times \mathbb{S}^2$.\\

For the purpose of building the effective theory, however, this metric must be rewritten in a complex chart. To that end, we try to find the general CY metric on a $\mathbb{C} \rightarrow \mathbb{CP}^1 \times \mathbb{CP}^1$ bundle; on the spheres of the base we take the round metric, given by the K\"ahler forms $j^L$ and $j^R$. It's easy to verify explicitly that, given any set of complex coordinates on the base $(y_L,y_R)$,

\begin{equation}
	j^L \wedge j^R = e^{-\Lambda k} dy^L \wedge dy^R \wedge d\bar y^L \wedge d\bar y^R
\end{equation}

with $k = k^L + k^R$ the total base potential, and for some $\Lambda$ depending on the overall size of the spheres (for unit radius, $\Lambda = 1$).\\

We also introduce the function $t$ of the fibral coordinate $\zeta$ as 

\begin{equation}
	t = |\zeta|^2 e^{\Lambda k}
\end{equation}

We then start from the following ansatz for the K\"ahler potential:

\begin{equation}
	k_X = f(t) + \alpha k^L + \tilde\alpha k^R
\end{equation}

where $\alpha,\tilde\alpha$, controlling the volume at $t=0$ of the base 2-spheres, parametrize the Ricci-flat K\"ahler resolutions of the cone. We are now set to prove that there is always an $f(t;\alpha,\tilde\alpha)$ that makes the metric Ricci-flat.\\

The corresponding K\"ahler form is straightforward:

\begin{equation}
	J = A^L j^L + A^R j^R + i e^{\Lambda k} (f' + t f'') (d\zeta + \Lambda \zeta \partial k) \wedge (\mathrm{c.c.})
\end{equation}

\newcommand{\fibral}{e^3 \wedge \bar e^{\bar 3}}

with $A^L = \alpha + \Lambda t f'(t)$ and  $A^R = \tilde\alpha + \Lambda t f'(t)$. This is more simply $J = J_M + M \fibral$, where $J_M$ is the purely basal part, $e^3 = d\zeta + \Lambda \zeta \partial k$ and $M$ is a scalar factor. The volume form is then clearly

\begin{equation}
	J \wedge J \wedge J = 3 A^L A^R M \, j^1 \wedge j^2 \wedge \fibral
\end{equation}

as all other terms in the cube vanish. Since the volume form is $\sqrt{\det g} \, d\Omega \wedge \bar \Omega$, with $\Omega = d\zeta \wedge dy^L \wedge dy^R$, and the Ricci tensor for a K\"ahler space is proportional to $\partial \bar \partial \ln \det g$, then the condition for Ricci-flatness is equivalent to the prefactor of $\Omega \wedge \bar \Omega$ in $J\wedge J \wedge J$ being constant, that is to say

\begin{equation}
	(\alpha + \Lambda t f')(\tilde{\alpha} + \Lambda t f') \frac{d}{dt} (\Lambda t f') = c \label{rflatcondition}
\end{equation}

or, having defined $y = \Lambda t f'$,

\begin{equation}
	(\alpha + y)(\tilde{\alpha} + y) y' = c
\end{equation}

Since $f(t)$ must be regular as $t=0$, and $f' = \frac{y}{\Lambda t}$, it must be that $y$ goes to zero at least as fast as $t$ as $t\rightarrow 0$; this condition eliminates the freedom from the constant of integration for equation \ref{rflatcondition}. The constant $c$ on the other hand can be readily reabsorbed into a $t$ rescaling. Therefore there should be a unique $y$ (and so a unique $f$ up to unconsequential constant shifts) that gives a Ricci-flat metric. Let us see this explicitly: we integrate \ref{rflatcondition} to obtain

\begin{equation}
	\frac{y^3}{3} + \frac{\alpha + \tilde{\alpha}}{2} y^2 + \alpha \tilde{\alpha} y = ct + d \label{rflatintegrated}
\end{equation}

And then the regularity condition $y(0)=0$ is satisfied with $d=0$, and this cubic equation for $y$ is immediately seen to have one single real solution for any positive values of $\alpha^i$, $c$.\\

Before exhibiting the explicit form of $y(t;\alpha,\tilde\alpha)$, let us express the K\"ahler form in terms of $y$ and show it's actually equal to the real-coordinate metric \ref{y20metric}. We have

\begin{align}
	\label{Jintermsofy}
	J & =  (\alpha + y) j^1 + (\tilde{\alpha} + y) j^2 + \frac{ie^{\Lambda k}}\Lambda y' \fibral\\
	  & =  (\alpha + y) j^1 + (\tilde{\alpha} + y) j^2 + \frac{ie^{\Lambda k} \,c}{\Lambda (\alpha + y)(\tilde{\alpha} + y)} \fibral
\end{align}

Now, we parametrize the fiber as $\zeta = e^{-\Lambda k/2} t^{1/2} e^{i\psi}$, and the 2-spheres with spherical coordinates $\theta_i$, $\phi_i$ which fixes $\Lambda = 1$. Then the metric corresponding to $J$ is

\begin{equation}
	ds^2 = A^L d\Omega^2_L + A^R d\Omega^2_R + \frac{y'}{t} \left( \frac{dt^2}{4} + t^2 (d\psi + \sigma)^2 \right)
\end{equation}

Where $\sigma = -i\frac{\Lambda}{2}(\partial k - \bar \partial k)$. But the $t-\psi$ part is simply

\begin{equation}
	ds^2 = \frac{1}{4y't} dy^2 + (y' t) (d\psi + \sigma)^2
\end{equation}

Exploiting both \ref{rflatcondition} and its integrated form \ref{rflatintegrated} we rewrite

\begin{align}
	y't & = \frac{1}{A^L A^R} \left( \frac{y^3}{3} + \frac{\alpha + \tilde{\alpha}}{2} y^2 + \alpha \tilde{\alpha} y \right)\\
	& = 3cr^2 \frac{1+ \frac{3}{2} \frac{\tilde{\alpha} - \alpha}{r^2} + \frac{\alpha^{2}(\alpha - 3 \tilde{\alpha})}{2r^6} }{1+ \frac{\tilde{\alpha} -\alpha}{r^2} }\\
	& = 3cr^2 \kappa(r)
\end{align}

provided we make the identifications

\begin{align}
	a^2 = \frac{1}{6}(\tilde{\alpha} - \alpha) && b^6 = \frac{\alpha^{2}(3\tilde{\alpha}-\alpha)}2
	\label{<++>}
\end{align}

The final coordinate change to the $r$ coordinate is then given by $r^2 = A^L = y + \alpha$ - note this renders the inherent symmetry between the left and right 2-cycles non-manifest. The resulting metric, after taking $c=1/3$, is precisely \ref{y20metric}. Thus, as the latter is the most general Calabi-Yau deformation of the $Y^{2,0}$ cone, we have to conclude that the two-parameter family of metrics \ref{Jintermsofy} in complex coordinates coincides with it.\\

Now we're left with solving for the explicit form of $y$. Switching temporarily to $z = y + (\alpha + \tilde\alpha)/2$ equation \ref{rflatintegrated} is brought into depressed form:

\begin{equation}
	z^3 - \frac{3}4 (\alpha - \tilde\alpha)^2 = ct + D
	\label{depressed}
\end{equation}

Where

\begin{equation}
	D = \frac{1}{12}(-\alpha^3 + 3 \alpha^2 \tilde\alpha + 3 \alpha\tilde{\alpha} - \tilde{\alpha}^3) = \frac{b^6-36a^6}{3}
	\label{dprime}
\end{equation}

So that the explicit solution for $y$ is

\begin{equation}
	z = |\alpha - \tilde \alpha| C_{1/3} \left( 12 \frac{ct + D}{|\alpha-\tilde\alpha|^3} \right) 
	\label{explicitz}
\end{equation}
\begin{equation}
	y = z - \frac{\alpha + \tilde{\alpha}}2
	\label{explicitY}
\end{equation}

where we defined the function $C_{1/3} = \ch(1/3 \; \ch^{-1}(x))$; that \ref{explicitY} solves \ref{depressed} can be readily verified by means of the trigonometric identity $\ch(3x) = 4 \ch^3(x) - 3 \ch(x)$.\\

Fixing $c=1/3$ for future convenience and introducing the notation $\delta = \alpha - \tilde{\alpha}$, $\sigma = \alpha + \tilde{\alpha}$, the K\"ahler form is explicitly given by

\begin{equation}
	J(\sigma,\delta) = \left(  z + \frac{\delta}{2}\right) j^1 + \left(z-\frac{\delta}{2}\right) j^2 + ie^k z' e^3 \wedge \bar e^{\bar 3}
	\label{}
\end{equation}

\begin{equation}
	z(t;\sigma,\delta) = \delta \;C_{1/3} \left( \delta^{-3} \left( 4t + \frac{\sigma(3\delta^2 - \sigma^2)}{2} \right) \right)
	\label{}
\end{equation}

~\\\cmmnt{e qua si copincollerà il resto delle note quanto saranno definitive}\\


