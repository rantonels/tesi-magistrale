
\cmmnt{definizione di HEFT come in \cite{MZ}}

In this chapter, we present the technique and results introduce in \cite{MZ} to find the effective theory for strongly-interacting CFTs with holographic duals. Instead of repeating the arguments presented therein, we provide an intuitive summary of the concepts involved.\\

Since we are considering strongly-interacting quantum field theories with minimal supersymmetry, the problem of identifying the low-energy effective field theory directly is generally untractable. However, as we have seen, the strong-coupling regime for the CFT corresponds to effectiveness of the supergravity approximation on the holographically dual string side. Therefore, the low-energy dynamics of the dual system can in principle be read and the resulting theory will coincide with the effective theory for the original QFT. Having been obtained by passing through the holographic dual, these will be termed holographic effective field theories (HEFTs).\\

In practice, it's found that for any given point in the longitudinal coordinates $x^{0,\ldots,3}$ the transverse supergravity configuration will belong to a manifold of different supergravity vacua, and that this manifold is finite-dimensional, in the sense that there is only a finite number of moduli parametrizing deformations of the vacuum configurations. This moduli space coincides of course with the field theory moduli space.\\

A first class of moduli are given by deformations of the dual geometry. These include K\"ahler moduli of the \emph{background} cone in which the branes are placed in, and then the position of the D3-branes themselves on that background - which manifests as a deformation in the resulting warped geometry. Another class instead will be given by the moduli corresponding to the deformations of the $B_2$ and R-R fields of IIB supergravity; while these would be full fields defined on the six-dimensional background, gauge invariance will result in only a finite number of topological invariants of the field configuration to be physical.\\

In short, there will be a finite number of flat directions parametrizing moduli space, and each of these moduli will result in a corresponding scalar field when we extend these deformations to depend on the longitudinal point. Reintroducing $\mathcal{N}=1$ supersymmetry, these will be the lowest components of chiral supermultiplets which will exhaust the degrees of freedom of the low-energy effective field theory.\\

Then, expanding the supergravity action in these modes the action governing these chiral fields can be found. This is nothing else than the explicit form of the effective theory for our original strongly-interacting theory.


\section{Topology}

\cmmnt{identificazione dei moduli del bulk: numeri di Betti e condizioni globali sui potenziali, moduli della struttura K\"ahler}

We now identify the moduli of the bulk string theory. This will include deformations of the background metric and of the RR potentials, parametrized by particular types of $k$-forms. Therefore, a great deal of information about them can be obtained just from examining the topology of the cone $X_6$.\\

First of all, we take as an assumption that the third Betti number of the cone vanishes:

\begin{equation}
	b_3(X) = 0 \label{bettiX3}
\end{equation}

It can be proven from Myers' theorem that $Y_5$ being Sasaki-Einstein means the following Betti numbers vanish:

\begin{equation}
	b_1(Y) = b_4(Y) = 0 \label{bettiY14}
\end{equation}

It's also possible to prove that

\begin{equation}
	b_1(X) = b_5(X) = b_6(X) = 0
\end{equation}

The vanishing of the odd Betti numbers for $X_6$ means $Dp$-branes with $p = 1,3,5$ cannot be wrapped around nontrivial $p$-cycles.\\

We recall the long sequence involving relative homology groups:

\begin{equation}
	\ldots \rightarrow H^{i-1}(Y) \rightarrow H^{i}(X,Y) \rightarrow H^i(X) \rightarrow H^i(Y) \rightarrow H^{i+1}(X,Y) \rightarrow \ldots
\end{equation}

where $H^i(X,Y;\mathbb{R})$ is the relative homology group - closed $k$-forms on $X$ vanishing on $Y$ modulo exact forms with the same property - and when the $;\mathbb{R}$ is omitted we implicitly mean the base field is $\mathbb{R}$. We cut the sequence short by setting $i=2$ and noting $H^1(Y) = 0$ as of \ref{bettiY14} and $H^3(X,Y) \subset H^3(X) = 0$ as of \ref{bettiX3}; the short exact sequence is

\begin{equation}
	0 \rightarrow H^2(X,Y) \rightarrow H^2(X) \rightarrow H^2(Y) \rightarrow 0
\end{equation}

Implying $H^2(X) = H^2(Y) \oplus H^2(X,Y)$. Applying Poincaré duality on the two components and counting dimensions gives

\begin{equation}
	b_2(X) = b_3(Y) + b_4(X) \label{bettidentity}
\end{equation}
\section{K\"ahler moduli}

We now consider the moduli describing the deformation of the K\"ahler (so, geometric) structure of the background. Since $b_3(X) = 0$ by hypothesis, the complex structure is rigid. There are instead moduli for the K\"ahler form $J$; in particular we know from \cmmnt{citazione teoremi esistenza} that every cohomology class $[J]$ of $H^2(X)$ contains a single representative Ricci-flat K\"ahler form $J$, so that $H^2(X)$ is the moduli space for the K\"ahler structure. We can expand the cohomology class as

\begin{equation}
	[J] = v^a [\omega_a] \label{integraldecomposition}
\end{equation}

with $[\omega_a]$ being a basis for the integral cohomology $H^2(X;\mathbb Z)$, as the latter modulo torsion is a lattice sitting in $H^2(X;\mathbb R)$. This means

\begin{equation}
	\delta [J] = \delta v^a [\omega_a]
\end{equation}

meaning there exist representatives in the classes such that the equation without square brackets holds. Since small variations of the K\"ahler form must be $(1,1)$ harmonic forms \cmmnt{reference}, we then know there exist $(1,1)$ harmonic representatives $\omega_a$ for the aforementioned basis of classes. Returning to \ref{integraldecomposition} we can rewrite it as

\begin{equation}
	J - v^a \omega_a \in [0]
\end{equation}

But for the LHS to belong to the zero class just means to be exact. Therefore

\begin{equation}
	J = J_0 + v^a \omega_a \label{JandJ0}
\end{equation}

with $J_0$ being exact and $(1,1)$. Note the linearity of this parametrization is an illusion of notation: the condition $\Delta \omega_a = 0$ depends on the metric and so on both $J_0$ and $v^a$.\\

It is then useful to decompose this set of $b_2(X)$ harmonic forms according to identity \ref{bettidentity} into $b_3(Y)$ noncompact elements $\tilde\omega_\beta$ and $b_4(X)$ normalizable forms $\hat\omega_\alpha$. By ``normalizable'' it's meant the hatted forms have finite norm according to the product

\begin{equation}
	\int_X \omega_a \wedge \star \omega_b =: \mathcal{M}_{ab}
\end{equation}

while the other $b_3(Y)$ don't. They are however all normalizable according to the ``warped'' product

\begin{equation}
	\int_X e^{-4A} \omega_a \wedge \star \omega_b =: \mathcal{G}_{ab}
	\label{}
\end{equation}

where the factor $e^{-4A}$, as will be explained later, is the warp factor resulting from the backreaction of the D-branes. More intuitively, in our particular case of $X$ being (asymptotically) a cone, this means that $||\hat\omega_\alpha||^2$ must drop at least as fast as $r^{-8}$, while $||\tilde\omega_\beta||^2$ will go as $r^{-4}$.

\section{Remaining moduli}

We are now in position to classify flat deformations of the axio-dilaton $\tau$ and the 2-forms $C_2$ and $B_2$, which we compose into a single complex 2-form $C_2 - \tau B_2$, plus $\tau$ itself. The former field's flat deformation will be generated by cohomology classes of $H_2(X)$, so in practice the harmonic forms $\omega_a$ found above can be used as a basis. Therefore the following decomposition is possible:

\begin{equation}
	C_2 - \tau B_2 = l_s^2 \left( \beta^\alpha \hat\omega_\alpha + \lambda^\beta \tilde \omega_\beta \right)
	\label{}
\end{equation}

The $b_4(X)$ moduli $\beta^a$ weighing the compact forms $\hat\omega_\alpha$ will result in dynamical chiral fields in the HEFT. Instead, the $b_3(Y)$ $\lambda^\beta$ moduli, to which we add one complex modulus for $\tau$, parametrize deformations which will turn out to be non-dynamical. The reason is precisely that the kinetic matrix for these fields will turn out to be $\mathcal{M}_{ab}$, which is only finite for the normalizable forms.\\

Finally, an obvious set of moduli $z_I^i$ ($I=1,\cdots,N$, $i = 1,2,3$) are to be introduced to parametrize the motion of the D3-branes on the background. As was hinted before, these alone are coordinates for the submanifold $\mathcal{M}_\text{mes}$ in $\mathcal{M}$, which to be precise should be quotiented by permutation of the branes. Therefore, $\mathcal{M}_\text{mes}$ based at any given point of moduli space is the symmetric product of $N$ copies of the background geometry, as it is at that particular point.

\cmmnt{allright}


\section{Chiral fields}

Finally, we have to introduce the chiral fields corresponding to the dynamical moduli. We use the moduli $\beta^\alpha$ and $z_I^i$ directly as the lowest component of the corresponding superfield, while to $v^a = (\hat v^\alpha,\tilde v^\beta)$ it is useful to associate fields $\rho^a = (\hat \rho_\alpha, \tilde \rho_\beta)$, obtained by a sort of Legendre transform:

\begin{equation}
	\Re{\rho_a} = \frac{1}{2}\sum_I \kappa_a(z_I, \bar z_I ; v) - \frac{1}{2 \Im \tau} I_{a\alpha\beta} \Im \beta^\alpha \Im \beta^\beta - \frac{1}{\Im \tau} I_{a\alpha\sigma} \Im \beta^\alpha \Im \lambda^\sigma
	\label{}
\end{equation}

This 

\section{Effective action}

\cmmnt{azione efficace calcolata in \cite{MZ}; forse anche la derivazione?}

\section{Example: the Klebanov-Witten HEFT}

As an example, we now summarize a direct application of this method to the conifold theory described in \ref{sec:KW}.\\

It's essential for the metric for the background to be presented in complex coordinates for the construction above to be applicable. We exploit the fact that at any generic point of moduli space \emph{except} the origin, where the background is the singular conifold, the geometry is that of a smooth complex 3-fold describable as the total space of the sum of the tautological bundle on a $\mathbb{CP}^1$ with itself: $\mathcal{O}_{\mathbb{CP}^1}(-1) \oplus \mathcal{O}_{\mathbb{CP}^1}(-1)$. The base $\mathbb{CP}^1 \sim \mathbb{S}^2$ of the bundle is the ``resolution'' of the conical singularity. We use this to build a chart on the resolved conifold by extending a stereographic chart on the base to get the complex coordinates $z^i = (\lambda, U, Y)$, the first stereographic for the base and the latter two fibral.\\

With this presentation, it's clear that the only 2-cycle of $X_6$ is the base $\mathbb{S}^2$, so that $b_2(X) = 1$. Moreover, we already know that as a real cone the conifold has base $SU(2)\times SU(2) / U(1) \approx \mathbb{S^2} \times \mathbb{S^3}$, and that the resolution cannot really change the topology of the base at infinity, so that the only 3-cycle in $Y_5$ is the $\mathbb{S}^3$ and $b_3(Y) = 1$. So, according to \ref{bettidentity}, $b_4(X_6) = 0$. This will mean, according to the previous identification, that we will have a single $\tilde v =: v$ modulus parametrizing a non-normalizable K\"ahler deformation and dual to a single chiral field $\tilde\rho =: \rho$, no $\beta$ fields, and two non-dynamical parametres $\lambda$ and $\tau$. \\

The K\"ahler modulus $v$ is identified with the volume of the base. Therefore, all Calabi-Yau deformations of the conifold are a one-parameter family and the singular conifold itself lies at the origin, $v=0$. The K\"ahler potential for $X_6$ in the stereographic coordinates is detailed for example in \cite{PandoZayas}, and takes the form:

\begin{equation}
	k(z,\bar z; v) = \frac{1}{2} \int_0^{s^2} d\ln x \; \gamma(x;v) \; + \frac{v}{2\pi} \ln(1+|\chi|^2)
	\label{}
\end{equation}

where $s^2 = (1+|\chi|^2)(|U|^2 + |Y|^2)$ and $\gamma$ must satisfy the following for the metric to be Ricci-flat:

\begin{equation}
	\gamma^3 + \frac{3v}{2\pi} \gamma^2 - x^2 = 0
	\label{}
\end{equation}

The potential $\kappa$ generating the unique harmonic form $\tilde\omega$ is given simply by derivative of $k$ with respect to the modulus, plus a $v$-dependent piece fixed by the asymptotic condition \ref{asycond}:

\begin{equation}
	\kappa = -\frac{1}{4} \int_0^{s^2} d\ln x \; \frac{\gamma}{\pi\gamma+v} \; + \frac{1}{2\pi} \ln(1+ |\chi|^2) - \frac{3}{8\pi}\ln v
	\label{}
\end{equation}

This in turn defines the relationship between $\rho$ and $v$ as

\begin{equation}
	\Re \rho = - \frac{1}{8} \sum_I \int_0^{s_I^2} d\ln x \; \frac{\gamma}{\pi\gamma + v} + \frac{1}{4\pi}\sum_I \ln(1+|\chi_I|^2) - \frac{3N}{16\pi} \ln v
\end{equation}

from which one can readily find the $1\times1$ $\mathcal{G}$ matrix:

\begin{equation}
	\mathcal{G} = - \frac{\partial \rho}{\partial v} = \frac{3}{16\pi}\sum_I (v+\pi\gamma)^{-1}
	\label{}
\end{equation}

All ingredients for writing down the HEFT are now available. The chiral part of the bosonic lagrangian for the effective low-energy theory of the KW model is given by

\begin{equation}
	\mathcal{L} = - \pi \mathcal{G}^{-1} \nabla \rho \wedge \star \nabla \bar \rho - 2\pi \sum_I J_{i\bar j} dz_I^i \wedge \star d\bar z_I^{\bar j} 
	\label{}
\end{equation}

where of course $J_{i\bar j} = \partial_i \bar \partial_j k(z,\bar z; v)$ is the metric tensor of the resolved conifold. The expressions here are deceivingly simple: both $\mathcal{G}$ and $J$ depend on the modulus $v$ which must be understood as a function of the chiral field $\rho$ by inversion of the Legendre transform. The connection for the covariant derivative, being a simple derivative of the K\"ahler potential, can also be explicited:

\begin{equation}
	\mathcal{A}_i dz^i = (4v + 4\pi \gamma)^{-1} \left( \frac{2v + \pi\gamma}{\pi(1+|\chi|^2)}\bar\chi d\chi - \frac{\gamma\left( \bar U dU + \bar Y dY \right)}{|U|^2 + |Y|^2} \right)
	\label{}
\end{equation}
