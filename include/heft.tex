
\cmmnt{definizione di HEFT come in \cite{MZ}}

\section{Bulk moduli}

\cmmnt{identificazione dei moduli del bulk: numeri di Betti e condizioni globali sui potenziali, moduli della struttura K\"ahler}

We now identify the moduli of the bulk string theory. This will include deformations of the background metric and of the RR potentials, parametrized by particular types of $k$-forms. Therefore, a great deal of information about them can be obtained just from examining the topology of the cone $X_6$.\\

First of all, we take as an assumption that the third Betti number of the cone vanishes:

\begin{equation}
	b_3(X) = 0 \label{bettiX3}
\end{equation}

It can be proven from Myers' theorem that $Y_5$ being Sasaki-Einstein means the following Betti numbers vanish:

\begin{equation}
	b_1(Y) = b_4(Y) = 0 \label{bettiY14}
\end{equation}

It's also possible to prove that

\begin{equation}
	b_1(X) = b_5(X) = b_6(X) = 0
\end{equation}

The vanishing of the odd Betti numbers for $X_6$ means $Dp$-branes with $p = 1,3,5$ cannot be wrapped around nontrivial $p$-cycles.\\

We recall the long sequence involving relative homology groups:

\begin{equation}
	\ldots \rightarrow H^{i-1}(Y) \rightarrow H^{i}(X,Y) \rightarrow H^i(X) \rightarrow H^i(Y) \rightarrow H^{i+1}(X,Y) \rightarrow \ldots
\end{equation}

where $H^i(X,Y;\mathbb{R})$ is the relative homology group - closed $k$-forms on $X$ vanishing on $Y$ modulo exact forms with the same property - and when the $;\mathbb{R}$ is omitted we implicitly mean the base field is $\mathbb{R}$. We cut the sequence short by setting $i=2$ and noting $H^1(Y) = 0$ as of \ref{bettiY14} and $H^3(X,Y) \subset H^3(X) = 0$ as of \ref{bettiX3}; the short exact sequence is

\begin{equation}
	0 \rightarrow H^2(X,Y) \rightarrow H^2(X) \rightarrow H^2(Y) \rightarrow 0
\end{equation}

Implying $H^2(X) = H^2(Y) \oplus H^2(X,Y)$. Applying Poincaré duality on the two components and counting dimensions gives

\begin{equation}
	b_2(X) = b_3(Y) + b_4(X)
\end{equation}

This result will be useful in parametrizing deformations of the axio-dilaton $\tau$ and the 2-forms $C_2$ and $B_2$. In particular, combining these fields into a single complex $2$-form $C_2 - \tau B_2$, for any given value of $\tau$ the deformations of this form are decomposable in $b_2(X)$ complex parameters, to which we then append one additional parameter for deformations of $\tau$. Making use of the above splitting, these are divided in $b_3(Y) + 1$ 'boundary' complex parameters counting non-dynamical deformations, and $b_4(X)$ 'bulk' dynamical complex moduli.\\

\cmmnt{$C_4$ moduli}

Having dealt with RR moduli, we now consider the moduli of the K\"ahler structure of the background. Since $b_3(X) = 0$ by hypothesis, the complex structure is rigid. There are instead moduli for the K\"ahler form $J$; in particular we know from \cmmnt{citazione teoremi esistenza} that every cohomology class $[J]$ of $H^2(X)$ contains a single representative Ricci-flat K\"ahler form $J$, so that $H^2(X)$ is the moduli space for the K\"ahler structure. We can expand the cohomology class as

\begin{equation}
	[J] = v^a [\omega_a] \label{integraldecomposition}
\end{equation}

with $[\omega_a]$ being a basis for the integral cohomology $H^2(X;\mathbb Z)$, as the latter modulo torsion is a lattice sitting in $H^2(X;\mathbb R)$. This means

\begin{equation}
	\delta [J] = \delta v^a [\omega_a]
\end{equation}

meaning there exist representatives in the classes such that the equation without square brackets holds. Since small variations of the K\"ahler form must be $(1,1)$ harmonic forms \cmmnt{reference}, we then know there exist $(1,1)$ harmonic representatives $\omega_a$ for the aforementioned basis of classes. Returning to \ref{integraldecomposition} we can rewrite it as

\begin{equation}
	J - v^a \omega_a \in [0]
\end{equation}

But for the LHS to belong to the zero class just means to be exact. Therefore

\begin{equation}
	J = J_0 + v^a \omega_a \label{JandJ0}
\end{equation}

with $J_0$ being exact and $(1,1)$. Note the linearity of this parametrization is an illusion of notation: the condition $\Delta \omega_a = 0$ depends on the metric and so on both $J_0$ and $v^a$.


\section{Boundary moduli}

\cmmnt{descrizione della CFT holografica tipica (quiver), di nuovo classificazione dei moduli}

\section{Effective action}

\cmmnt{azione efficace calcolata in \cite{MZ}; forse anche la derivazione?}

\section{The Klebanov Witten HEFT}

\cmmnt{Calcolo esplicito della HEFT per il KW}
