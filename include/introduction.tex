Strongly-coupled quantum field theories represent canonical examples of physical systems whose study is extremely challenging. Even the question of the mere existence of any interacting QFT in four dimensions from a formal standpoint has not been settled. In addition to this, strong couplings are not amenable to the tools of perturbation theory. The interest in this class of theories stems actually from practical considerations - many of them represent realistic models for physical phenomena, e.g. the theory of strong interaction.

A subset of questions concerns whether a given strongly-interacting theory is described at low energy by an effective local field theory, and if so, what are its degrees of freedom and their precise dynamics. Often, part of the structure of the effective theory is constrained by symmetries, but no general method exists to fix it completely. Recently\cite{MZ}, a novel approach for determining the effective Lagrangian was introduced that makes use of tools from an apparently unrelated area of physics: string theory.

It is remarkable that string theory was originally conceived as a description of hadronic physics, so a low-energy effective theory for what ultimately turned out to be a gauge theory, QCD. When string theory was found to have unsuitable qualities for this application, it was replaced by the theory of quantum chromodynamics - however it also proved to be effective for solving a seemingly unrelated problem of fundamental physics: quantizing gravity. Since then, string theory blossomed into a vast and rich field reaching into numerous areas of mathematics and physics, and of course a candidate for a ``Theory of Everything'' describing the entirety of fundamental physics.

Among the most unexpected discoveries in strings, made decades after their conception, is a series of unusual exact equivalences between string theories set in particular ten-dimensional backgrounds and four-dimensional gauge QFTs. More generally, one finds families of exact equivalences between local quantum field theories and higher-dimensional theories containing gravity, which are termed ``holographic''. This explains the original partial success of strings in modeling strong interactions, assuming that some or perhaps most gauge theories have or can be approximated as having a holographic description as a ten dimensional theory involving strings. This roundtrip has therefore brought strings back to strongly-coupled gauge theories. Various aspects, qualitative and most importantly quantitative, of QFTs can be studied directly by means of their holographic string dual, a gravitational theory, if it exists. 

The first and most important case of such a duality\cite{Maldacena} equates IIB superstring theory set on the background

\begin{equation}
	\operatorname{AdS}_5 \times \mathbb{S}^5
\end{equation}

(the ``bulk'') with maximally supersymmetric Yang-Mills theory on $\mathbb{R}^{1,3}$ (the ``boun\-da\-ry''), which is a conformal field theory. The denomination of ``AdS/CFT'' (anti-de Sitter / conformal field theory) correspondence for holographic dualities stems from this (even though cases either with no AdS geometry or not conformal are known). One direction in which to generalize this construction is to replace $\mathbb{S}^5$ with other compact 5-manifolds $Y_5$. This yields dualities involving more complex and interesting field theories, less constrained by symmetries; therefore this thesis will be focused on this class of correspondences.

Actually, since string theory in general is very challenging to study, AdS/CFT only becomes truly useful in terms of describing the dynamics of the CFT if string theory can be approximated by a weakly coupled effective field theory of its own, supergravity. This limit corresponds to the CFT being strongly-coupled and having a large number of colours. Therefore the regime accessible through holography is precisely the strongly-coupled region where the gauge theory would be normally impossible to investigate.

Recently, a novel approach was introduced\cite{MZ} for determining the effective theory to such duals of $\operatorname{AdS}_5 \times Y_5$. A procedure for identifying the degrees of freedom and the Lagrangian for the effective low-energy theory for a class of gauge theories with $\operatorname{AdS_5} \times Y_5$ holographic duals is provided, by expanding the supergravity action on the dual bulk geometry. Interestingly, these are somewhat special in that they include models of minimal supersymmetry ($\mathcal{N}=1$), which makes for more realistic but by converse less constrained theories than typical holographic field theories, with higher supersymmetry. The ability to pinpoint the exact effective Lagrangian is then particularly noteworthy.

The original contribution in this work is the specialization of this construction to a specific field theory, the $Y^{2,0}$ theory, a strongly-coupled superconformal quiver theory for which we will therefore fix the exact effective Lagrangian, entirely through the geometry of the relative string background. This will require necessarily the determination of the general Calabi-Yau deformation of the background in complex coordinates, which was not known previously.

This thesis will be structured as follows. We will first provide a general introduction to IIB superstring theory, D-brane stacks on cones and the resulting gauge field theories, and holography. Then, we will summarize the relevant results and techniques from \cite{MZ}. Finally, we will present a complete parametrization of the geometry of the $Y^{2,0}$ theory and will apply those results and techniques to identify the exact effective Lagrangian of the field theory.

