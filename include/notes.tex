%        File: notes.tex
%     Created: mar ago 23 10:00  2016 C
% Last Change: mar ago 23 10:00  2016 C
%
%\documentclass[a4paper]{article}
%\usepackage{amssymb}
%\usepackage[]{amsmath}
%\usepackage[]{hyperref}
%
%\usepackage[]{graphicx}
%
\graphicspath{{images/}}
%\usepackage{color}
%
%\DeclareMathOperator{\ch}{ch}

\label{chap:y20}

%\setlength{\parindent}{0cm}
%i
%\begin{document}

Having secured the tools required, we are now ready to take on the holographic effective theory of the $Y^{2,0}$ theory introduced in section \ref{sec:squares}, to which this thesis is dedicated, and whose low-energy effective dynamics has not been investigated so far.

Exploiting the architecture established in the previous chapter, we will identify the fields and parameters on the gravitational dual which enter in the low-energy description, and determine the exact form of the effective action for this strongly-coupled, minimally supersymmetric gauge theory. 

The bulk of this calculation turns out to be occupied by the determination of the Ricci-flat metric of the general Calabi-Yau deformation of the $X^{2,0}$ cone over the Sasaki-Einstein base $Y^{2,0}$; the deformations are in this case parametrized by two moduli, measuring the volumes of a pair of 2-cycles. Since the 4-cycle product of the 2-cycle is non-zero, this model features the possibility for the blowup of a 4-cycle. Due to the introduction of this 4-cycle blowup (ultimately arising from the $\mathbb{Z}_2$ orbifold) the metric is quite more complicated than the deformation of the conifold, which only featured a single 2-cycle. As part of our original contributions we will thus present therefore our determination of the deformed metric in complex coordinates for generic values of the two moduli.

\section{General properties}

The features of the SCFT analyzed in section \ref{sec:squares} can be rederived holographically. First of all, the homology of the singular cone will allow for the counting of supergravity moduli and marginal deformations that can be seen to match with those found from the field theory side. We recall (see \eqref{y20conemetric}) the metric on the $X^{2,0}$ cone is

\begin{equation}
	ds_6^2 = dr^2 + r^2 ds_5^2
	\label{}
\end{equation}
\begin{equation}
	ds_5^2 = \frac{1}{9}\left( d\psi + \sum_{i=1,2} \cos\theta_i d\phi_i\right)^2 + 
	\frac{1}{6} \sum_{i=1,2} \left(d\theta_i^2 + \cos^2\theta_i d\phi_i^2 \right)
\end{equation}
\begin{equation}
	\psi \in [0,2\pi]
	\label{}
\end{equation}

Since the conifold base had topology $\mathbb{S}^2 \times \mathbb{S}^3$\cite{Candelas}, this base will be $Y^{2,0} \cong (\mathbb{S}^2 \times \mathbb{S}^3)/\mathbb{Z}_2$. The new Betti numbers of the cone are easily read from the fact that the generic resolution of $X^{2,0}$ is a bundle $\C \rightarrow \cpone \times \cpone$, which we will prove in the next section. First, we have $b_3(Y^{2,0}) = b_3(\ess^2 \times \ess^3) = 1$, while $b_4(X^{2,0}) = b_4(\cpone \times \cpone) = 1$. Finally, using \eqref{bettidentity}, $b_2(X_6) = 2$. To summarize:

\begin{equation}
	b_2(X_6) = 2,\quad b_4(X_6) = 1,\quad b_3(Y_5) = 1\,.
	\label{}
\end{equation}


According to the discussion of section \ref{sec:heftkm}, there will be $b_2(X) = 2$ independent harmonic forms $\omega_a$ from the two 2-cohomology classes, and these will parametrize deformations of the K\"ahler form:

\begin{equation}
	J = J_0 + v^a \omega_a\,,
	\label{}
\end{equation}

with two associated moduli $v^a$. These forms will be divided in $b_3(Y) = 1$ ``non-compact'' form $\tildd\omega = \omega_2$, and $b_4(X) = 1$ ``compact'' form $\hatt \omega= \omega_1$. The latter, associated with the blow-up of the 4-cycle $\cpone\times\cpone$, is the novel feature with respect to the Klebanov-Witten model, which only featured $\tildd\omega$.

In addition, $\omega_a$ will generate $A_2 - \tau B_2$ deformations:

\begin{equation}
	(A_2 - \tau B_2) = l_s^2 \left( \beta \, \hatt \omega + \lambda \, \tildd \omega \right) \,.
	\label{}
\end{equation}

However, only the normalizable 2-form $\hatt \omega$ will yield a dynamical field $\beta$, as shown in section \ref{sec:hefttwo}. The parameter $\lambda$ will be a marginal parameter, alongside the axio-dilaton $\tau$.

Once the $v^a = (\hat v, \tilde v)$ K\"ahler moduli have been transformed into the $\rho^a = (\hat \rho, \tilde \rho)$ fields according to \eqref{rhov}, it is now possible to match the counting of moduli and marginal parameters between the bulk and the $Y^{2,0}$ CFT as following:

\begin{center}
\begin{tabular}{ccc}
	& $\ads 5 \times Y^{2,0}$ & CFT \\ \midrule \midrule
			& $ z_I^i$ & $3N$ VEVs of mesons\\ \cmidrule{2-3} 
dynamic 		& $\hatt \rho$ & \multirow{3}{*}{$\chi-1 = 3$ VEVs of baryons} \\
moduli			& $\tildd \rho$ & \\
			& $ \beta$ & \\ \midrule
marginal	& $\tau$ 	&  \multirow{2}{*}{$2$ marginal deformations}	\\
parameters			& $\lambda$ &	
\end{tabular}
\end{center}

Since the dynamical fields of the effective theory have been identified and matched, the explicit construction of the $\omega_a$ forms will allow for the specification of their dynamics. In practice, this requires determining the metric in complex coordinates of the general Calabi-Yau deformation of the $X^{2,0}$ cone. This metric is known \cite{PandoZayasy20} in real, ``polar'' coordinates (see \eqref{genCYy20cones}) but we need its explicit form in a chart compatible with the K\"ahler structure. In the next section, we present our solution to this problem in the form of an explicit expression for the Calabi-Yau metric in a set of complex coordinates, and verify the match with the known parametrization \eqref{genCYy20cones}.


\section{K\"ahler form}

The metric of the general Calabi-Yau deformation of the $X^{2,0}$ cone is already well-known in real coordinates as (repeating \eqref{genCYy20cones})

\begin{align}
\begin{split}
ds^2 \,= \;\, &\kappa^{-1}(r)\,dr^2 \;+\; \frac{1}{9} \kappa(r) r^2 \large(d\psi + \cos\theta_L d\phi_L + \cos\theta_R d\phi_R\large)^2\\
& + \frac{1}{6} r^2 d\Omega_L^2 \,+ \, \frac{1}{6}(r^2+a^2) d\Omega_R^2\,, \label{y20metric} 
\end{split}
\end{align}

\begin{equation}
	\kappa(r) = \cfrac{1 + \cfrac{9a^2}{r^2} - \cfrac{b^6}{r^6}}{1+ \cfrac{6a^2}{r^2}}\,,
\end{equation}

with $a,b$ the two unique real moduli. The topology is that of an $\mathbb{R}^2$ bundle over $\mathbb{S}^2 \times \mathbb{S}^2$. We take it as an assumption that this matches with the complex structure associated to the K\"ahler form so that this is the total space of a $\mathbb{C}$ bundle over $\mathbb{CP}^1 \times \mathbb{CP}^1$ - we will confirm this a posteriori when we will provide the complex-coordinates expression and show it agrees with the real form.

With this assumption, we search for the general CY metric on a $\mathbb{C} \rightarrow \mathbb{CP}_L^1 \times \mathbb{CP}_R^1$ bundle; on the spheres of the base we take the round metric, given by the K\"ahler forms $j^L$ and $j^R$ generated by K\"ahler potentials\footnote{Attention should be paid to the fact that the potentials are different for different coordinates.} $k^L$, $k^R$. We thus, having chosen any local complex charts $y^L$, $y^R$ on the two basal spheres (e.g.: stereographic) and a fibral coordinate $\zeta$ on $\mathbb{C}$, have a local chart $z^i = (y^L,y^R,\zeta)$ for the bundle. Asymptotically (that is, $|\zeta| \rightarrow \infty$) $y^{L,R}$ and $\operatorname{arg} \zeta$ parametrize $Y_5$, while $|\zeta|$ maps to the radial coordinate; the precise relationship will be clarified shortly.

It is easy to verify explicitly that, given any set of complex coordinates on the base $(y^L,y^R)$,

\begin{equation}\label{spheresareEinstein}
	j^L \wedge j^R = e^{-\lamkah}\, \left(dy^L \wedge dy^R \wedge d\bar y^L \wedge d\bar y^R\right)\,,
\end{equation}

for some $\Lambda$ depending on the overall size of the spheres (for the unit sphere, $\Lambda = 1$).

While \eqref{spheresareEinstein} is true in any chart, let us show it explicitly for the standard stereographic coordinates. If $y^L$ is a stereographic coordinate on $\mathbb{CP}^1$, the K\"ahler potential is

\begin{equation}
	k^L = \log(1+|y^L|^2)\,,
	\label{}
\end{equation}

and the K\"ahler form is given by the Fubini-Study metric:

\begin{equation}
	j^L = i \partial \bar\partial \ln(1+ |y^L|^2) = \frac{dy^L \wedge d\bar y^L}{\left(1+|y^L|^2\right)^2}\,.
	\label{}
\end{equation}

Then clearly

\begin{equation}
	j^L = e^{-2 k_L } \,dy^L \wedge d\bar y^L\,.
	\label{}
\end{equation}

We also introduce the radial coordinate $t(\zeta,y^L,y^R)$ as 

\begin{equation}\label{tdef}
	t := |\zeta|^2 e^{\Lambda \left(k^L + k^R\right) }\,.
\end{equation}

We then propose the following ansatz for the K\"ahler potential of the resolved cone:

\begin{equation}
	k_0 = f(t;\alpha,\tildd\alpha) + \alpha k^L + \tildd\alpha k^R \label{y20potentialansatz}
\end{equation}

where the moduli $\alpha,\tilde\alpha$, controlling the volume at $t=0$ of the base 2-spheres, should parametrize the Ricci-flat K\"ahler resolutions of the cone. We now prove that there is always an $f(t;\alpha,\tilde\alpha)$ that makes the metric Ricci-flat.

The K\"ahler form generated by \eqref{y20potentialansatz} is straightforward:

\begin{align}
	\begin{split}\label{y20kahlerformansatz}
	J &= \, \AL j^L + \, \AR j^R \\&+ i e^{\lamkah} (f' + t f'') \, (d\zeta + \Lambda \zeta \partial k) \wedge (\mathrm{c.c.})
\end{split}
\end{align}

\newcommand{\fibral}{e^3 \wedge \bar e^{\bar 3}}


%This is more simply expressed as

%\begin{equation}
%J = J_M + M \fibral\,,
%\end{equation}

%where $J_M$ is the purely basal part, $e^3 = d\zeta + \Lambda \zeta \partial k$ and $M:= ie^{\Lambda k} \left( f' + t f'' \right)$ is a scalar factor. The volume form is then
The form $J = J_{\ibj}\, dz^i \wedge d\bar z^{\bar\jmath}$ defines a hermitian metric $ds^2 = J_{\ibj}\, dz^i d\bar z^{\bar\jmath}$, or equivalently $g_{\ibj} = J_{\ibj}\,$. We would like to identify the condition on $f(t)$ for such a metric to be Ricci-flat. In a K\"ahler manifold, and in a given local chart, the Ricci form\footnote{the Ricci form on a K\"ahler manifold is related to the Ricci tensor as $\rho = i R_{\ibj} \, dz^i \wedge d\bar z^{\bar\jmath}$, thus $R_{\ibj} = 0$ iff $\rho = 0$. } is given by \cite{Ballmann}

\begin{equation}
	\rho = i \partial \bar\partial \log \sqrt{ \det g_{\ibj}}\,,
	\label{}
\end{equation}

so that Ricci-flatness, $\rho = 0$, means $\det g = \det {J_{\ibj}}$ is a constant. On the other hand, note the volume form induced by the metric would be

\begin{equation}
	d\vol_{X_6} = \left(\frac{i}{2}\right)^3 \left(\det g_{\ibj}\right) \, d^{\,6} z
	\label{volumeintermsofmetric}
\end{equation}

with $d^{\,6} z = d\zeta \twedge dy^L \twedge dy^R \twedge d\bar \zeta \twedge d\bar y^L \twedge d\bar y^R$. However, for a K\"ahler $3$-fold the volume form admits the equivalent expression in terms of the K\"ahler form:

\begin{equation}
	3! \, d\vol_{X_6} = J\wedge J \wedge J 	\label{}
\end{equation}

Having defined $e^3 := d\zeta + \Lambda \zeta \partial (\kah)$, this is

\begin{equation}
	= 3 \AL \AR i e^{\lamkah} \left(f'+tf''\right) \left(j^L \wedge j^R \wedge \fibral \right)\,
	\label{}
\end{equation}

and using \eqref{spheresareEinstein}:

\begin{equation}
	= (\AL)(\AR) \left( f' +tf'' \right) d^{\,6} z
	\label{volumeintermsofkahler}
\end{equation}

By comparing \eqref{volumeintermsofkahler} with expression \eqref{volumeintermsofmetric} for the volume form, we have to deduce that Ricci-flatness is equivalent to the following expression being constant:

%as all other terms in the cube vanish. Since the volume form is also $\sqrt{\det g} \, d\Omega \wedge \overline \Omega$, with $\Omega = d\zeta \wedge dy^L \wedge dy^R$ the holomorphic three-form, and the Ricci tensor for a K\"ahler space is proportional to $\partial \bar \partial \ln \det g$, then the condition for Ricci-flatness is equivalent to the prefactor of $\Omega \wedge \bar \Omega$ in $J\wedge J \wedge J$ being constant, that is to say

\begin{equation}
	(\alpha + \Lambda t f')(\tildd{\alpha} + \Lambda t f') \frac{d}{dt} (\Lambda t f') =: c 
\end{equation}

or, having defined $\y(t) := \Lambda t f'(t)$,

\begin{equation}
	(\alpha + \y)\,(\tildd{\alpha} + \y)\, \y' = c\,. \label{rflatcondition}
\end{equation}

This differential equation (second order in $f(t)$) is thus the condition for the metric resulting from the ansatz \eqref{y20potentialansatz} to be Ricci-flat, and therefore a supergravity solution.

Since $f(t)$ and $f'(t)$ must be regular as $t=0$, and $\Lambda t f'(t) = \y$, it must be that $\y \rightarrow 0$ as $t\rightarrow 0$; this condition eliminates the freedom from the constant of integration for equation \eqref{rflatcondition}. The constant $c$ on the other hand can be readily reabsorbed into a $t$ rescaling. Therefore there should be a unique $\y$ (and so a unique $f$ up to unconsequential constant shifts) that gives a Ricci-flat metric. In appendix \ref{appendix:solutioncubic} we prove that the solution is indeed unique, and the solution to \eqref{rflatcondition} is given by


\begin{equation}
	\y(t;\alpha,\tildd\alpha)  = |\alpha - \tildd\alpha| \Cfun { 12 \frac{ct + D}{|\alpha-\tildd\alpha|^3} } - \frac{\alpha + \tildd\alpha}{2}\,,
	\label{ysolutionmain}
\end{equation}
\begin{equation}
	 D := \frac{1}{12}(-\alpha^3 + 3 \alpha^2 \tilde\alpha + 3 \alpha\tilde{\alpha} - \tilde{\alpha}^3) \,.
	\label{}
\end{equation}

While essential for our derivation of the HEFT, the explicit form above of $\y(t;\alpha,\tilde\alpha)$ is not necessary to verify this metric matches with the real-coordinate form \eqref{y20metric}: let us express the K\"ahler form in terms of $\y$ and show it actually coincides with the latter. We have

\begin{align}
	\label{Jintermsofy}
	J & =  (\alpha + \y) j^L + (\tilde{\alpha} + \y) j^R + \frac{ie^{\Lambda (\kah)}}\Lambda \y' \fibral\\
	  & =  (\alpha + \y) j^L + (\tilde{\alpha} + \y) j^R + \frac{ie^{\Lambda (\kah)} \,c}{\Lambda (\alpha + \y)(\tilde{\alpha} + \y)} \fibral
\end{align}

Now, we parametrize the fiber as (compatibly with \eqref{tdef}) 

\begin{equation}
	\zeta = e^{-\Lambda k/2}\, \sqrt{t} \,e^{i\psi}\,,
\end{equation}

and the 2-spheres with spherical coordinates $\theta_i$, $\phi_i$ which fixes $\Lambda = 1$. Then the metric corresponding to $J$ is

\begin{equation}
	ds_6^2 = (\alpha+\y)\, d\Omega^2_L + (\tildd\alpha + \y) \,d\Omega^2_R + \frac{\y'}{t} \left( \frac{dt^2}{4} + t^2 (d\psi + \sigma)^2 \right)\,,\label{tempmetric}
\end{equation}

where 

\begin{equation}
	\sigma := -i\frac{\Lambda}{2}(\partial k - \bar \partial k) = \cos\theta_L d\phi_L + \cos\theta_R d\phi_R
	\label{}
\end{equation}

and $d\Omega_{L,R}^2 = d\theta_{L,R}^2 + \sin^2\theta  d\phi^2_{L,R}$. But the $(t,\psi)$ part of the metric \eqref{tempmetric} is simply

\begin{equation}
	ds^2 = \frac{1}{4y't} dy^2 + (y' t) (d\psi + \sigma)^2
\end{equation}

Exploiting both the Ricci-flatness condition \eqref{rflatcondition} and its integrated form \eqref{rflatintegrated} we rewrite the expression $\y't$ as

\begin{align}
	\y'\,t & = \left(\frac{1}{(\alpha + \y)(\tilde\alpha + \y)}\right) \left( \frac{\y^3}{3} + \frac{\alpha + \tilde{\alpha}}{2} \y^2 + \alpha \tilde{\alpha} \y \right)\\
	& = 3c\,r^2 \left(
	{1+ \frac{3}{2} \frac{\tilde{\alpha} - \alpha}{r^2} + {\alpha^{2}(\alpha - 3 \tilde{\alpha})}{2r^6} }
	\right)\bigg/\left({1+ \frac{\tilde{\alpha} -\alpha}{r^2} }\right)\\
	& = 3c\,r^2 \left({1+\frac{9a^2}{r^2} - \frac{b^6}{r^6}}\right)\bigg/\left({1 + \frac{6a^2}{r^2}}\right) \\
	& = 3c\,r^2 \kappa(r)
\end{align}

provided we make the identifications

\begin{equation}
	a^2 = \frac{1}{6}(\tilde{\alpha} - \alpha)\,,\quad b^6 = \frac{\alpha^{2}(3\tilde{\alpha}-\alpha)}2\,,\quad r^2 = \y + \alpha
	\label{}
\end{equation}

Therefore, the final coordinate change to the (asymptotically) conical $r$ coordinate is then given by $r^2 := y + \alpha$ - note this renders the inherent symmetry between the left and right 2-cycles non-manifest\footnote{Clearly, we could swap $\alpha$ and $\tilde{\alpha}$ in all of the above definitions, with no consequence.}. The resulting metric, after taking $c=1/3$, is precisely the known real-coordinate metric \eqref{y20metric}. Thus, as the latter is the most general Calabi-Yau deformation of the $X^{2,0}$ cone, we have to conclude that the two-parameter family of metrics \eqref{Jintermsofy} in complex coordinates coincides with it.

Using the freedom to scale $t$ to fix $c=1/3$ for future convenience and introducing the notation 

\begin{equation}\label{convenientmoduli}
	\elta := \alpha - \tilde{\alpha}\,,\quad \sigma := \alpha + \tilde{\alpha}\,,
\end{equation}

the Ricci-flat K\"ahler form is explicitly given by

\begin{equation}
	J(\sigma,\elta) = \left(  \z + \frac{\elta}{2}\right) j^L + \left(\z-\frac{\elta}{2}\right) j^R + ie^{\kah} \z' e^3 \wedge \bar e^{\bar 3}\,
	\label{}
\end{equation}

\begin{equation}
	\z(t;\sigma,\elta) := \elta \,\Cfun { \elta^{-3} \left( 4t + \frac{\sigma(3\elta^2 - \sigma^2)}{2} \right) }\,.
	\label{}
\end{equation}

Or, equivalently, in terms of the $\y$ function \eqref{ysolutionmain}:

\begin{equation}
	J(\sigma,\elta) = (\y+\alpha) j^L + (\y+\tilde{\alpha})j^R + ie^{\kah} \y' e^3 \wedge \bar e^{\bar 3}
	\label{}
\end{equation}

Unfortunately, a closed-form expression for the K\"ahler potential seems impossible. The function $f(t)$ can nevertheless be written in integral form, as such:

\begin{equation}
	f(t;\sigma,\elta) = \int_0^{t} d\ln t' \,\y(t')
	\label{}
\end{equation}

\section{K\"ahler moduli}

As we have seen, a very convenient basis of moduli for the K\"ahler structure is given by the sum and difference $\sigma$ and $\elta$ of the volume of the base spheres defined in \eqref{convenientmoduli}, up to a normalization which we will shortly determine. These correspond respectively to blowing up the two basal 2-cycles (which equates to a blowup of the product 4-cycle of the base) and to an antisymmetric blowing and shrinking of the two $\ess^2$ (a blowup of the difference 2-cycle). We'll examine this geometric structure in more detail in this section.

We will construct the two harmonic forms $\hatt\omega$ and $\tildd\omega$ by differentiating $J$ directly with respect to the relevant moduli, since (see \eqref{JandJ0})

\begin{equation}
	J = J_0 \,+ \, \hatt v \,\hatt \omega + \tildd v \,\tildd \omega
	\label{}
\end{equation}

A basis $(\hatt\omega,\tildd\omega)$ of harmonic two-forms will allow us to also parametrize deformations of the $A_2$ and $B_2$ fields according to \eqref{ABvariation}, in addition to $J$. Thus, the choice of $\omega_a$ reduces to a sensible choice of moduli $\hatt v$, $\tildd v$.

%As an aside, we check all forms obtained in this way from $J$ are primitive: in the $dy^1$, $dy^2$, $e^3$ basis the Kahler form is $\mathrm{diag}(A^L j^L_{1\bar 1}, A^R j^R_{2\bar 2}, i e^k y')$ so the contraction of $J$ with any derivative $\partial_x J$ of it with respect to a parameter is

%\begin{align}
%\begin{split}
%	J^{a\bar b}\partial_x J_{a\bar b} = \mathrm{Tr}\left( (J_{a\bar b})^{-1} \partial_x J_{a\bar b} \right) = A^{-1} \partial_x A + \tilde A^{-1} \partial_x \tilde A + (y')^{-1} \partial_x y' \\= \partial_x \ln \left( A \tilde A y' \right)	
%	\label{}
%\end{split}\end{align}
%
%which vanishes thanks to the Ricci-flatness equation.

As it is clear from our general discussion, we expect to be able to choose $(\hatt\omega,\tildd\omega)$ such that $\hatt\omega$ is normalizable in the sense of \eqref{defM} - this would be the form generating the blowup of the 4-cycle. The other form $\tildd\omega$ will be non-normalizable (but warp-normalizable according to \eqref{defG}) and will correspond to a 2-cycle blowup, the same that was already present in the Klebanov-Witten theory. It is easy to see that the modulus $\hatt v$ relative to $\hatt \omega$ must be proportional to the sum $\sigma$ of the basal volumes, since the corresponding harmonic form $\frac{\partial J}{\partial \sigma}$ is normalizable. To show this, we consider the asymptotic behaviour of the K\"ahler form as $t\rightarrow \infty$. Defined

\begin{equation}
	T := \left( 4t + \frac{\sigma(3\elta^2 - \sigma^2)}{2} \right)
	\label{}
\end{equation}

we have

\begin{align}
	\z \sim T^{1/3}\,, && \z' \sim T^{-2/3}\,,\\	
	\frac{\partial \z}{\partial \sigma} \sim \frac{\partial T}{\partial \sigma} T^{-2/3}\,, && \frac{\partial \z'}{\partial \sigma} \sim \frac{\partial T}{\partial \sigma} T^{-5/3}\,,
\end{align}

where we have omitted constant factors. Therefore we have the following asymptotic behaviour for $J$

\begin{equation}
	J \sim \left(T^{1/3} + \frac{\elta}{2}\right) j^L + \left(T^{1/3} - \frac{\elta}{2}\right)j^R + ie^{\kah} T^{-2/3} e^3 \wedge \bar e^{\bar 3}\,,
	\label{}
\end{equation}

and for its derivative with respect to the sum modulus $\sigma$

\begin{equation}
	\frac{\partial J}{\partial \sigma} \sim \frac{\partial T}{\partial \sigma}\left( T^{-2/3} j^L + T^{-2/3} j^R + i e^{\kah} T^{-5/3}e^3 \wedge \bar e^{\bar 3} \right)\,,
	\label{}
\end{equation}


so that the norm goes as $\left\Vert \pder{J}{\sigma} \right\Vert^2 = \left(\pder{J}{\sigma} \wedge \star \pder{J}{\sigma}\right) \frac{1}{d\vol} \sim t^{-2} \sim r^{-12}$, which is integrable\footnote{Asymptotically, as $t\gg \alpha,\tilde{\alpha}$, the metric reduces to the sharp cone $dr^2 + r^5 ds_5^2$, and in this regime $t \propto r^6$; therefore a function on $X$ is integrable if it decays faster than $r^{-6} \sim t^{-1}$.}.

By contrast, the remaining harmonic form must be nonrenormalizable. For example, differentiating with respect to $\elta$, we obtain

\begin{equation}
	\frac{\partial J}{\partial \elta} \sim j^L - j^R + ie^{\kah} \frac{\partial T}{\partial \elta} T^{-4/3} e^3 \bar e^{\bar 3}
	\label{}
\end{equation}

with norm $ \left\Vert \pder{J}{\elta} \right\Vert \sim t^{-2/3} \sim r^{-4}$, not integrable. We note however that it is still warp-integrable, in accordance with our general discussion in \ref{sec:heftkm}.

Having ascertained we would like $\hatt v \propto \sigma$ and $\tildd v \propto \elta$, one would like to also fix the right normalization for the K\"ahler moduli. For this, note that\footnote{We note that $	\int_{C^1} \pder{J}{\alpha} = \int_{C^1} \pder{(y + \alpha)}{\alpha}\bigg|_{t=0} j^L  = \int_{C^1} j^L = 4\pi
	$, where we've exploited the fact that $y(t=0) = 0$, as it is clear from \eqref{rflatintegrated}. The other three cases are identical.} 
, denoted $C^1$, $C^2$ the basis 2-spheres, and $\alpha^i = (\alpha,\tildd\alpha)$,

\begin{equation}
	\int_{C^i} \pder{J}{\alpha^j} = 4\pi\, \delta_{ij}
	\label{}
\end{equation}

%\begin{equation}
%	\int_{C^1} \pder{J}{\alpha} = \int_{C^1} \pder{J}{\tilde\alpha} = 0\, \quad\int_{C^2} \pder{J}{\tilde\alpha} = 4\pi\,. 
%	\label{}
%\end{equation}

These are also the intersection number of $C^1$, $C^2$ with the Poincar\'e dual 4-cycles of these forms $\pder{J}{\alpha^i}$; since the $C^i$ form a basis, a pair 4-cycles with the same intersection numbers will necessarily belong the dual classes. We consider the (noncompact) 4-cycles $D^1$, $D^2$ given respectively by the fibres of $C^1$, $C^2$, recalling the cone is a $\C\rightarrow \mathbb{CP}^1 \times \mathbb{CP}^1$ bundle. Since it is easily seen that

\begin{equation}
	D^1 \cdot C^1 = 0\,,\quad D^1 \cdot C^2 = 1 \,, \quad D^2 \cdot C^1 = 1 \,, \quad D^2 \cdot C^2 = 0\,,
\end{equation}

then we can identify the Poincar\'e duals:

\begin{align}
	- \frac{1}{4\pi} \pder{J}{\alpha} \; \longleftrightarrow \; D^2 \, \\ - \frac{1}{4\pi}\pder{J}{\tildd\alpha} \; \longleftrightarrow \; D^1\,.
	\label{}
\end{align}

Then a useful normalization for $\hatt\omega$ would be

\begin{align}
	\hatt\omega = \omega_1 = \frac{1}{2\pi}\left( \pder{J}{\sigma} \right) = \pder{J}{\hatt v}\,, \quad\quad \text{with } \hatt v := 2\pi \sigma
	\label{}
\end{align}

which makes it so $\int_{C^i} \hatt\omega = 2$ is integer. The dual to $\hatt \omega $ is 

\begin{equation}\label{defE}
	E := -2(D^1+D^2)\,;
\end{equation}

it is easy to show this is actually the base $\mathbb{CP}^1 \times \mathbb{CP}^1$. Thus, not only does $\hatt\omega$ generate the blowup of the 4-cycle made by the product of the two basal spheres $E = \ess^2 \times \ess^2 $ (and $\hatt v$ parametrizes its volume), $\hatt\omega$ and $E$ are actually Poincar\'e dual.

\begin{figure}[H]
\centering
\def\svgwidth{100pt}
\captionsetup{width=0.8\textwidth}
\input{images/bundlescheme.pdf_tex}
\caption{Schematic representation of the resolved $X^{2,0}$ as a line bundle, with the relevant 2- and 4-cycles.}
\end{figure}

Similarly, we choose

\begin{align}
	\tildd \omega = \omega_2 = \frac{1}{2\pi} \left( \pder{J}{\elta} \right) = \pder{J}{\tildd v} && \tildd v := 2\pi\elta
	\label{choiceoftildev}
\end{align}

dual to the 4-cycle 

\begin{equation}
	F := 2 (D^1 - D^2)\,.
	\label{defF}
\end{equation}

Actually, there is an inherent arbitrariness in the non-renormalizable form $\tildd\omega$ as it could be shifted by an arbitrary multiple of the normalizable form $\hatt\omega$. Still, we stand by choice \eqref{choiceoftildev} for $\tildd v$ and $\tildd\omega$ as it proves to be convenient for explicit calculations.


This choice for the harmonic 2-forms and the moduli $v^a$ allows for easy computation of the intersection numbers:

\begin{align}
	I_0 = \int \hatt\omega \wedge \hatt\omega \wedge \hatt\omega = E \cdot E \cdot E  = 8 \\
	I_1 = \int \hatt\omega \wedge \hatt\omega \wedge \tildd\omega = E \cdot E \cdot F  = 0\\
	I_2 = \int \hatt\omega \wedge \tildd\omega \wedge \tildd\omega = E \cdot F \cdot F = -8
	\label{}
\end{align}


\section{Chiral fields and effective Lagrangian}

Now that we have parametrized the K\"ahler structure and identified a basis of harmonic 2-forms, we are now able to specify the chiral content of the effective theory, and its Lagrangian, putting to use the results detailed in sections \ref{sec:heftchiral}, \ref{sec:heftlagrangian}.

The HEFT will feature the following $3N + 3$ chiral fields:

\begin{center}\begin{tabular}{c | l c}
	$z_I^i$ & $= (y_I^1, y_I^2, \zeta_I)$ & D3-brane positions on $X$\\
	$\hatt\rho = \rho_1$ & related to $\hatt v$ & 4-cycle blowup deformation of $X$\\
	$\tildd\rho = \rho_2$ & related to $\tildd v$ & 2-cycle blowup deformation of $X$\\
	$\beta$ &  & $C_2 - \tau B_2$ normalizable deformation
\end{tabular}\end{center}


and the following $2$ non-dynamical chiral parameters:

\begin{center}
\begin{tabular}{c | c}
	$\lambda$ &  $C_2 - \tau B_2$ non-renormalizable deformation\\
	$\tau$ &  axio-dilaton
\end{tabular}\end{center}

We can match these directly with the field theory objects discovered in section \ref{sec:squares}. $z_I^i$ map directly with the $3N$ mesonic VEVs parametrizing $\mmes$. $\hatt\rho$, $\tildd\rho$ match with the two VEVs of baryons generating the resolution of the cone as in section \ref{sec:squaresmoduli}, while $\beta$ is the third ``non-geometric'' baryonic modulus. The axio-dilaton $\tau$ is dual to the marginal coupling relative to the sum of the gauge couplings $\tau_1 + \tau_2 + \tau_3 + \tau_4$, while $\lambda$ correspond to the second marginal deformation found in section \ref{sec:squaresmarginal}.

The chiral fields $\rho_a = (\hatt \rho,\tildd{\rho})$ are related to the moduli $v_a = (\hatt v, \tildd v)$ by the transform \eqref{rhov} described in the previous chapter; as anticipated we will only need to specialize the precise form of the real part of $\rho_a(v_a)$:

\begin{align}
\begin{split}
	\Re \hatt \rho &= \frac{1}{2} \sum_I \hatt\kappa(z_I,\bar z_I ; v)  - \frac{1}{2\Im \tau} I_0 (\Im \beta)^2 - \frac{1}{\Im \tau} I_1 \Im \beta \Im \lambda\\
	&= \frac{1}{2} \sum_I \hatt\kappa(z_I,\bar z_I; v) -\frac{4}{\Im\tau} \left( \Im\beta \right)^2
	\label{rhofunv}
\end{split}
\end{align}
\begin{align}
\begin{split}
	\Re \tildd{\rho} &= \frac{1}{2} \sum_I \tildd\kappa(z_I,\bar z_I; v) - \frac{1}{2 \Im \tau}I_1(\Im\beta)^2-\frac{1}{\Im\tau}I_2 \Im\beta \Im\lambda\\
	&= \frac{1}{2} \sum_I \hatt\kappa(z_I,\bar z_I; v) + \frac{8}{\Im\tau} \Im\lambda \Im\beta 
	\label{}
\end{split}
\end{align}

where $\kappa_a(z_I,\bar z_I; v) = (\hatt\kappa,\tildd\kappa)$ are defined as the potentials that generate the $\omega_a = (\hatt\omega, \tildd\omega)$, as in

\begin{equation}
	\omega_a = i \partial\bar \partial \kappa_a
	\label{defkappamain}
\end{equation}

and also satisfy the following asymptotic condition for their derivatives:

\begin{equation}
	\pder{\kappa_a}{v_a} \sim r^{-k} \sim t^{-k/6}, \quad k \geq 2
	\label{potentialcondition}
\end{equation}


We are now able to present the bosonic part of the effective Lagrangian. This holds in a generic moduli space point where no D3-branes coincide, therefore there is first of all a decoupled sector of $N$ copies of $U(1)$ SYMs, the normal abelian gauge theory each D-brane hosts\footnote{Again, in the case of $n$ branes coinciding, a decoupled $\ssn=4$ $SU(n)$ Yang-Mills sector appears.}. Then, the rest of the bosonic effective Lagrangian describes the chiral fields listed above:

\begin{equation}
	\mathcal{L}_\mathrm{chiral} = - \pi \mathcal{G}^{ab} \nabla \rho_a \wedge \star \nabla \bar {\rho_b} - 2 \pi \sum_I J_{i\bar j} dz^i d\bar{z}^{\bar j} - \frac{\pi\mathcal{M}}{\Im \tau} d\beta \wedge \star d\bar\beta
	\label{y20lagrangian}
\end{equation}

where the kinetic factors are computable as follows (using \eqref{Gshortcut}):

\begin{equation}
	\mathcal{G}_{ab} = \int_X e^{-4A} \omega_a \wedge \star \omega_b = - \int_X e^{-4A} J \wedge \omega_a \wedge \omega_b = - \frac{\partial \Re \rho_a}{\partial v_b}
	\label{}
\end{equation}

\begin{equation}
	\mathcal{M} = \int_X \hatt \omega \wedge \star \hatt \omega = - \int J \wedge \hatt \omega \wedge \hatt \omega = - \hatt v I_0 = 8 \hatt v
	\label{Mfinal}
\end{equation}

($\mathcal G^{ab}$ being of course the inverse matrix of $\mathcal G_{ab}$) and the covariant derivative $\nabla$ is

\begin{align}
	\nabla \hatt \rho &= d \hatt \rho - \mathcal{A}_{1i}^{I} dz_I^i - \frac{8i}{\Im \tau} \Im \beta \, d\beta\\
	\nabla \tildd \rho &= d \tildd \rho - \mathcal{A}_{2i}^{I} dz_I^i + \frac{8i}{\Im \tau} \Im \lambda \, d\beta
	\label{}
\end{align}

\begin{equation}
	\mathcal{A}_{ai}^I = \pder{\kappa_a(z_I,\bar z_I; v)}{z_I^i}
	\label{}
\end{equation}

To compute the coefficients $\mathcal{G}$,$\mathcal{A}$ we first determine the form of the $\kappa$ potentials.

\section{$\kappa$ potentials}

The harmonic forms $\hatt\omega$, $\tildd\omega$ must be generated by potentials $\hatt\kappa$, $\tildd\kappa$ as in \eqref{defkappamain}. These potentials are necessary to compute the transformation from the K\"ahler moduli $v^a$ to the chiral fields $\rho_a$.

In accord to what was discussed in section \ref{sec:heftkm}, since 

\begin{equation}
J_0 =  i \partial\bar \partial k_0
\end{equation}

and so

\begin{equation}
	\omega_a = \pder{J}{v^a} = i \partial \bar \partial \,\pder{k_0}{v^a}\,
	\label{}
\end{equation}

and we would like $\omega_a = i \partial\bar \partial \kappa_a$, it must be that 

\begin{equation}
\kappa_a = \pder{k_0}{v^a} + h(v)
\end{equation}

with $h(v)$ an arbitrary function of the moduli which would then be fixed as to satisfy the condition \eqref{potentialcondition} (up to an additive constant). However, as will be seen shortly, $\pder{k_0}{v^a}$ itself satisfies the asymptotic condition, so that $h(v)$ is actually a constant, which we will omit.\\

Recalling (see \eqref{y20potentialansatz}) $k_0 = f(t) + \frac{\sigma+\elta}{2} k^L + \frac{\sigma-\elta}{2} k^R$, and $f(t) = \int_0^t d \, \ln(t') \y(t')$, we find

\begin{align}
 	2\pi \, \hatt\kappa(t;\sigma,\elta) =	\pder{k_0}{\sigma} & = \left( \int d\, \ln t' \pder{y}{\sigma} \right) + \frac{1}2 k^L + \frac{1}2 k^R \\
	2\pi \, \tildd\kappa(t;\sigma,\elta) =  \pder{k_0}{\elta} & = \left( \int d\, \ln t' \pder{y}{\elta} \right) + \frac{1}2 k^L - \frac{1}2 k^R
	\label{}
\end{align}


so that the derivatives of the $\kappa$ potentials become

\begin{equation}
\arraycolsep=1.4pt\def\arraystretch{2.2}
\pder{\kappa_a}{v^b} = \frac{\partial^2 k_0}{\partial v^a \partial v^b} = \frac{1}{4 \pi^2} \int_0^t d\, \ln(t')
	\begin{pmatrix}
		\frac{\partial^2 y}{\partial \sigma^2} & \frac{\partial^2 y}{\partial \sigma \partial \elta} \\
		\frac{\partial^2 y}{\partial \sigma\partial \elta} & \frac{\partial^2 y}{\partial \elta^2}
	\end{pmatrix}_{ab}
	\label{}
\end{equation}

The explicit forms of the second derivatives of the $y$ function, rather convoluted, are listed in appendix \ref{sec:yderivatives}. It is clear they have at most asymptotic behaviour $\sim t^{-2/3}$, which will be the same as that of their $\int d\,\ln t'$, so that the $\kappa_a$ defined above satisfy \eqref{potentialcondition} and no addition of a function of the moduli $h(v)$ is necessary.

Then, this allows immediately for the computation of the $\mathcal{G}_{ab}$ matrix:

\begin{equation}
	\mathcal{G}_{ab} = - \pder{\Re \rho_a}{ v^b}  = - \sum_I \pder{ \kappa_a(z_I,\bar z_I; v)}{v^b} = - \frac{1}{4\pi^2} \sum_I \int_0^{t_I} d\,\ln t' \frac{\partial^2 y}{\partial v^a \partial v^b} (t' ; v)
	\label{Gfinal}
\end{equation}

again resting on the explicit form of the second derivatives of $\y$. The integrals are not solvable in closed form. The matrix will always be invertible and its inverse $\mathcal{G}^{ab}$ is the kinetic matrix for the $\rho$ fields.

The connection $\mathcal{A}_{ai}^I$ instead can be found more explicitly. We treat the $z_I^3 = \zeta_I$ and $z_I^{1,2} = y^{L,R}$ cases separately.

\begin{equation}
	\mathcal{A}_{aI}^i = \frac{\partial^2 k_0}{\partial \zeta_I \partial v^a} = \frac{\partial^2 f}{\partial \zeta_I \partial v^a}
	\label{}
\end{equation}

but, recalling $t = |\zeta|^2 e^{\kah}$, $\pder{f(t)}{\zeta} = \bar \zeta e^{\kah} f'(t) =\bar \zeta e^{\kah} \y(t)/t = (\bar \zeta)^{-1} \y(t)$ so that this is simply

\begin{equation}
	= \bar \zeta^{-1} \pder{y}{v^a}	\label{}
\end{equation}

and

\begin{align}
	\mathcal{A}_{1I}^3 =& \frac{1}{2\pi} \bar \zeta^{-1} \pder {y}\sigma\\
	\mathcal{A}_{2I}^3 =& \frac{1}{2\pi} \bar \zeta^{-1} \pder {y}\elta
	\label{Afinal3}
\end{align}

The $i=1,2$ components, instead, are

\begin{equation}
	\mathcal{A}_{aI}^i = \frac{\partial^2 k_0}{\partial y_I^i \partial v^a} = \frac{\partial^2 (\alpha^i k^i)}{\partial y_I^i \partial v^a}  = \pder{\alpha^i}{v^a} \pder{k^i}{y_I^i}
	\label{Afinal12}
\end{equation}

(no summation on $i$ is implied), so essentially:

\begin{align}
	\mathcal{A}_{1I}^1 = \mathcal{A}_{2I}^1 = & \frac{1}{4\pi} \pder{k^L}{y^L} \\
	\mathcal{A}_{1I}^2 = - \mathcal{A}_{2I}^2 = & \frac{1}{4\pi} \pder{k^R}{y^R} 
	\label{}
\end{align}

This concludes the computation of the kinetic terms and couplings in the effective theory. While these results are, unfortunately, implicit, they encode exactly the dynamics of the HEFT.

\section{Baryonic VEVs}

We now sketch how the moduli $\hatt \rho$, $\tildd\rho$, $\beta$ reconnect with the VEVs of the fundamental baryonic operators $\mathcal{B}^{A,B,C,D}$ defined in section \ref{sec:squaresmoduli}. In \cite{MZ}, these VEVs are computed starting from the non-perturbative contributions from Euclidean D3-branes (E3-branes) wrapping four-cycles, and it is found that for a generic baryonic o\-pe\-ra\-tor~$\mathcal{B}$

\begin{equation}
	\langle \mathcal{B} \rangle \propto \mathcal{A}(\beta) \prod_I \zeta_D(z_I) \, e^{-2\pi n^\alpha \rho_\alpha}
	\label{baryonasbaryons}
\end{equation}

$D$ is the divisor around which the E3-instanton wraps, and $\zeta_D(z)$ is the section such that $D = \{z :  \zeta(z) = 0\}$. $n^a$ are the coefficients in the expansion of $D$ in the basis given by the duals of $(\hatt\omega,\tildd\omega)$:

\begin{equation}
	D = \hatt n E + \tildd n F\,;
	\label{}
\end{equation}

where $E$ and $F$ are defined as in \eqref{defE}, \eqref{defF}.

$\mathcal{A}(\beta)$ is a holomorphic function, actually a theta function of $\beta$. The proportionality factor in \eqref{baryonasbaryons} depends only on the marginal parameters $\tau$, $\lambda$.

Now, parametrize the basal spheres $\cpone_L$, $\cpone_R$ with projective coordinates $[z_L^1 : z_L^2]$ and $[z_R^1 : z_R^2]$. Then the two families of divisors

\begin{align}
	\zeta_{[a_1:a_2]} (z) &:= a_1 z_L^1 + a_2 z_L^2 = 0\\
	\tildd\zeta_{[b_1:b_2]} (z) &:= b_1 z_R^1 + b_2 z_R^2 = 0
	\label{}
\end{align}

can be easily mapped respectively with $\B^A$, $\B^C$ and $\B^B$, $\B^D$. Topologically, they are simply $D^2$ and $D^1$ respectively. Since $D^1 = \frac{1}{4}(F-E)$ and $D^2 = -\frac{1}{4}(F+E)$ this means \eqref{baryonasbaryons} can be specialized as

\begin{align}
	\begin{split}
	\langle\B^A\rangle &\propto \mathcal{A}(\beta)\; \prod_I \zeta_{[a_1:a_2]} (z_I) \; e^{\frac{\pi}{2} (\hatt\rho + \tildd \rho)}\\	
	\langle\B^B\rangle &\propto \mathcal{A}(\beta)\; \prod_I \tildd\zeta_{[b_1:b_2]} (z_I) \; e^{\frac{\pi}{2} (\hatt\rho - \tildd \rho)}\\
	\langle\B^C\rangle &\propto \tildd{\mathcal{A}}(\beta)\; \prod_I \zeta_{[a_1:a_2]} (z_I) \; e^{\frac{\pi}{2} (\hatt\rho + \tildd \rho)}\\
	\langle\B^D\rangle &\propto \tildd{\mathcal{A}}(\beta)\; \prod_I \tildd\zeta_{[b_1:b_2]} (z_I) \; e^{\frac{\pi}{2} (\hatt\rho - \tildd\rho)}
	\label{BasRhos}\end{split}
\end{align}

The difference between the $\B^A$ and $\B^C$ VEVs is that the D-brane gauge field is asymptotic to two different classes of the torsional part of $H_3(Y)$, which is $\mathbb{Z}_2$; this reflects in different functions $\mathcal{A}$, $\tildd{\mathcal{A}}$ of $\beta$. An identical relationship holds between $\B^B$ and $\B^D$. 

The $z_I$-dependent piece is actually a homogeneous degree-$N$ polynomial in $a_1,a_2$ (or $b_1,b_2$). The resulting $N+1$ coefficients (for the monomials $a_1^N$, $a_1^{N-1} a_2$, \ldots) correspond naturally with the $N+1$ baryons in each class.

This rough identification will be essential in translating the symmetries of the quiver theory into the effective theory.

\section{Comments on the effective theory}

We summarize the results obtained. The HEFT for the $Y^{2,0}$ model will be an $\ssn = 1$ field theory, with chiral superfields $\hatt \rho$, $\tildd \rho$, $\beta$, and Lagrangian given by \eqref{y20lagrangian}. The kinetic matrices $\mathcal{G}$, $\mathcal{M}$, and connection $\mathcal{A}$ are given respectively in \eqref{Gfinal}, \eqref{Mfinal}, \eqref{Afinal3} and \eqref{Afinal12}. These quantities are expressed in terms of the $y$ function and its derivatives with respect to the moduli; these are listed in appendix \ref{sec:yderivatives}.

We note part of the complexity of the effective Lagrangian computed in this chapter is due to our insistence in writing it explicitly. In fact, as it was seen in section \ref{sec:heftlagrangian}, there is a much more compact, though opaque, formulation in terms of a simple implicit K\"ahler potential $K$, function of $(\hatt\rho,\tildd\rho,\beta,z_I^i)$ and their conjugates. The Lagrangian \eqref{y20lagrangian} can then be reformulated as a superspace integral:

\begin{equation}
	\mathcal{L}_\mathrm{chiral} = \int d^4 \theta K
	\label{}
\end{equation}

and $K$ is, according to \eqref{heftkahler}:

\begin{gather}
	K = 2\pi \sum_I k_0(z_I, \overline{z}_I,\hatt v, \tildd v) \\
	= \sum_I \left(\, 2\pi f(t_I) \;+\; \hatt v\, (k^L + k^R) + \tildd v\,(k^L - k^R)\, \right)\,,
	\label{effectivekahler}
\end{gather}

which is deceivingly simple-looking, since the moduli $(\hatt v,\tildd v)$ are complicated functions of the fields $(\hatt \rho,\tildd \rho)$.

As a sanity check, we verify the effective theory found implements the original symmetries of the $Y^{2,0}$ theory. 

For example, the action of the flavour symmetry $SU(2)_L \times SU(2)_R$ is evident: it is the isometry group of $\cpone \times \cpone \cong \ess^2 \times \ess^2$. $SU(2)_L$ acts on $y^L_I$ as rotations\footnote{Very explicitly, if $y$ is a stereographic coordinate on $\cpone$, then $SU(2)$ matrices $\begin{pmatrix}a & b\\ c & d\end{pmatrix}$ act on $y$ directly as fractional transformations: $y \rightarrow (ay+b)/(cy+d)$.}, moving the positions of the D-branes on the left $\cpone$; similarly $SU(2)_R$ acts on $y^R_I$. All the other fields, $\zeta_I$, $\rho_a$, $\beta$ are untouched.

A less trivial example concerns symmetries that are spontaneously broken in the generic moduli space point, like the superconformal group. There must be a non-linear implementation of the broken generators in the effective theory. Let us verify this explicitly for dilations. For the action to be scale-invariant, the K\"ahler potential $K$ must scale with dimension $\Delta_K = 2$. 

Clearly, the basal coordinates $y^i_I$ cannot transform under scaling, and neither do the basal K\"ahler potentials $k^i$. Thus, $\Delta_K = 2$ can occur only if we assign, consistently with \eqref{effectivekahler}, the scaling dimensions

\begin{equation}
	\Delta_{\hatt v} = \Delta_{\tildd v} = \Delta_f = \Delta_K = 2
	\label{}
\end{equation}


Now, since $f = \int d\ln t' \, \y(t';v^a)$, then $f(t;v^a)$ scales like $\y(t;v^a)$; we recall $\y$ is schematically (see \eqref{ysolutionmain})

\begin{equation}
	\y \sim \tildd v F\left( \frac{t}{{\tildd v}^3} \right) - \hatt v\,,
	\label{}
\end{equation}

where $F$ is a non-homogeneous function. For this to scale with dimension $2$ it is clear we should have $t_I$ scaling as $\tildd v^3$, or $\zeta_I \sim {\tildd v}^{3/2} \Rightarrow \Delta_\zeta = 3$. With the given assignment of scaling dimensions the theory is clearly scale-invariant. However, we still need to specify the (non-linear) action of dilations on the chiral fields $(\hatt\rho,\tildd\rho,\beta)$, instead of the K\"ahler moduli $v^a$. 

We exploit the expressions \eqref{BasRhos} for the VEVs of fundamental baryons. We note that since the four fundamental chirals $A$, $B$, $C$, $D$ all scale with $\Delta = \frac{3}{4}$, the fundamental baryons $\mathcal{B}^A$, $\mathcal{B}^B$, $\mathcal{B}^C$, $\mathcal{B}^D$ have well-defined dimension $\frac{3N}{4}$. Moreover, the $z$-dependent part in \eqref{BasRhos} does not scale, as it only depends on the $y^i_I$. 

Finally, the scaling can not come from $\mathcal{A}(\beta)$, $\tildd{\mathcal{A}}(\beta)$ and in fact it is easily shown that $\Delta_\beta = 0$. This means that

\begin{equation}
	\Delta \left( e^{\frac{\pi}{2} (\hatt \rho + \tildd \rho)} \right) = \frac{3N}{4}\,,\quad
	\Delta \left( e^{\frac{\pi}{2} (\hatt \rho - \tildd \rho)} \right) = \frac{3N}{4}\,;
	\label{}
\end{equation}

that is:

\begin{equation}
	\Delta\left( e^{\pi\hatt\rho} \right) = \frac{3N}{2}\,,\quad \Delta\left( e^{\pi\tildd\rho} \right) = 0\,.
\end{equation}

Thus, only these particular exponentials of $\rho_a$ are dilation eigenstates. Dilations therefore act like shifts of $\Re \hatt \rho$, $\Re \tildd \rho$, which is not $\mathbb{C}$-linear.

There is yet another simple geometric symmetry. The resolved cone is a $\C \rightarrow \cpone \times \cpone$ fibration, thus also trivially a $U(1)$ fibration considering the phase $\psi$ of the fibral coordinate $\zeta$. This means there is a simple symmetry of the effective theory consisting in the rotation of the position of the branes along each $U(1)$ fiber:

\begin{equation}
	\zeta_I \rightarrow e^{i\theta} \zeta_I\,, \quad y^i_I \rightarrow y^i_I\,, \quad v^a \rightarrow v^a\,,\quad \beta \rightarrow \beta\,.
	\label{}
\end{equation}

It is easy to recognize this the action of R-charge (up to constants). The K\"ahler potential \eqref{effectivekahler} does not depend on the phase of $\zeta_I$ and thus has vanishing R-charge. Therefore the field theory R-charge survives in the holographic effective field theory as a global symmetry. The action of this symmetry on the $\rho_a$ is again unusual. We note that the fundamental baryons $\B^A, \ldots$ have R-charge $\frac{N}{2}$. Thus, according to \eqref{BasRhos}, under R-charge we should have the transformation law

\begin{equation}
	e^{\pi \hatt\rho} \rightarrow e^{-i\frac{N}{2}\theta} e^{\pi\hatt \rho}\,,\quad e^{\pi\tildd\rho} \rightarrow e^{\pi\tildd\rho}
	\label{}
\end{equation}

The bottom line is that it is the combinations $e^{\pi \hatt\rho}$, $e^{\pi\tildd\rho}$ that have definite R-charge. An R-transformation is therefore implemented as a shift in the imaginary part of $\rho_a$, which is related, as it was seen in section \ref{sec:heftchiral}, to moduli of the $C_4$ form. This is somewhat expected, since the D3-branes being rotated along the $U(1)$ fiber carry $F_5$ charge; we do not investigate this aspect further, however.

The fact that the superpotential $K$ has $\Delta_K = 2$ and $R_K = 0$ (under some implementation of dilations and R-charge we have provided) is actually sufficient data \cite{terningmodernsusy} to deduce the effective theory has full $\ssn = 1$ superconformal symmetry - though non-linearly implemented.

Finally, we can also verify the implementation of the three non-geometric $U(1)$ symmetries described in section \ref{sec:squaresuones}. Being non-geometric, these will not act on $z_I^i$, but only on the baryonic moduli $\hatt\rho$, $\tildd\rho$, $\beta$. Take the non-anomalous baryon number $U(1)_B$, and consider the operator VEVs

\begin{equation}
	\langle\B^A \B^A\rangle \propto \mathcal{A}^2(\beta) e^{\pi\hatt\rho}\,,\quad 
	\langle{\B^C\B^D}\rangle \propto \tildd{\mathcal{A}}^2(\beta) e^{\pi\hatt\rho}\,.
	\label{}
\end{equation}

Both these operators have baryon number $N-N=0$. Therefore, both $\hatt\rho$ and $\beta$ are uncharged under $U(1)_B$. Instead, since

\begin{equation}
	\langle \B^A \rangle \propto e^{\pi\tildd\rho}
	\label{}
\end{equation}

has baryon number $N$, then it's clear $U(1)_B$ acts as shifts of $\Im \tildd\rho$. This is evidently a symmetry of Lagrangian \eqref{y20lagrangian}, since the latter does not depend on the imaginary part of $\tildd\rho$.

Analogously, we have two extra symmetries relative to the shifts of $\Im \beta$, $\Im \hatt\rho$, since these also do not appear in the effective Lagrangian. These are immediately matched to the two anomalous $U(1)_\text{AN,1}$, $U(1)_\text{AN,2}$ symmetries.

\section{Conclusions}

We take the reappearance of the symmetries of the field theory as non-trivial evidence our effective description is accurate. Thus, the main objective of this thesis, the determination of the exact effective Lagrangian for the $Y^{2,0}$ theory at a generic moduli space point has been successfully reached. A posteriori, we can now understand the significance of this specific result: the $Y^{2,0}$ is the absolutely simplest theory for which all possible distinct features of a holographic effective field theory are present. The theory includes one single chiral field from each class (normalizable K\"ahler modulus, non-normalizable K\"ahler modulus, 2-form-field modulus), one extra marginal parameter, one non-anomalous $U(1)$ and two anomalous $U(1)$s (the minimal non-zero number, since this is $2b_4(X)$). Thus, this model is a perfect testbed for the general techniques described in \cite{MZ} for constructing HEFTs, and the present work constitutes evidence the techniques presented therein are valid.

Nevertheless, it is important to keep in mind the range of applicability of this result. As seen in section \ref{sec:Maldacena}, the holographic dual to the quiver field theory is correctly modeled by supergravity only in the strong-coupling limit, and even then only by classical, or equivalently weakly-coupled supergravity in the large $N$ limit. This is therefore inherently a perturbative result, and one should be concerned about the possibility of unaccounted non-perturbative phenomena. We give a relevant example pertaining the question of the anomalous $U(1)$s: these are two anomalous rigid symmetries of the microscopic field theory, yet appear as exact symmetries of the effective Lagrangian.

Indeed, these symmetries are broken by non-perturbative string effects on the bulk side. E1- and E3-branes, respectively two- and four-dimensional instantons, can wrap around two- and four-cycles $D_2$, $D_4$ in the cone $X_6$. Their (Euclideanized) action is schematically (see section \ref{sec:dbraneaction})

\begin{align}
\begin{split}
	S_{E1} = \frac{1}{g_s}\int_{D_2} d\vol + \int_{D_2} C_2\,,\\
	S_{E3} = \frac{1}{g_s}\int_{D_4} d\vol + \int_{D_4} C_4\,,
	\label{epaction}
\end{split}
\end{align}

i.e., the volume of the worldvolume, plus a coupling to the relevant RR potential. As instantons, they contribute to amplitudes a factor

\begin{equation}
	e^{-S_{Ep}} = \exp\left({-\frac{1}{g_s}\vol(D_{p+1}) - \int_{D_{p+1}} C_{p+1} }\right)\,.
\end{equation}

The $e^{-1/g_s}$ dependence, invisible to a series expansion in the string coupling, establishes this is a non-perturbative effect. In the specific case of the $Y^{2,0}$, we note integrals of the RR forms around cycles are related to the imaginary part of the baryonic moduli and the marginal parameter $\lambda$:

\begin{align}
\begin{split}
	\Im \lambda \sim \int_{P} C_2\,,\quad	\Im \beta \sim \int_{S} C_2\,, \\
	\Im \hatt\rho \sim \int_{E} C_4\,,\quad	\Im \tildd \rho \sim \int_{F} C_4\,.
\end{split}
\end{align}

Here, $E$ and $F$ are the compact and non-compact 4-cycles respectively Poincar\'e dual to $\hatt\omega$, $\tildd\omega$. $P$ and $S$ are the two-cycles arising from the decomposition \eqref{decomphomo}; in particular $S$ corresponds to the four-cycle $E$ and $P$ to the $\mathbb{S}^3$ 3-cycle of $Y_5$, according to that relationship. (We note $S$ is just the difference $\cpone_L - \cpone_R$).

Then, evidently the action \eqref{epaction} for an E-brane wrapping a cycle $D$ is not invariant under shifts of the imaginary part of the corresponding baryonic modulus. Therefore the $\Im\beta$ shift symmetry is broken non-perturbatively by E1-branes wrapping $S$, and the $\Im\hatt\rho$ by E3-branes wrapping $E$.

Instead, the baryonic symmetry given by shifts of $\Im\tildd\rho$ is protected from this effect because the 4-cycle $F$ is non-compact, thus no E3-brane with finite volume, and thus action, can be wrapped on it. This matches perfectly with the vanishing of the $U(1)_B$ anomaly in the holographic dual.

